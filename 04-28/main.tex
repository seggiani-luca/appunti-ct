
\documentclass[a4paper,11pt]{article}
\usepackage[a4paper, margin=8em]{geometry}

% usa i pacchetti per la scrittura in italiano
\usepackage[french,italian]{babel}
\usepackage[T1]{fontenc}
\usepackage[utf8]{inputenc}
\frenchspacing 

% usa i pacchetti per la formattazione matematica
\usepackage{amsmath, amssymb, amsthm, amsfonts}

% usa altri pacchetti
\usepackage{gensymb}
\usepackage{hyperref}
\usepackage{standalone}

% imposta il titolo
\title{Appunti Fondamenti di Automatica}
\author{Luca Seggiani}
\date{2025}

% disegni
\usepackage{pgfplots}
\pgfplotsset{width=10cm,compat=1.9}

% imposta lo stile
% usa helvetica
\usepackage[scaled]{helvet}
% usa palatino
\usepackage{palatino}
% usa un font monospazio guardabile
\usepackage{lmodern}

% tikz in sans
\tikzset{every picture/.style={/utils/exec={\sffamily}}}

\renewcommand{\rmdefault}{ppl}
\renewcommand{\sfdefault}{phv}
\renewcommand{\ttdefault}{lmtt}

% circuiti
\usepackage{circuitikz}
\usetikzlibrary{babel}

% disponi il titolo
\makeatletter
\renewcommand{\maketitle} {
	\begin{center} 
		\begin{minipage}[t]{.8\textwidth}
			\textsf{\huge\bfseries \@title} 
		\end{minipage}%
		\begin{minipage}[t]{.2\textwidth}
			\raggedleft \vspace{-1.65em}
			\textsf{\small \@author} \vfill
			\textsf{\small \@date}
		\end{minipage}
		\par
	\end{center}

	\thispagestyle{empty}
	\pagestyle{fancy}
}
\makeatother

% disponi teoremi
\usepackage{tcolorbox}
\newtcolorbox[auto counter, number within=section]{theorem}[2][]{%
	colback=blue!10, 
	colframe=blue!40!black, 
	sharp corners=northwest,
	fonttitle=\sffamily\bfseries, 
	title=Teorema~\thetcbcounter: #2, 
	#1
}

% disponi definizioni
\newtcolorbox[auto counter, number within=section]{definition}[2][]{%
	colback=red!10,
	colframe=red!40!black,
	sharp corners=northwest,
	fonttitle=\sffamily\bfseries,
	title=Definizione~\thetcbcounter: #2,
	#1
}

% disponi problemi
\newtcolorbox[auto counter, number within=section]{problem}[2][]{%
	colback=green!10,
	colframe=green!40!black,
	sharp corners=northwest,
	fonttitle=\sffamily\bfseries,
	title=Problema~\thetcbcounter: #2,
	#1
}

% disponi codice
\usepackage{listings}
\usepackage[table]{xcolor}

\lstdefinestyle{codestyle}{
	backgroundcolor=\color{black!5}, 
	commentstyle=\color{codegreen},
	keywordstyle=\bfseries\color{magenta},
	numberstyle=\sffamily\tiny\color{black!60},
	stringstyle=\color{green!50!black},
	basicstyle=\ttfamily\footnotesize,
	breakatwhitespace=false,         
	breaklines=true,                 
	captionpos=b,                    
	keepspaces=true,                 
	numbers=left,                    
	numbersep=5pt,                  
	showspaces=false,                
	showstringspaces=false,
	showtabs=false,                  
	tabsize=2
}

\lstdefinestyle{shellstyle}{
	backgroundcolor=\color{black!5}, 
	basicstyle=\ttfamily\footnotesize\color{black}, 
	commentstyle=\color{black}, 
	keywordstyle=\color{black},
	numberstyle=\color{black!5},
	stringstyle=\color{black}, 
	showspaces=false,
	showstringspaces=false, 
	showtabs=false, 
	tabsize=2, 
	numbers=none, 
	breaklines=true
}

\lstdefinelanguage{javascript}{
	keywords={typeof, new, true, false, catch, function, return, null, catch, switch, var, if, in, while, do, else, case, break},
	keywordstyle=\color{blue}\bfseries,
	ndkeywords={class, export, boolean, throw, implements, import, this},
	ndkeywordstyle=\color{darkgray}\bfseries,
	identifierstyle=\color{black},
	sensitive=false,
	comment=[l]{//},
	morecomment=[s]{/*}{*/},
	commentstyle=\color{purple}\ttfamily,
	stringstyle=\color{red}\ttfamily,
	morestring=[b]',
	morestring=[b]"
}

% disponi sezioni
\usepackage{titlesec}

\titleformat{\section}
{\sffamily\Large\bfseries} 
{\thesection}{1em}{} 
\titleformat{\subsection}
{\sffamily\large\bfseries}   
{\thesubsection}{1em}{} 
\titleformat{\subsubsection}
{\sffamily\normalsize\bfseries} 
{\thesubsubsection}{1em}{}

% disponi alberi
\usepackage{forest}

\forestset{
	rectstyle/.style={
		for tree={rectangle,draw,font=\large\sffamily}
	},
	roundstyle/.style={
		for tree={circle,draw,font=\large}
	}
}

% disponi algoritmi
\usepackage{algorithm}
\usepackage{algorithmic}
\makeatletter
\renewcommand{\ALG@name}{Algoritmo}
\makeatother

% disponi numeri di pagina
\usepackage{fancyhdr}
\fancyhf{} 
\fancyfoot[L]{\sffamily{\thepage}}

\makeatletter
\fancyhead[L]{\raisebox{1ex}[0pt][0pt]{\sffamily{\@title \ \@date}}} 
\fancyhead[R]{\raisebox{1ex}[0pt][0pt]{\sffamily{\@author}}}
\makeatother

\begin{document}

% sezione (data)
\section{Lezione del 28-04-25}

% stili pagina
\thispagestyle{empty}
\pagestyle{fancy}

% testo
\subsection{Diagrammi di Nyquist}
Vediamo quindi l'ultimo strumento che sfrutteremo per poi progettare i regolatori, cioè i \textbf{diagrammi di Nyquist}.
Questi rappresentano la forma polare della risposta in frequenza associata alla funzione di trasferimento di un sistema lineare $G(s)$.

Tracciare un diagramma di Nyquist significherà quindi tracciare una curva sul piano complesso paramaterizzata dalla pulsazione (o frequenza).

Avremo che:
\begin{itemize}
	\item L'asse delle ascisse riporta i valori di $\mathrm{Re}\left( G(j \omega) \right)$;
	\item L'asse delle ordinate riporta i valori di $\mathrm{Im}\left( G(j \omega) \right)$;
	\item Tutti gli assi sono espressi in scala lineare.
\end{itemize}

Il diagramma di Nyquist non prevede l'addittività dei termini elementari.
Per pulsazioni elevate non riesce a specificare nel dettaglio l'andamento della risposta armonica quando il modulo tende a diventare piccolo.

Altre proprietà dei diagrammi di Nyquist sono:
\begin{itemize}
	\item I diagrammi sono graduati in funzione della pulsazione $\omega$;
	\item I diagrammi polari sono utili per lo studio della stabilità dei sistemi in retroazione, attraverso il cosiddetto \textbf{criterio di stabilità di Nyquist};
	\item Se conosciamo la funzione di trasferimento $G(s)$, il diagramma polare si può tracciare per punti separando le parti reali e immaginarie di $G(j\omega)$ e determinandone i valori corrispondenti a certi valori di $\omega$, cioè dicendo:
		$$
		G(s) \xrightarrow{s = j\omega} G(j\omega), \quad \omega = 0, 1, 10, ...
		$$
\end{itemize}

\subsubsection{Tracciamento del diagramma di Nyquist}
Vediamo quindi nel dettaglio come tracciare il diagramma di Nyquist.
Avremo effettivamente la possibilità di sfruttare 2 metodi:
\begin{enumerate}
	\item 
		Per il primo metodo partiremo dalla funzione di trasferimento in forma generica:
		$$
		G(j \omega) = K \frac{ (j\omega - z_1) ... (j\omega - z_m) }{ (j\omega - p_1) ... (j\omega - p_n) }
		$$
		e la divideremo in parte reale e complessa:
		\[
			\begin{cases}
				\mathrm{Re}\left( G(j\omega) \right) = K \frac{M_{z1} ... M_{zm}}{M_{p1} ... M_{pn}} \\
				\mathrm{Im} \left( G(j\omega) \right) = \phi_{z1} + ... + \phi_{zm} - \phi_{p1} - ... - \phi_{pn}
			\end{cases}
		\]
		\[
			\implies
			G(j\omega) = K \frac{M_{z1} ... M_{zm}}{M_{p1} ... M_{pn}} e^{\theta_1 - \phi_1 - \phi_2 - \phi_3
			}
		\]
		A questo punto devremo solo valutare i vari punti:
		$$
		p_i = \left( \mathrm{Re}\left( G(j\omega_i) \right), \mathrm{Im}\left( G(j\omega_i) \right) \right), \quad \omega_i = 0, 1, 10, ...
		$$
		e collegarli per ottenere il diagramma di Nyquist.

		Chiamiamo questo metodo \textbf{metodo analitico};
	
	\item
		Per il secondo metodo cercheremo un'approssimazione di tipo \textit{qualitativo}, come avevamo fatto per i diagrammi di Bode.

		# tipi sistema li ha fatti ?

		Supponiamo che il sistema sia di tipo zero, cioè poniamo il sistema con funzione di trasferimento:
		$$
		G(s) = k_1 \frac{ b_0 + b_1 s + b_2 s^2 + ... + s^m }{ s^h ( a_h + a_{h - 1} s + a_{h - 2} s^2 + ... + s^{n - h} ) }
		$$
		e imponiamo $h = 0$.

		\begin{itemize}
			\item 
		A questo punto vorremo trovare il punto a $\omega = 0$, dato dal guadagno statico:
		$$
			G(0) = \lim_{\omega \rightarrow 0} G(j \omega) = k_1 \cdot \frac{b_0}{a_0}
		$$
		con fase:
		\[
			\begin{cases}
				\phi_{\omega \rightarrow 0} = 0, \quad G(0) > 0 \\	
				\phi_{\omega \rightarrow 0} = -\pi, \quad G(0) < 0 \\	
			\end{cases}
		\]
		per sistemi non di tipo $0$, invece, cioè con $h \neq 0$, avremo:
		$$
			G(0) = \lim_{\omega \rightarrow 0} |G(j\omega)| = \infty
		$$
		con fase:
		$$
		\angle G(j\omega) = \frac{1}{s^h} = -h \cdot \frac{\pi}{2}
		$$
		se $k_1 > 0$.
	\item 
		Potrebbe poi interessarci il punto a $\omega \rightarrow \infty$, dato da:
		\[
				\lim_{\omega \rightarrow \infty} G(j\omega) = 	
			\begin{cases}
				0, \quad n > m \\
				k_1, \quad n = m
			\end{cases}
		\]
		con fase:
		$$
		\angle G(j\omega) = \frac{1}{s^{n - m}}
		$$
\end{itemize}
\end{enumerate}

\end{document}
