
\documentclass[a4paper,11pt]{article}
\usepackage[a4paper, margin=8em]{geometry}

% usa i pacchetti per la scrittura in italiano
\usepackage[french,italian]{babel}
\usepackage[T1]{fontenc}
\usepackage[utf8]{inputenc}
\frenchspacing 

% usa i pacchetti per la formattazione matematica
\usepackage{amsmath, amssymb, amsthm, amsfonts}

% usa altri pacchetti
\usepackage{gensymb}
\usepackage{hyperref}
\usepackage{standalone}

% imposta il titolo
\title{Appunti Fondamenti di Automatica}
\author{Luca Seggiani}
\date{2025}

% disegni
\usepackage{pgfplots}
\pgfplotsset{width=10cm,compat=1.9}

% imposta lo stile
% usa helvetica
\usepackage[scaled]{helvet}
% usa palatino
\usepackage{palatino}
% usa un font monospazio guardabile
\usepackage{lmodern}

% tikz in sans
\tikzset{every picture/.style={/utils/exec={\sffamily}}}

\renewcommand{\rmdefault}{ppl}
\renewcommand{\sfdefault}{phv}
\renewcommand{\ttdefault}{lmtt}

% circuiti
\usepackage{circuitikz}
\usetikzlibrary{babel}

% disponi il titolo
\makeatletter
\renewcommand{\maketitle} {
	\begin{center} 
		\begin{minipage}[t]{.8\textwidth}
			\textsf{\huge\bfseries \@title} 
		\end{minipage}%
		\begin{minipage}[t]{.2\textwidth}
			\raggedleft \vspace{-1.65em}
			\textsf{\small \@author} \vfill
			\textsf{\small \@date}
		\end{minipage}
		\par
	\end{center}

	\thispagestyle{empty}
	\pagestyle{fancy}
}
\makeatother

% disponi teoremi
\usepackage{tcolorbox}
\newtcolorbox[auto counter, number within=section]{theorem}[2][]{%
	colback=blue!10, 
	colframe=blue!40!black, 
	sharp corners=northwest,
	fonttitle=\sffamily\bfseries, 
	title=Teorema~\thetcbcounter: #2, 
	#1
}

% disponi definizioni
\newtcolorbox[auto counter, number within=section]{definition}[2][]{%
	colback=red!10,
	colframe=red!40!black,
	sharp corners=northwest,
	fonttitle=\sffamily\bfseries,
	title=Definizione~\thetcbcounter: #2,
	#1
}

% disponi problemi
\newtcolorbox[auto counter, number within=section]{problem}[2][]{%
	colback=green!10,
	colframe=green!40!black,
	sharp corners=northwest,
	fonttitle=\sffamily\bfseries,
	title=Problema~\thetcbcounter: #2,
	#1
}

% disponi codice
\usepackage{listings}
\usepackage[table]{xcolor}

\lstdefinestyle{codestyle}{
		backgroundcolor=\color{black!5}, 
		commentstyle=\color{codegreen},
		keywordstyle=\bfseries\color{magenta},
		numberstyle=\sffamily\tiny\color{black!60},
		stringstyle=\color{green!50!black},
		basicstyle=\ttfamily\footnotesize,
		breakatwhitespace=false,         
		breaklines=true,                 
		captionpos=b,                    
		keepspaces=true,                 
		numbers=left,                    
		numbersep=5pt,                  
		showspaces=false,                
		showstringspaces=false,
		showtabs=false,                  
		tabsize=2
}

\lstdefinestyle{shellstyle}{
		backgroundcolor=\color{black!5}, 
		basicstyle=\ttfamily\footnotesize\color{black}, 
		commentstyle=\color{black}, 
		keywordstyle=\color{black},
		numberstyle=\color{black!5},
		stringstyle=\color{black}, 
		showspaces=false,
		showstringspaces=false, 
		showtabs=false, 
		tabsize=2, 
		numbers=none, 
		breaklines=true
}

\lstdefinelanguage{javascript}{
	keywords={typeof, new, true, false, catch, function, return, null, catch, switch, var, if, in, while, do, else, case, break},
	keywordstyle=\color{blue}\bfseries,
	ndkeywords={class, export, boolean, throw, implements, import, this},
	ndkeywordstyle=\color{darkgray}\bfseries,
	identifierstyle=\color{black},
	sensitive=false,
	comment=[l]{//},
	morecomment=[s]{/*}{*/},
	commentstyle=\color{purple}\ttfamily,
	stringstyle=\color{red}\ttfamily,
	morestring=[b]',
	morestring=[b]"
}

% disponi sezioni
\usepackage{titlesec}

\titleformat{\section}
	{\sffamily\Large\bfseries} 
	{\thesection}{1em}{} 
\titleformat{\subsection}
	{\sffamily\large\bfseries}   
	{\thesubsection}{1em}{} 
\titleformat{\subsubsection}
	{\sffamily\normalsize\bfseries} 
	{\thesubsubsection}{1em}{}

% disponi alberi
\usepackage{forest}

\forestset{
	rectstyle/.style={
		for tree={rectangle,draw,font=\large\sffamily}
	},
	roundstyle/.style={
		for tree={circle,draw,font=\large}
	}
}

% disponi algoritmi
\usepackage{algorithm}
\usepackage{algorithmic}
\makeatletter
\renewcommand{\ALG@name}{Algoritmo}
\makeatother

% disponi numeri di pagina
\usepackage{fancyhdr}
\fancyhf{} 
\fancyfoot[L]{\sffamily{\thepage}}

\makeatletter
\fancyhead[L]{\raisebox{1ex}[0pt][0pt]{\sffamily{\@title \ \@date}}} 
\fancyhead[R]{\raisebox{1ex}[0pt][0pt]{\sffamily{\@author}}}
\makeatother

\begin{document}

% sezione (data)
\section{Lezione del 20-03-25}
		
% stili pagina
\thispagestyle{empty}
\pagestyle{fancy}

% testo
\subsection{Proprietà della trasformata di Laplace}
Vediamo alcune proprietà della trasformata di Laplace introdotta alla scorsa lezione.

\begin{enumerate}
	\item \textbf{Linearità:} si ha che, prese due trasformate:
		\[
			\begin{cases}
				\mathcal{L}\{f_1(t)\} = F_1(s), \quad \mathrm{Re}(s) > a_1 \\ 	
				\mathcal{L}\{f_2(t)\} = F_2(s), \quad \mathrm{Re}(s) > a_2 \\ 	
			\end{cases}
		\]
		la loro combinazione lineare è semplicemente la trasformata:
		$$
		\mathcal{L}\{c_1 \cdot f_1(t) + c_2 \cdot f_2(t)\} = c_1 \cdot F_1(s) + c_2 \cdot F_2(s), \quad \mathrm{Re}(s) \geq \max(a_1, a_2)
		$$
		La dimostrazione è banale e deriva dalla linearità dell'operatore integrale sulla base del quale avevamo definito la trasformata stessa.
	\item \textbf{Traslazione:} abbiamo visto che il termine $e^{-s t_0}$ rappresenta una traslazione temporale, ovvero:
		$$
		\mathcal{L}\{ f(t - t_0) \} = e^{-s t_0} F(s)
		$$
		Questo si dimostra da:
		$$
		\mathcal{L}\{f(t - t_0)\} = \int_0^{+\infty} f(t - t_0) \cdot e^{-s t} \, dt = \int_{-t_0}^{+\infty} f(\tau) \cdot e^{-s(\tau + t_0)} \, d\tau 
		$$
		$$
		= e^{-s t_0} \int_0^{+ \infty} f(\tau) \cdot e^{-s \tau} \, d\tau = e^{-s t_0} \cdot F(s)
		$$ # ci dovrebbe essere heaviside di mezzo, così è malposta
		operando il cambio di variabile $\tau = t - t_0$. \qed
	\item \textbf{Cambio di scala:} prendiamo il "cambio di scala" alla variabile temporale $f(at)$. Avremo che il risultatò dipenderà dalla L-trasformata di $f$, che avevamo chiamato $F$, come:
		$$
			\mathcal{L}\{f(at)\} = \frac{1}{a} \cdot F\left(\frac{s}{a}\right) 
		$$
		Questo si dimostra applicando la definizione col cambio di variabile: $\tau = at$, con $d\tau = a \, dt$:
		$$
		\mathcal{L}\{f(at) \} = \int_0^{+ \infty} f(at) e^{-st} \, dt = \int_0^{+ \infty} f(\tau) e^{-s \frac{\tau}{a}} \frac{d\tau}{a}
		$$
		$$
		= \frac{1}{a} \int_0^{+ \infty} f(\tau) e^{-s \frac{\tau}{a}} \, d\tau = \frac{1}{a} \cdot F\left( \frac{s}{a} \right)
		$$
	\item \textbf{Derivazione in s:} vale la proprietà rispetto alle derivate su $s$:
		$$
			\mathcal{L}(t \cdot f(t)) = - \frac{d F(s)}{ds}
		$$
		Per la derivazione a catena si ha quindi:
		$$
			\mathcal{L}\{t^n \cdot f(t)\} = (-1)^n \frac{d^n F(s)}{ds^n}
		$$
		e quindi riguardo al gradino:
		$$
		\mathcal{L}\left\{ \frac{t^k}{k!} \cdot H(t) \right\} = \frac{1}{s^{k + 1}}
		$$
	\item \textbf{Traslazione in s:} vale rispetto alla traslazione su s:
		$$
		F(s - a) = \mathcal{L}\{ e^{at}\cdot f(t) \}
		$$
\end{enumerate}

# dimostra tutte quest ultime

\subsubsection{Sinusoide}
Usiamo la proprietà di linearità appena trovata per calcolare l'L-trasformata di $f(t) = H(t)\sin(\omega t)$.
Avremo che, dalla formula di Eulero:
$$
\mathcal{L}\{\sin(\omega t)\} = \mathcal{L}\left\{ \frac{e^{j \omega t} - e^{- j \omega t}}{2j} \right\}
$$
Potremo allora applicare la linearità:
$$
= \frac{1}{2j} \mathcal{L}\{e^{j \omega t}\} - \frac{1}{2j} \mathcal{L}\{e^{-j \omega t}\}
$$
da cui si continua con semplici calcoli (attraverso la trasformata dell'esponenziale vista alla scorsa lezione):
$$
= \frac{1}{2j}\left( \frac{1}{s - j\omega} - \frac{1}{s + j \omega} \right) = \frac{1}{2j}\left( \frac{2j\omega}{s^2 + \omega^2} \right) = \frac{\omega}{s^2 + \omega^2}
$$ \qed

\subsubsection{Cosinusoide}
Valgono considerazioni analoghe per la funzione $f(t) = H(t) \cos(\omega t)$.
Avremo che, dalla formula di Eulero:
$$
\mathcal{L}\{\sin(\omega t)\} = \mathcal{L}\left\{ \frac{e^{j \omega t} - e^{- j \omega t}}{2} \right\}
$$
Potremo allora applicare la linearità:
$$
= \frac{1}{2} \mathcal{L}\{e^{j \omega t}\} - \frac{1}{2} \mathcal{L}\{e^{-j \omega t}\}
$$
da cui:
$$
= \frac{1}{2}\left( \frac{1}{s - j\omega} - \frac{1}{s + j \omega} \right) = \frac{1}{2}\left( \frac{2s}{s^2 + \omega^2} \right) = \frac{s}{s^2 + \omega^2}
$$ \qed

\subsubsection{Sinusoidi/cosinusoidi con scostamento}
Nel caso di scostamento dato da un termine $\phi$, cioè $f(t) = \sin(\omega t + \phi)$, avremo dall'applicazione delle formule di addizione degli angoli:
$$
\mathcal{L}(\sin(\omega t + \phi)) = \mathcal{L}( \sin(\omega t) \cdot \cos(\phi) + \cos(\omega t) \cdot \sin(\phi) ) = \cos(\phi) \cdot \mathcal{L} \{ \sin(\omega t) \} + \sin(\phi) \cdot \mathcal{L}(\cos (\omega t))
$$
e dalle formule appena trovate::
$$
= \cos(\phi) \cdot \frac{w}{s^2 + \omega ^2} + \sin(\phi) \cdot \frac{s}{s^2 + \omega ^2} = 
$$

\subsection{Operazioni integro-differenziali con Laplace}
La comodità della trasformata di Laplace è che ci permette di esprimere per via algebrica gli operatori integro-differenziali.

\subsubsection{Derivata con Laplace}
Vale che:
\begin{theorem}{Derivata con Laplace}
	Per una certa funzione $f$, sotto larghe ipotesi, vale:
	$$
	\mathcal{L}\left\{ \frac{d}{dt} f(t) \right\} = s F(s) - f(0)
	$$
\end{theorem}

Questo si dimostra ricordando la formula di integrazione per parti:
$$
\int u \, dv = u \cdot v - \int v \, du
$$
e applicando:
$$
\mathcal{L}\left\{ \frac{df}{dt} \right\} = \int_0^{+\infty} \frac{df}{dt} \cdot e^{-st} \, dt = f(t) \cdot e^{-(st)} \Big|_0^{+\infty} - \int_0^{+ \infty} -s \cdot f(t) \cdot e^{-st} \, dt = s F(s) - f(0)
$$ \qed

Una nota importante è il valore della funzione in $0$.
Nel caso di funzioni discontinue (come ad esempio il gradino) si deve infatti scegliere fra $0^-$ e $0^+$, cioè:
$$
\mathcal{L}^{-} \{f(t)\} = \lim_{\epsilon \rightarrow 0^-} \int_\epsilon^{+ \infty} f(t) e^{-st} \, dt
$$
$$
\mathcal{L}^{+} \{f(t)\} = \lim_{\epsilon \rightarrow 0^+} \int_\epsilon^{+ \infty} f(t) e^{-st} \, dt
$$

Varrà:
$$
\mathcal{L}^{-} = \mathcal{L}^{+} + \int_{0^-}^{0^+} f(t) e^{-st} \, dt = \mathcal{L}^{+} + a_0
$$
dove la costante $a_0$ rappresenterà gli effetti impulsivi che ci portano al "salto" in $0$.

\subsubsection{Esempio: derivata del gradino}
Possiamo usare la formula appena trovata per la derivazione per dimostrare che in qualche modo l'impulso di Dirac rappresenta la derivata del gradino di Heaviside.
Notiamo infatti che:
$$
\mathcal{L}\left\{ \frac{d}{dt} H(t) \right\} = s \cdot \frac{1}{s} - f(0^-) = 1 = \mathcal{L}(\sigma(t))
$$
assunto $t_0 = 0$.

\subsubsection{Integrale con Laplace}
Vale che:
\begin{theorem}{Integrale con Laplace}
Per una certa funzione $f$, sotto larghe ipotesi, vale:
$$
\mathcal{L}\left\{ \int_0^t f(t) \, dt \right\} = \frac{F(s)}{s}
$$
\end{theorem}

Questo si dimostra partendo dalla formula di derivazione, e ponendo:
$$
g(t) = \int_0^t f(\tau) \, d\tau, \quad g(0) = 0
$$
da cui:
$$
\frac{d g(t)}{dt} = f(t)
$$
e quindi:
$$
\mathcal{L}\{f(t)\} = F(s) = \mathcal{L} \left\{ \frac{dg}{dt} \right\} = s \cdot \mathcal{L}\{g(t)\} - g(0) \implies \mathcal{L}\{g(t)\} = \frac{F(s)}{s}
$$
da cui la tesi. \qed

\subsubsection{Convoluzione con Laplace}
Vediamo infine la convoluzione:
\begin{theorem}{Convoluzione con Laplace}
	Per due funzioni $f_1, f_2$, sotto larghe ipotesi e definito un opportuno prodotto di convoluzione, vale:
	$$
		\mathcal{L}\{ f_1(t) \, * \, f_2(t) \} = F_1(s) \cdot F_2(s)
	$$
	assunto:
	\[
		\begin{cases}
			\mathcal{L} \{ f_1(t) \} = F_1(s) \\ 	
			\mathcal{L} \{ f_2(t) \} = F_2(s) 	
		\end{cases}
	\]
\end{theorem}

Per noi il prodotto di convoluzione sarà il solito:
$$
f_1(t) \, * \, f_2(t) = \int_{-\infty}^{+ \infty} f_1(\tau) \cdot f_2(t - \tau) \, d\tau
$$
che sotto le ipotesi di funzioni "tagliate" a 0, come abbiamo assunto, equivale a:
$$
= \int_0^t f_1(\tau) \cdot f_2(t - \tau) \, d\tau = \int_0^{+\infty} f_1(\tau) \cdot f_2(t - \tau) \, d\tau
$$
Potremo allora dire:
$$
\mathcal{L} \{ f_1(t) \, * \, f_2(t) \} = \int_0^{+\infty} \left( \int_0^{+ \infty} f_1(\tau) \cdot f_2(t - \tau) \, d\tau \right) e^{-st} \, dt
$$
Posto $e^{-st} = e^{-s(t + \tau - \tau)}$ si ha:
$$
= \int_0^{+\infty} f_1(\tau) \cdot e^{-st} \left( \int_0^{+\infty} f_2(t - \tau) \cdot e^{-s(t - \tau)} \, dt \right) \, d\tau
$$
a questo punto posto $v = t - \tau$ e quindi $dv = dt$ si ha:
$$
= \int_0^{+\infty} f_1(\tau) \cdot e^{- s \tau} \left( \int_{-\tau}^{+\infty} f_2(v) \cdot e^{-sv} \, dv \right) \, d\tau
$$
che notando la non interdipendenza dei due integrali (il $-\tau$ a pedice non "prende" nulla con $f_2$ definita per $t > 0$) si divide in:
$$
= \int_0^{+\infty} f_1(\tau) \cdot e^{- s \tau} \, d\tau \cdot \int_0^{+\infty} f_2(v) \cdot e^{-sv} \, dv = F_1(s) + F_2(s)
$$
da cui la tesi. \qed



\end{document}
