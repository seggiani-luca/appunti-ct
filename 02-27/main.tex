
\documentclass[a4paper,11pt]{article}
\usepackage[a4paper, margin=8em]{geometry}

% usa i pacchetti per la scrittura in italiano
\usepackage[french,italian]{babel}
\usepackage[T1]{fontenc}
\usepackage[utf8]{inputenc}
\frenchspacing 

% usa i pacchetti per la formattazione matematica
\usepackage{amsmath, amssymb, amsthm, amsfonts}

% usa altri pacchetti
\usepackage{gensymb}
\usepackage{hyperref}
\usepackage{standalone}

% imposta il titolo
\title{Appunti Fondamenti di Automatica}
\author{Luca Seggiani}
\date{2025}

% disegni
\usepackage{pgfplots}
\pgfplotsset{width=10cm,compat=1.9}

% imposta lo stile
% usa helvetica
\usepackage[scaled]{helvet}
% usa palatino
\usepackage{palatino}
% usa un font monospazio guardabile
\usepackage{lmodern}

% tikz in sans
\tikzset{every picture/.style={/utils/exec={\sffamily}}}

\renewcommand{\rmdefault}{ppl}
\renewcommand{\sfdefault}{phv}
\renewcommand{\ttdefault}{lmtt}

% circuiti
\usepackage{circuitikz}
\usetikzlibrary{babel}

% disponi il titolo
\makeatletter
\renewcommand{\maketitle} {
	\begin{center} 
		\begin{minipage}[t]{.8\textwidth}
			\textsf{\huge\bfseries \@title} 
		\end{minipage}%
		\begin{minipage}[t]{.2\textwidth}
			\raggedleft \vspace{-1.65em}
			\textsf{\small \@author} \vfill
			\textsf{\small \@date}
		\end{minipage}
		\par
	\end{center}

	\thispagestyle{empty}
	\pagestyle{fancy}
}
\makeatother

% disponi teoremi
\usepackage{tcolorbox}
\newtcolorbox[auto counter, number within=section]{theorem}[2][]{%
	colback=blue!10, 
	colframe=blue!40!black, 
	sharp corners=northwest,
	fonttitle=\sffamily\bfseries, 
	title=Teorema~\thetcbcounter: #2, 
	#1
}

% disponi definizioni
\newtcolorbox[auto counter, number within=section]{definition}[2][]{%
	colback=red!10,
	colframe=red!40!black,
	sharp corners=northwest,
	fonttitle=\sffamily\bfseries,
	title=Definizione~\thetcbcounter: #2,
	#1
}

% disponi problemi
\newtcolorbox[auto counter, number within=section]{problem}[2][]{%
	colback=green!10,
	colframe=green!40!black,
	sharp corners=northwest,
	fonttitle=\sffamily\bfseries,
	title=Problema~\thetcbcounter: #2,
	#1
}

% disponi codice
\usepackage{listings}
\usepackage[table]{xcolor}

\lstdefinestyle{codestyle}{
		backgroundcolor=\color{black!5}, 
		commentstyle=\color{codegreen},
		keywordstyle=\bfseries\color{magenta},
		numberstyle=\sffamily\tiny\color{black!60},
		stringstyle=\color{green!50!black},
		basicstyle=\ttfamily\footnotesize,
		breakatwhitespace=false,         
		breaklines=true,                 
		captionpos=b,                    
		keepspaces=true,                 
		numbers=left,                    
		numbersep=5pt,                  
		showspaces=false,                
		showstringspaces=false,
		showtabs=false,                  
		tabsize=2
}

\lstdefinestyle{shellstyle}{
		backgroundcolor=\color{black!5}, 
		basicstyle=\ttfamily\footnotesize\color{black}, 
		commentstyle=\color{black}, 
		keywordstyle=\color{black},
		numberstyle=\color{black!5},
		stringstyle=\color{black}, 
		showspaces=false,
		showstringspaces=false, 
		showtabs=false, 
		tabsize=2, 
		numbers=none, 
		breaklines=true
}

\lstdefinelanguage{javascript}{
	keywords={typeof, new, true, false, catch, function, return, null, catch, switch, var, if, in, while, do, else, case, break},
	keywordstyle=\color{blue}\bfseries,
	ndkeywords={class, export, boolean, throw, implements, import, this},
	ndkeywordstyle=\color{darkgray}\bfseries,
	identifierstyle=\color{black},
	sensitive=false,
	comment=[l]{//},
	morecomment=[s]{/*}{*/},
	commentstyle=\color{purple}\ttfamily,
	stringstyle=\color{red}\ttfamily,
	morestring=[b]',
	morestring=[b]"
}

% disponi sezioni
\usepackage{titlesec}

\titleformat{\section}
	{\sffamily\Large\bfseries} 
	{\thesection}{1em}{} 
\titleformat{\subsection}
	{\sffamily\large\bfseries}   
	{\thesubsection}{1em}{} 
\titleformat{\subsubsection}
	{\sffamily\normalsize\bfseries} 
	{\thesubsubsection}{1em}{}

% disponi alberi
\usepackage{forest}

\forestset{
	rectstyle/.style={
		for tree={rectangle,draw,font=\large\sffamily}
	},
	roundstyle/.style={
		for tree={circle,draw,font=\large}
	}
}

% disponi algoritmi
\usepackage{algorithm}
\usepackage{algorithmic}
\makeatletter
\renewcommand{\ALG@name}{Algoritmo}
\makeatother

% disponi numeri di pagina
\usepackage{fancyhdr}
\fancyhf{} 
\fancyfoot[L]{\sffamily{\thepage}}

\makeatletter
\fancyhead[L]{\raisebox{1ex}[0pt][0pt]{\sffamily{\@title \ \@date}}} 
\fancyhead[R]{\raisebox{1ex}[0pt][0pt]{\sffamily{\@author}}}
\makeatother

\begin{document}

% sezione (data)
\section{Lezione del 27-02-25}

% stili pagina
\thispagestyle{empty}
\pagestyle{fancy}

% testo
Riprendiamo la trattazione dei sistemi non lineari sfruttando un esempio:

\subsubsection{Esempio: regolazione della velocità di crociera}
Poniamo di voler modellizzare il sistema di trazione longitudinale di un automobile, implementando un sistema di controllo che la mantenga a velocità di crociera costante.
L'\textbf{ingresso manipolabile} sarà il \textit{pedale dell'acceleratore}, l'\textbf{uscita} del sistema la \textit{velocità}, e il \textbf{disturbo} dato dalla \textit{pendenza} del tratto di strada che stiamo percorrendo.

Il diagramma a blocchi del sistema potrà essere realizzato come segue:

\begin{center}
	\begin{tikzpicture}
		\draw (0,0) rectangle (2, 1);
		\draw (6,0) rectangle (8, 1);
		
		\draw[-stealth] (-2, 0.5) -> (0, 0.5);
		\draw[-stealth] (2, 0.5) -> (3.9, 0.5);
		\draw[-stealth] (4, 0.5) -> (6, 0.5);
		\draw[-stealth] (8, 0.5) -> (10, 0.5);
		
		\draw[-stealth] (4, 2) -> (4, 0.6);

		\draw[fill=white] (4, 0.5) circle (0.1);
		
		\node at (1, 0.5) {Motore};
		\node at (7, 0.5) {Veicolo};
		\node at (-1.3, 0.8) {Acceleratore};
		\node at (9, 0.8) {Velocità};
		\node at (4, 2.3) {Forze esterne};
	\end{tikzpicture}
\end{center}

Un modello matematico potrebbe essere il seguente:
$$
m \frac{dv}{dt} + \alpha |v|v + \beta v = \gamma u - m g (\sin{\theta})
$$
dove $\alpha |v|v + \beta v$ rappresentano due componenti, una \textbf{quadratica} e una \textbf{lineare}, dell'attrito.
Possiamo interpretare la componente quadratica come l'attrito con l'aria, mentre la componente lineare come l'attrito del pneumatico con la superificie stradale.
$\gamma u$ sarà invece la spinta del motore, chiamata $u$ l'apertura della valvola dell'acceleratore in un dato istante temporale e quindi $\gamma$ una qualche costante di proporzionalità legata alle caratteristiche del motore e della trasmissione della vettura.
Infine, $g\sin(\theta)$ rappresenterà l'accelerazione gravitazionale in direzione longitudinale al veicolo (posta una pendenza di $\theta$ con $\theta < 30^\circ$).

Notiamo che questa equazione è non lineare.
Potremmo pensare di semplificarla: ad esempio rimuovendo il termine quadratico (trascurabile alle basse velocità) e il riportando il termine sinusoidale al primo componente dello sviluppo di MacLaurin (valido per piccoli angoli), ottenendo quindi:

$$
m \frac{dv}{dt} + \alpha |v|v + \beta v = \gamma u - m g (\sin{\theta}) \sim m \frac{dv}{dt} + \beta v = \gamma u - mg\theta
$$

Nota questa possibilità, decidiamo comunque di continuare con la trattazione del sistema completo.
Prese le variabili di stato comuni ai sistemi meccanici $\mathbf{x} = \begin{pmatrix}x & v \end{pmatrix}$, che rinominiamo in $\mathbf{x} = \begin{pmatrix} x_1 & x_2 \end{pmatrix}$, si ha il sistema:
\[
	\begin{cases}
		x_1' = x_2 \\
		x_2' = -\frac{\alpha}{m} |x_2| x_2 - \frac{\beta}{m} x_2 + \frac{\gamma}{m} u - g \sin{\theta} \\ 
		y = x_2
	\end{cases}
\]
con la condizione iniziale $\mathbf{x}(t_0) = x_0$.

\par\smallskip

Nelle equazioni dell'esempio appena visto non compare $t$, quindi il sistema è tempo invariante. 
Di contro, il sistema non ricade nelle forme viste prima, ma in una forma più generale:
\begin{definition}{Forma non lineare generale}
Introduciamo la forma generica per sistemi non lineari:
\[
	\begin{cases}
		x' = f(x, u) \\
		y = g(x, u)
	\end{cases}
\]
\end{definition}

Un approccio alla gestione di sistemi di questo tipo è la \textbf{linearizzazione}.
Per capire i punti ottimali su cui effettuare una linearizzazione, è opportuno introdurre il concetto di \textbf{equilibrio} e \textbf{stabilità}.

\subsection{Equilibri e stabilità}
Diamo una definizione di \textbf{stato di equilibrio}:
\begin{definition}{Stato di equilibrio}
	Dato il sistema $x' = f(x, u)$, e dato un segnale di ingresso $u(t) = \overline{u}$, uno stato $\overline{x}$ si dice di equilibrio se soddisfa la relazone:
	$$
	0 = f(\overline{x}, \overline{u})
	$$
\end{definition}

In questo caso, il sistema non si muove dallo stato $\overline{x}$, come risulta immediato da $\overline{x}' = f(\overline{x}, \overline{u}) = 0$.

\subsubsection{Linearizzazione attorno ad uno stato di equilibrio}
Partiamo dalla forma generale vista adesso, e prendiamo il sistema come stazionario.
Dato un ingresso $\overline{u}$ e il suo stato di equilibrio corrispondente $\overline{x}$, definiamo due nuove variabili che rappresentano la \textbf{variazione di equilibrio}:
\[
	\begin{cases}
		\Delta x = x - \overline{x} \\ 	
		\Delta u = u - \overline{u} \\ 	
	\end{cases}
\]

A questo punto possiamo scrivere le equazioni di stato attraverso la serie di Taylor centrata nello stato di equilibrio troncata al primo ordine:
$$
x' = f(x, u)
= f(\overline{x}, \overline{u}) + \frac{\partial f}{\partial x} \Bigg|_{x = \overline{x}, \ u = \overline{u}} \Delta x
+ \frac{\partial f}{\partial u} \Bigg|_{x = \overline{x}, \ u = \overline{u}} \Delta u
+ O(|\Delta x|^2)
+ O(|\Delta u|^2)
$$

Abbiamo quindi linearizzato un sistema non lineare attorno ad un punto di equilibrio.
Le matrici che moltiplicano i termini del primo ordine sono chiaramente le \textbf{Jacobiane}:
$$
A = \frac{\partial f}{\partial x} \Bigg|_{\overline{x}, \overline{u}} = \begin{pmatrix}
	\frac{\partial f_1}{\partial x_1} & \frac{\partial f_1}{\partial x_2} & ... \\ 
	\frac{\partial f_2}{\partial x_1} & \frac{\partial f_2}{\partial x_2} & ... \\ 
	... & ... & ...
\end{pmatrix}
$$
$$
B = \frac{\partial f}{\partial x} \Bigg|_{\overline{x}, \overline{u}} = \begin{pmatrix}
	\frac{\partial f_1}{\partial u_1} & \frac{\partial f_1}{\partial u_2} & ... \\ 
	\frac{\partial f_2}{\partial u_1} & \frac{\partial f_2}{\partial u_2} & ... \\ 
	... & ... & ...
\end{pmatrix}
$$

da cui riscriviamo la formula di Taylor riportata prima come:
$$
x' = A \cdot \Delta x + B \cdot \Delta u
$$ 

Abbiamo quindi trovato la forma a variabili di stato.
Associando allo stato di equilibrio anche un'uscita di equililbrio $\overline{y} = g(\overline{x}, \overline{u})$, e chiamando $\Delta y$ la variazione di uscita, si può ricavare il sistema completo a variabili di stato:

\[
	\begin{cases}
		x' = Ax + Bu \\ 
		y = Cx + Du
	\end{cases}
\]

\subsubsection{Esempio: linearizzazione del problema di regolazione della velocità di crociera}
Riprendendo l'esempio precedente (3.0.1) della regolazione della velocità di crociera, avevamo ricavato il sistema non lineare in forma generale:

$$
x' = f(x, u) = 
	\begin{cases}
		x_1' = x_2 \\
		x_2' = -\frac{\alpha}{m} |x_2| x_2 - \frac{\beta}{m} x_2 + \frac{\gamma}{m} u - g \sin{\theta} \\ 
		y = x_2
	\end{cases}
$$
da cui le matrici:
$$
A = \frac{\partial f}{\partial x} = \begin{pmatrix}
	\frac{\partial f_1}{\partial x_1} & \frac{\partial f_1}{\partial x_2} \\ 
	\frac{\partial f_2}{\partial x_1} & \frac{\partial f_2}{\partial x_2} \\ 
\end{pmatrix} = \begin{pmatrix}
	0 & 1 \\ 
	0 & -\frac{\alpha}{m}|\overline{x_2}| - \frac{\beta}{m}
\end{pmatrix}
$$
$$
B = \frac{\partial f}{\partial u} = \begin{pmatrix}
	\frac{\partial f_1}{\partial u_1} & \frac{\partial f_1}{\partial u_2} \\ 
	\frac{\partial f_2}{\partial u_1} & \frac{\partial f_2}{\partial u_2} \\ 
\end{pmatrix} = \begin{pmatrix}
	0 & 0 \\ 
	\frac{\gamma}{m} & -g\cos{\theta} 
\end{pmatrix} \\
$$

Ciò di cui abbiamo bisogno sarà quindi una condizione di equilibrio.
Se imponiamo $\theta = 0$ e $u = 0$, cioè piano orizzontale con nessuna forza di controllo, otteniamo $x_2 = 0$ per qualsiasi $x_1$, quindi uno stato di equilibrio.

Sostituendo, si ricava:

$$
A = \frac{\partial f}{\partial x} \Bigg|_{x_1 = 0, \ x_2 = 0} = \begin{pmatrix}
	0 & 1 \\ 
	0 & -\frac{\beta}{m}
\end{pmatrix}
$$
$$
A = \frac{\partial f}{\partial u} \Bigg|_{u = 0, \ \theta = 0} = \begin{pmatrix}
	0 & 1 \\ 
	\frac{\gamma}{m} & -g
\end{pmatrix}
$$

Da cui il sistema:
\[
	\begin{cases}			
x' = \begin{pmatrix}
	0 & 1 \\ 
	0 & -\frac{\beta}{m}
\end{pmatrix} x + \begin{pmatrix}
	0 & 0 \\ 
	\frac{\gamma}{m} & -g
\end{pmatrix} u \\ 
y = \begin{pmatrix}
	0 & 1
\end{pmatrix} x
	\end{cases}
\]

\par\smallskip

Abbiamo quindi che, sebbene il mondo sia principalmente non lineare, e i sistemi non lineari siano di difficile trattazione, possiamo sempre trovare una forma linearizzata attorno a dei punti di equilibrio e quindi ricondurci a sistemi lineari.
Da ora in avanti considereremo quindi solo sistemi di questo tipo.

\subsection{Stabilità di Lyapunov}
La \textbf{stabilità} di un sistema considera le conseguenze sul movimento del sistema di un incertezza sul valore iniziale dello stato.
L'idea è che, in un sistema \textit{stabile}, \textit{piccole} perturbazioni dello stato iniziale rispetto ad un valore di riferimento provocano solo \textit{piccole} perturbazioni sul movimento dello stato, che ci aspettiamo si annullino sul lungo termine.

\begin{definition}{Stato di equilibrio stabile (Lyapunov)}
	Poniamo un ingresso costante $u(t) = \overline{u}$ con corrispondente stato di equilibrio $\overline{x}$, e consideriamo un movimento \textit{perturbato} ottenuto da $u(t) = \overline{u}$ ma a partire da $x_0 \neq \overline{x}$.
	$\overline{x}$ si dice stabile se, per ogni $\epsilon > 0$, esiste $\sigma > 0$ tale che per tutti gli $x_0$ che soddisfano:
	$$
		|x_0 - \overline{x}| \leq \delta 
	$$
	si ha:
	$$
		|x(t) - \overline{x} | \leq \epsilon, \quad t \geq 0 
	$$
\end{definition}
notiamo come la definizione ricalca quella di limite, si può infatti definire:
\begin{definition}{Stabilità asintotica}
	Uno stato di equilibrio stabile $\overline{x}$ è anche asintoticamente stabile se vale:
	$$
		\lim_{t\rightarrow+\infty} |x(t) - \overline{x}| = 0 
	$$
\end{definition}

Questa definizione è effettivamente una versione più forte dello stato di equilibrio stabile che prevede piena convergenza in prospettiva di $t \rightarrow \infty$, anziché confinamento in un cono (cilindro) diretto in direzione tempo positiva con base $B(\overline{x}, \delta)$.

\subsection{Formula di Lagrange}
Avevamo definito il sistema in forma variabili di stato:
\[
	\begin{cases}
		x' = Ax + Bu \\ 
		y = Cx + Du
	\end{cases}
\]

Per sistemi LTI possiamo calcolare in forma chiusa la funzione di transizione dello stato:
\[
	\begin{cases}			
		x(t) = \phi(t, t_0, x(t_0), u(t)) \\ 
		y(t) = \gamma(t, t_0, x(t_0), u(t)  
	\end{cases}
\]

Riassumiamo alcuni concetti che ci saranno utili.
Notiamo che un equazione differenziale scalare e omogenea:
\[
	\begin{cases}
		x'(t) = a \cdot x(t) \\ 
		x(t_0 = 0) = x_0
	\end{cases}
\]
ha una soluzione del tipo $x(t) = x_0 e^{at}$, con la possibilità di $a \in \mathbb{C}$ e quindi:
$$
e^{at} = e^{x + iy} = e^x \left( \cos(y) + i \sin(y) \right)
$$

Notiamo poi che sviluppando secondo la serie di Taylor la funzione esponenziale si ha:
$$
e^{at} = \sum_{i = 0}^\infty a^i \cdot \frac{t^i}{i!} = 1 + at + a^2 \frac{t^2}{2!} + ... + a^r \frac{t^r}{r!}
$$

Nei nostri sistemi avevamo una matrice $A$, cioè un'equazione differenziale vettoriale:
\[
	\begin{cases}			
		x'(t) = Ax(t) \\ 
		x(t_0 = 0) = x_0
	\end{cases}
\]

quindi la soluzione dovrà necessariamente essere:
$$
x(t) = e^{At} x_0
$$

Stiamo effettivamente elevando un esponenziale ad una matrice (gli analisti avevano ragione!).
Potremo quindi usare lo sviluppo di Taylor per trovare $e^{At}$:
$$
e^{At} = \sum_{i = 0}^\infty A^i \cdot \frac{t^i}{i!} = I + At + A^2 \frac{t^2}{2!} + ... + A^r \frac{t^r}{r!}
$$

Chiamiamo la $A$ in $e^{At}$ \textbf{matrice di transizione}, perché determina il \textbf{moto libero del sistema} in assenza di ingresso ($u(t) = 0$):

$$
x(t) = \phi(t, t_o, x(t_0), u = 0) = e^A(t- t_0)x(t_0)
$$

Notiamo poi che nel caso di condizioni iniziali a $t_0 \neq 0$ dovremo traslare la soluzione come:
$$
x(t) = Ae^{(t - t_0)} x_0
$$

\end{document}
