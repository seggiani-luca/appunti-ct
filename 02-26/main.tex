
\documentclass[a4paper,11pt]{article}
\usepackage[a4paper, margin=8em]{geometry}

% usa i pacchetti per la scrittura in italiano
\usepackage[french,italian]{babel}
\usepackage[T1]{fontenc}
\usepackage[utf8]{inputenc}
\frenchspacing 

% usa i pacchetti per la formattazione matematica
\usepackage{amsmath, amssymb, amsthm, amsfonts}

% usa altri pacchetti
\usepackage{gensymb}
\usepackage{hyperref}
\usepackage{standalone}

% imposta il titolo
\title{Appunti Fondamenti di Automatica}
\author{Luca Seggiani}
\date{2025}

% disegni
\usepackage{pgfplots}
\pgfplotsset{width=10cm,compat=1.9}

% imposta lo stile
% usa helvetica
\usepackage[scaled]{helvet}
% usa palatino
\usepackage{palatino}
% usa un font monospazio guardabile
\usepackage{lmodern}

% tikz in sans
\tikzset{every picture/.style={/utils/exec={\sffamily}}}

\renewcommand{\rmdefault}{ppl}
\renewcommand{\sfdefault}{phv}
\renewcommand{\ttdefault}{lmtt}

% circuiti
\usepackage{circuitikz}
\usetikzlibrary{babel}

% disponi il titolo
\makeatletter
\renewcommand{\maketitle} {
	\begin{center} 
		\begin{minipage}[t]{.8\textwidth}
			\textsf{\huge\bfseries \@title} 
		\end{minipage}%
		\begin{minipage}[t]{.2\textwidth}
			\raggedleft \vspace{-1.65em}
			\textsf{\small \@author} \vfill
			\textsf{\small \@date}
		\end{minipage}
		\par
	\end{center}

	\thispagestyle{empty}
	\pagestyle{fancy}
}
\makeatother

% disponi teoremi
\usepackage{tcolorbox}
\newtcolorbox[auto counter, number within=section]{theorem}[2][]{%
	colback=blue!10, 
	colframe=blue!40!black, 
	sharp corners=northwest,
	fonttitle=\sffamily\bfseries, 
	title=Teorema~\thetcbcounter: #2, 
	#1
}

% disponi definizioni
\newtcolorbox[auto counter, number within=section]{definition}[2][]{%
	colback=red!10,
	colframe=red!40!black,
	sharp corners=northwest,
	fonttitle=\sffamily\bfseries,
	title=Definizione~\thetcbcounter: #2,
	#1
}

% disponi problemi
\newtcolorbox[auto counter, number within=section]{problem}[2][]{%
	colback=green!10,
	colframe=green!40!black,
	sharp corners=northwest,
	fonttitle=\sffamily\bfseries,
	title=Problema~\thetcbcounter: #2,
	#1
}

% disponi codice
\usepackage{listings}
\usepackage[table]{xcolor}

\lstdefinestyle{codestyle}{
		backgroundcolor=\color{black!5}, 
		commentstyle=\color{codegreen},
		keywordstyle=\bfseries\color{magenta},
		numberstyle=\sffamily\tiny\color{black!60},
		stringstyle=\color{green!50!black},
		basicstyle=\ttfamily\footnotesize,
		breakatwhitespace=false,         
		breaklines=true,                 
		captionpos=b,                    
		keepspaces=true,                 
		numbers=left,                    
		numbersep=5pt,                  
		showspaces=false,                
		showstringspaces=false,
		showtabs=false,                  
		tabsize=2
}

\lstdefinestyle{shellstyle}{
		backgroundcolor=\color{black!5}, 
		basicstyle=\ttfamily\footnotesize\color{black}, 
		commentstyle=\color{black}, 
		keywordstyle=\color{black},
		numberstyle=\color{black!5},
		stringstyle=\color{black}, 
		showspaces=false,
		showstringspaces=false, 
		showtabs=false, 
		tabsize=2, 
		numbers=none, 
		breaklines=true
}

\lstdefinelanguage{javascript}{
	keywords={typeof, new, true, false, catch, function, return, null, catch, switch, var, if, in, while, do, else, case, break},
	keywordstyle=\color{blue}\bfseries,
	ndkeywords={class, export, boolean, throw, implements, import, this},
	ndkeywordstyle=\color{darkgray}\bfseries,
	identifierstyle=\color{black},
	sensitive=false,
	comment=[l]{//},
	morecomment=[s]{/*}{*/},
	commentstyle=\color{purple}\ttfamily,
	stringstyle=\color{red}\ttfamily,
	morestring=[b]',
	morestring=[b]"
}

% disponi sezioni
\usepackage{titlesec}

\titleformat{\section}
	{\sffamily\Large\bfseries} 
	{\thesection}{1em}{} 
\titleformat{\subsection}
	{\sffamily\large\bfseries}   
	{\thesubsection}{1em}{} 
\titleformat{\subsubsection}
	{\sffamily\normalsize\bfseries} 
	{\thesubsubsection}{1em}{}

% disponi alberi
\usepackage{forest}

\forestset{
	rectstyle/.style={
		for tree={rectangle,draw,font=\large\sffamily}
	},
	roundstyle/.style={
		for tree={circle,draw,font=\large}
	}
}

% disponi algoritmi
\usepackage{algorithm}
\usepackage{algorithmic}
\makeatletter
\renewcommand{\ALG@name}{Algoritmo}
\makeatother

% disponi numeri di pagina
\usepackage{fancyhdr}
\fancyhf{} 
\fancyfoot[L]{\sffamily{\thepage}}

\makeatletter
\fancyhead[L]{\raisebox{1ex}[0pt][0pt]{\sffamily{\@title \ \@date}}} 
\fancyhead[R]{\raisebox{1ex}[0pt][0pt]{\sffamily{\@author}}}
\makeatother

\begin{document}

% sezione (data)
\section{Lezione del 26-02-25}

% stili pagina
\thispagestyle{empty}
\pagestyle{fancy}

% testo
\subsubsection{Modello a stati}
Un possibile approccio alla modellistica di un sistema è la creazione di una \textbf{macchina a stati finiti} (FSM, \textit{Finite State Machine}).

Questo chiaramente porterà alla definizione di un determinato numero di \textbf{stati} in cui il sistema si potrà trovare in qualsiasi momento.
Azioni sul sistema corrisponderanno quindi a \textbf{transizioni} fra uno stato un altro, e ogni transizione comporterà un aggiornamento dell'uscita del sistema.
La risposta del sistema a ogni azione dall'esterno dipende quindi non solo dall'azione, ma dallo stato interno corrente del sistema, che è quindi dotato di \textbf{memoria}.

La modellizzazione con macchine a stati finiti risulta particolarmente utile in informatica, in quanto modellizza molto bene una vasta gamma di sistemi digitali.

\subsection{Modellizzazione a scatola bianca semplice}
Iniziamo a vedere come si possono modellizzare sistemi fisici attraverso approcci a scatola bianca. 

\subsubsection{Esempio: serbatoio a galleggiante}
Prendiamo il caso di un serbatoio, con in in ingresso un rubinetto di portata $p$, di cui si vuole tenere costante il livello del liquido ad un valore $\overline{h}$. 
Sistemi di questo tipo sono comuni in diversi contesti, fra cui gli sciaquoni dei water e le vaschette dei carburatori per motori a scoppio.

Chiamiamo il livello del liquido ad ogni istante temporale $h(t)$.
Il valore desiderato di tale funzione, che indichiamo come $h_o(t) = \overline{h}$, verrà detto \textbf{segnale di riferimento}.

L'ingresso del sistema su cui vogliamo agire sarà $u(t) = p(t)$, cioè la portata del rubinetto (che modificheremo ad esempio controllando una valvola).

Nel caso non ci siano disturbi, ergo tutto il liquido immesso nel serbatoio resti nel serbatoio, avremo una variazione di volume $\Delta V$ pari a:
$$
\Delta V = A \cdot (h_1 - h_0) = \int_{t_0}^{t_1} u(\tau) \, d\tau
$$
dove $A$ rappresenta l'area di base del serbatoio.

Da questo si ricava:
$$
A \cdot \frac{dh}{dt} = u(t)
$$

Chiamando $x \leftarrow h$ si otterrà l'equazione differenziale:
$$
x' = \frac{1}{A}u(t)
$$
Di cui chiaramente ci interessa la \textbf{condizione iniziale} $x_0$, cioè il livello del liquido all'istante temporale $t=0$.

\subsubsection{Esempio: circuito di carica di un condensatore}
Prendiamo il circuito formato da un condensatore $C$ posto in serie ad un generatore di corrente $i$:

\begin{center}
	\begin{circuitikz}
		\draw (0, 0) -- (3, 0)
			to[capacitor, v<=$V$] (3, -3)
			-- (0, -3)
			to[current source, i=$i$] (0, 0);
			
	\end{circuitikz}
\end{center}

Avremo che la variazione di carica sulle armature $\Delta q$ varrà:
$$
\Delta q = C \cdot (V_1 - V_0) = \int_{t_0}^{t_1} i(\tau) d \tau \implies C \frac{dV}{dt} = i(t)
$$
da cui si ricava l'equazione differenziale del tutto analoga a prima:
$$
x' = \frac{1}{C} i(t)
$$

L'unica cosa che cambia, rispetto al sistema precedente, sono i nomi delle variabili e le grandezze fisiche che rappresentano.
Dal punto di vista matematico, quindi, possiamo assumere i due sistemi come equivalenti.

\subsubsection{Esempio: legge di Newton}
Vediamo come equazioni analoghe alle precedenti si ricavano in verità da qualsiasi sistema meccanico che obbedisce alla legge di Newton:
$$
F = m a \implies m \frac{dv}{dt} = F(t)
$$
da cui:
$$
x' = \frac{1}{m}F(t)
$$
notando che chiamiamo $x \leftarrow v$ (ci riferiamo alle velocità, non alle posizioni).

\par\smallskip

Gli elementi visti finora non sono altro che \textbf{integratori} o \textit{operatori integrali}, cioè sistemi dinamici che prendono $x'$ e restituiscono $x$ per una certa variabile.

Indichiamo questi come segue:

\begin{center}
	\begin{tikzpicture}
		\draw (0,0) rectangle (1, 1);
		\draw[-stealth] (-2, 0.5) -> (0, 0.5);
		\draw[-stealth] (1, 0.5) -> (3, 0.5);
		\node at (0.5, 0.5) {$\int$};
		\node at (-1, 0.75) {$x'$};
		\node at (2, 0.7) {$x$};
	\end{tikzpicture}
\end{center}

e, se vogliamo, vedremo che nel dominio di Laplace vale la divisione per la variabile $s = \sigma + i \omega$:

\begin{center}
	\begin{tikzpicture}
		\draw (0,0) rectangle (1, 1);
		\draw[-stealth] (-2, 0.5) -> (0, 0.5);
		\draw[-stealth] (1, 0.5) -> (3, 0.5);
		\node at (0.5, 0.5) {$\frac{1}{s}$};
		\node at (-1, 0.75) {$x'$};
		\node at (2, 0.7) {$x$};
	\end{tikzpicture}
\end{center}

\subsection{Sistemi Lineari e Tempo Invarianti (LTI)}
Vediamo un particolare tipo di sistema dinamico:
\begin{definition}{Sistema LTI}
	Un sistema LTI (Lineare e Tempo Invariante) rispetta le seguenti proprietà:
	\begin{itemize}
		\item \textbf{Omogeneità:} se si scala l'ingresso $u(t)$, l'uscita $y(t)$ viene scalata dello stesso fattore:
			$$
				au(t) \implies ay(t)
			$$

		\item \textbf{Sovrapposizione degli effetti:} se un modello ha risposte $y_1(t)$ e $y_2(t)$ a due ingressi qualsiasi $u_1(t)$ e $u_2(t)$, allora la risposta a un ingresso dato dalla combinazione lineare di $u_1(t)$ e $u_2(t)$:
			$$
				u(t) = \alpha_1 u_1(t) + \alpha_2 u_2(t)
			$$

			sarà data ancora da una combinazione lineare delle uscite in risposta ai singoli ingressi:
			$$
				y(t) = \alpha_1 y_1(t) + \alpha_2 y_2(t)
			$$
			
			Notiamo che in questo la sovrapposizione degli effetti non rappresenta altro che la proprietà di \textit{linearità} dei sistemi lineari.
	
		\item \textbf{Tempo invarianza:} iil sistema si comporta allo stesso modo indipendentemente da quando avviene l'azione, cioè nelle equazioni che lo descrivono non vi è una dipendenza esplicita dal tempo.
			Questo significa che un ingresso $u(t - \tau)$ produce un uscita $y(t - \tau)$, cioè solo traslata in tempo:

		\begin{center}
			\begin{tikzpicture}
				\draw (0,0) rectangle (2, 1);
				\draw[-stealth] (-2, 0.5) -> (0, 0.5);
				\draw[-stealth] (2, 0.5) -> (4, 0.5);
				\node at (1, 0.5) {$\mathrm{LTI}$};
				\node at (-1, 0.7) {$u(t - \tau)$};
				\node at (3, 0.7) {$y(t - \tau)$};
			\end{tikzpicture}
		\end{center}
	\end{itemize}
\end{definition}

I sistemi lineari e tempo invarianti ci sono particolarmente utili in quanto sono completamente compresi e facili da risolvere, e riescono a modellizzare in maniera abbastanza accurata una vasta gamma di fenomeni.

\subsection{Modellizzazione a variabili di stato}
Estendiamo l'approccio a scatola bianca introducendo formalmente una forma che sfrutta equazioni differenziali lineari per modellizzare uno \textit{stato interno} in determinate \textbf{variabili di stato}.

\begin{definition}{Modello a variabili di stato}
	Introduciamo il modello per sistemi a equazioni differenziali lineari:
	\[
		\begin{cases}
			x' = Ax + Bu \\ 
			y = Cx + Du
		\end{cases}
	\]
	dove $u \in \mathbb{R}^r$ è l'\textbf{ingresso} e $x \in \mathbb{R}^n$ è lo \textbf{stato} del sistema.
	$x'$ sarà quindi la variazione dello stato, e $y \in \mathbb{R}^m$ l'uscita del sistema.
\end{definition}

Abbiamo quindi un sistema di equazioni vettoriali differenziali che mettono in relazione fra di loro lo \textit{stato interno presente} $x$, la variazione di stato $x'$, l'ingresso $u$ e l'uscita $y$ del sistema.

Le dimensioni delle matrici che correlano variazione di stato e uscita a stato interno presente e ingresso, cioè le matrici $A, B, C, D$, si ricaveranno dalle dimensioni delle variabili ($r, n$ e $m$):

$$
A \in \mathbb{R}^{n \times n}, \, 
B \in \mathbb{R}^{n \times r}, \, 
C \in \mathbb{R}^{m \times n}, \, 
D \in \mathbb{R}^{m \times r}, \, 
$$

La tempo invarianza del sistema è data proprio dalla tempo invarianza delle matrici, che non hanno il tempo come argomento ma sono costanti.

\subsubsection{Proprietà strutturali}
L'analisi delle matrici $A, B, C, D$ definisce le cosiddette \textbf{proprietà strutturali} del sistema.
Ad esempio abbiamo che:
\begin{itemize}
	\item La matrice $A$ ci dà informazioni riguardo alla proprietà di \textbf{stabilità} del sistema;
	\item Le matrici $A, B$ ci danno informazioni riguardo alla \textbf{controllabilità} del sistema;
	\item Le matrici $A, C$ ci danno informazioni riguardo all'\textbf{osservabilità} del sistema.
\end{itemize}

\subsubsection{Diagramma a blocchi}
Possiamo usare l'operatore integrale definito prima per creare il diagramma a blocchi del sistema a equazioni differenziali linari:

\begin{center}
	\begin{tikzpicture}
		% integratore
		\draw (0,0) rectangle (1, 1);

		% A 
		\draw (0,-1) rectangle (1, -2);
		
		% D
		\draw (0,-3) rectangle (1, -4);
		
		% B 
		\draw (-5,0) rectangle (-4, 1);
		
		% C 
		\draw (3,0) rectangle (4, 1);

		\draw[-stealth] (-2, 0.5) -> (0, 0.5);
		\draw[-stealth] (1, 0.5) -> (3, 0.5);
		
		\draw[-stealth] (-7, 0.5) -> (-5, 0.5);
		\draw[-stealth] (-4, 0.5) -> (-2.1, 0.5);
		
		\draw[-stealth] (4, 0.5) -> (5.9, 0.5);
		\draw[-stealth] (6, 0.5) -> (8, 0.5);

		\draw (2, 0.5) -> (2, -1.5);
		\draw[-stealth] (2, -1.5) -> (1, -1.5);

		\draw (0, -1.5) -> (-2, -1.5);
		\draw[-stealth] (-2, -1.5) -> (-2, 0.4);


		\draw (-6, 0.5) -> (-6, -3.5);
		\draw (-6, -3.5) -> (0, -3.5);

		\draw (1, -3.5) -> (6, -3.5);
		\draw[-stealth] (6, -3.5) -> (6, 0.4);


		\node at (0.5, 0.5) {$\int$};
		\node at (0.5, -1.5) {$A$};
		\node at (0.5, -3.5) {$D$};
		
		\node at (-4.5, 0.5) {$B$};
		\node at (3.5, 0.5) {$C$};

		\node at (-6, 0.75) {$u$};
		\node at (7, 0.75) {$y$};
		
		\node at (-1, 0.75) {$x'$};
		\node at (2, 0.7) {$x$};
	
		% sommatore B - integratore
		\draw[fill=white] (-2, 0.5) circle (0.1);
		
		% sommatore C - uscita 
		\draw[fill=white] (6, 0.5) circle (0.1);

	\end{tikzpicture}
\end{center}

Notando che il simbolo $\circ$ rappresenta una somma.

\subsubsection{Esempio: partitore di tensione}
Vediamo per il modello introdotto l'esempio del \textit{partitore resistivo} o partitore di tensione.

\begin{center}
	\begin{circuitikz}
		\draw (0,0) -- (2,0)
			to[resistor, R=$R_1$] (2,-2)
			to[resistor, R=$R_2$] (2,-4) node[ground]{};

		\draw (2, -2) -- (4, -2);

		\node at (-0.5, 0) {$V_{in}$};
		\node at (4.5, -2) {$V_{out}$};
	\end{circuitikz}
\end{center}

La relazione che otteniamo, dalla legge del partitore di tensione, per il voltaggio, è:
$$
V_{out} = \frac{R_2}{R_1 + R_2} V_{in} = \alpha V_{in}, \quad \alpha = \frac{R_1}{R_1 + R_2}
$$

da cui:
$$
A = B = C = 0, \ D = \alpha 
$$

\subsubsection{Esempio: sistema massa-molla-smorzatore}
Vediamo il sistema fisico dato da un carrello di massa $M$ legato ad una parete da una molla con costante $K$ e sottoposto ad attrito proporzionale alla sua velocità $v$.
Agiamo sul sistema introducendo una \textit{forza} $F$ sul carrello, che consideriamo come \textbf{entrata} del sistema.

Il sistema verrà descritto dall'equazione:
$$
F(t) = M \frac{dv}{dt} + Bv + Kx
$$

Decidiamo di usare come \textbf{variabili di stato} la \textit{posizione} $x$ e la \textit{velocità} $v$: $x_s = \binom{x}{v}$.
Vedremo che questo accade spesso nel caso di sistemi meccanici.
Avremo quindi la variazione di stato:
$$
	\begin{cases}		
		\frac{dx}{dt} = v \\
		\frac{dv}{dt} = -\frac{K}{M}x - \frac{B}{M}v + \frac{1}{M}F
	\end{cases}
$$

Avremo allora la prima equazione del sistema:
$$
\begin{pmatrix}
	\frac{dx}{dt} \\ 
	\frac{dv}{dt}
\end{pmatrix}
=
\begin{pmatrix}
	0 & 1 \\ 
	-\frac{K}{M} & -\frac{B}{M}
\end{pmatrix}
\begin{pmatrix}
	x \\ 
	v
\end{pmatrix}
+
\begin{pmatrix}
	0 \\ 
	\frac{1}{M}
\end{pmatrix}
F
$$
da cui le matrici $A, B$:
$$
A = 
\begin{pmatrix}
	0 & 1 \\ 
	-\frac{K}{M} & -\frac{B}{M}
\end{pmatrix}, \quad 
B = 
\begin{pmatrix}
	0 \\ 
	\frac{1}{M}
\end{pmatrix}
$$

Scegliamo quindi come \textbf{uscita} la \textit{posizione} $x$, come:
$$
y = x = 
\begin{pmatrix}
	1 & 0
\end{pmatrix}
 \binom{x}{v}
$$
da cui le matrici $C, D$:
$$
C =
\begin{pmatrix}
	1 & 0
\end{pmatrix}, \quad 
D =
\begin{pmatrix}
0
\end{pmatrix}
$$ 

Notiamo che si potrebbe scegliere, analogamente, la v\textit{velocità} $v$ come \textbf{uscita}, da cui si otterrebbe:
$$
y = v = 
\begin{pmatrix}
	0 & 1
\end{pmatrix}
 \binom{x}{v}
$$
da cui le matrici $C, D$:
$$
C =
\begin{pmatrix}
	0 & 1
\end{pmatrix}, \quad 
D =
\begin{pmatrix}
0
\end{pmatrix}
$$

\par\medskip 

Abbiamo quindi visto come il modello a variabili di stato permette di rappresentare una vasta gamma di situazioni fisiche con facilità e secondo un formato standard, con cui sono tra l'altro compatibili diversi sistemi di calcolo e simulazione al calcolatore.

\subsubsection{Limiti dei sistemi lineari}
Notiamo come il mondo reale spesso è \textbf{non lineare}.
Prendiamo il caso del serbatoio visto prima, e assumiamo la sezione $A$ come non costante lungo l'altezza $h$, ergo esiste una funzione $A(h)$ che lega altezza a sezione.
Avremo quindi la relazione:
$$
h' = B(h) u, \quad B(h) = \frac{1}{A_h} \implies A(h) \, dh = u \, dt
$$

Al momento di apertura della valvola di uscita avremo quindi una variazione del livello:
$$
h' = B(h)(u - y) = \frac{u - y}{A(h)}
$$

Assunta \textit{sezione} in variazione lineare, potremo sfruttare la legge:
$$
y = k\sqrt(h)
$$
da cui:
$$
h' = \frac{-k\sqrt(h)}{A(h)} + \frac{u}{A(h)}
$$

Notiamo come questa forma si avvicina al modello standard visto prima, ma presenta una relazinoe non lineare all'altezza $h$.
Indichiamo quindi questa relazione come:
$$
h' = g(h) + f(h) u
$$

Chiamiamo questa forma \textbf{forma bilineare}.
\begin{definition}{Forma bilineare}
	Introduciamo la forma bilineare:
	\[
		\begin{cases}
			x' = f(x) + g(x) u \\ 
			y = h(x)
		\end{cases}
	\]
\end{definition}

In questo caso, $f(x)$ rappresenta l'evoluzione dello stato indipendente da $u$, e $g(x)$ rappresenta la risposta di stato, variabile (linearmente) in $x$, ad $u$.
La risposta del sistema è state invece modellata come solo dipendente dallo stato interno $x$.
Sistemi di questo tipo sono onnipresenti, ma chiaramente più difficili da gestire.

\end{document}
