
\documentclass[a4paper,11pt]{article}
\usepackage[a4paper, margin=8em]{geometry}

% usa i pacchetti per la scrittura in italiano
\usepackage[french,italian]{babel}
\usepackage[T1]{fontenc}
\usepackage[utf8]{inputenc}
\frenchspacing 

% usa i pacchetti per la formattazione matematica
\usepackage{amsmath, amssymb, amsthm, amsfonts}

% usa altri pacchetti
\usepackage{gensymb}
\usepackage{hyperref}
\usepackage{standalone}

% imposta il titolo
\title{Appunti Fondamenti di Automatica}
\author{Luca Seggiani}
\date{2025}

% disegni
\usepackage{pgfplots}
\pgfplotsset{width=10cm,compat=1.9}

% imposta lo stile
% usa helvetica
\usepackage[scaled]{helvet}
% usa palatino
\usepackage{palatino}
% usa un font monospazio guardabile
\usepackage{lmodern}

% tikz in sans
\tikzset{every picture/.style={/utils/exec={\sffamily}}}

\renewcommand{\rmdefault}{ppl}
\renewcommand{\sfdefault}{phv}
\renewcommand{\ttdefault}{lmtt}

% circuiti
\usepackage{circuitikz}
\usetikzlibrary{babel}

% disponi il titolo
\makeatletter
\renewcommand{\maketitle} {
	\begin{center} 
		\begin{minipage}[t]{.8\textwidth}
			\textsf{\huge\bfseries \@title} 
		\end{minipage}%
		\begin{minipage}[t]{.2\textwidth}
			\raggedleft \vspace{-1.65em}
			\textsf{\small \@author} \vfill
			\textsf{\small \@date}
		\end{minipage}
		\par
	\end{center}

	\thispagestyle{empty}
	\pagestyle{fancy}
}
\makeatother

% disponi teoremi
\usepackage{tcolorbox}
\newtcolorbox[auto counter, number within=section]{theorem}[2][]{%
	colback=blue!10, 
	colframe=blue!40!black, 
	sharp corners=northwest,
	fonttitle=\sffamily\bfseries, 
	title=Teorema~\thetcbcounter: #2, 
	#1
}

% disponi definizioni
\newtcolorbox[auto counter, number within=section]{definition}[2][]{%
	colback=red!10,
	colframe=red!40!black,
	sharp corners=northwest,
	fonttitle=\sffamily\bfseries,
	title=Definizione~\thetcbcounter: #2,
	#1
}

% disponi problemi
\newtcolorbox[auto counter, number within=section]{problem}[2][]{%
	colback=green!10,
	colframe=green!40!black,
	sharp corners=northwest,
	fonttitle=\sffamily\bfseries,
	title=Problema~\thetcbcounter: #2,
	#1
}

% disponi codice
\usepackage{listings}
\usepackage[table]{xcolor}

\lstdefinestyle{codestyle}{
		backgroundcolor=\color{black!5}, 
		commentstyle=\color{codegreen},
		keywordstyle=\bfseries\color{magenta},
		numberstyle=\sffamily\tiny\color{black!60},
		stringstyle=\color{green!50!black},
		basicstyle=\ttfamily\footnotesize,
		breakatwhitespace=false,         
		breaklines=true,                 
		captionpos=b,                    
		keepspaces=true,                 
		numbers=left,                    
		numbersep=5pt,                  
		showspaces=false,                
		showstringspaces=false,
		showtabs=false,                  
		tabsize=2
}

\lstdefinestyle{shellstyle}{
		backgroundcolor=\color{black!5}, 
		basicstyle=\ttfamily\footnotesize\color{black}, 
		commentstyle=\color{black}, 
		keywordstyle=\color{black},
		numberstyle=\color{black!5},
		stringstyle=\color{black}, 
		showspaces=false,
		showstringspaces=false, 
		showtabs=false, 
		tabsize=2, 
		numbers=none, 
		breaklines=true
}

\lstdefinelanguage{javascript}{
	keywords={typeof, new, true, false, catch, function, return, null, catch, switch, var, if, in, while, do, else, case, break},
	keywordstyle=\color{blue}\bfseries,
	ndkeywords={class, export, boolean, throw, implements, import, this},
	ndkeywordstyle=\color{darkgray}\bfseries,
	identifierstyle=\color{black},
	sensitive=false,
	comment=[l]{//},
	morecomment=[s]{/*}{*/},
	commentstyle=\color{purple}\ttfamily,
	stringstyle=\color{red}\ttfamily,
	morestring=[b]',
	morestring=[b]"
}

% disponi sezioni
\usepackage{titlesec}

\titleformat{\section}
	{\sffamily\Large\bfseries} 
	{\thesection}{1em}{} 
\titleformat{\subsection}
	{\sffamily\large\bfseries}   
	{\thesubsection}{1em}{} 
\titleformat{\subsubsection}
	{\sffamily\normalsize\bfseries} 
	{\thesubsubsection}{1em}{}

% disponi alberi
\usepackage{forest}

\forestset{
	rectstyle/.style={
		for tree={rectangle,draw,font=\large\sffamily}
	},
	roundstyle/.style={
		for tree={circle,draw,font=\large}
	}
}

% disponi algoritmi
\usepackage{algorithm}
\usepackage{algorithmic}
\makeatletter
\renewcommand{\ALG@name}{Algoritmo}
\makeatother

% disponi numeri di pagina
\usepackage{fancyhdr}
\fancyhf{} 
\fancyfoot[L]{\sffamily{\thepage}}

\makeatletter
\fancyhead[L]{\raisebox{1ex}[0pt][0pt]{\sffamily{\@title \ \@date}}} 
\fancyhead[R]{\raisebox{1ex}[0pt][0pt]{\sffamily{\@author}}}
\makeatother

\begin{document}

% sezione (data)
\section{Lezione del 30-04-25}

% stili pagina
\thispagestyle{empty}
\pagestyle{fancy}

% testo
\subsection{Criterio di Nyquist}
Il criterio di Nyquist è un metodo per studiare nel piano complesso l'effetto della reazione negativa sui poli del sistema in catena chiusa con controllore $C(s)$, cioè $W(s) = \frac{C(s)G(s)}{1 + C(s)G(s)}$ a partire dalla conoscenza della funzione di trasferimento in catena aperta $G(s)$.

Vediamo quindi la struttura della catena chiusa nel dettaglio:
$$
W(s) = \frac{K_C \cdot G(s)}{1 + K_C \cdot G(s)} = K_C K_E \cdot \frac{n(s)}{d(s) + K \cdot n(s)} = \frac{\hat{G}(s)}{1 + \hat{G}(s)} = \frac{Y(s)}{U(s)}
$$
La stabilità del sistema in ciclo chiuso $W(s)$ sarà data dalle radici dell'equazione caratteristica:
$$
1 + \hat{G}(s) = 0
$$
cioè dai poli in ciclo chiuso.

Vorremo allora imporre le seguenti condizioni di stabilità riguardo ai poli in ciclo chiuso:
\begin{enumerate}
	\item Nessuna radice deve andare nel semipiano $\mathrm{Re} > 0$ al variare del guadagno $K_C$;
	\item $\hat{G}(j \omega) \neq -1$.
\end{enumerate}

A questo punto esplicitiamo il polinomio caratteristico come:
$$
1 + \hat{G}(s) = 1 + K \frac{ \prod_{i = 1}^m (s - z_i) }{ \prod_{i = 1}^n (s - p_i) } \implies 1 + \hat{G}(j \omega) = \alpha \frac{ \prod_{i = 1}^n (j \omega - r_i) }{ \prod_{i = 1}(j \omega - p_i) }
$$
con:
\[
	\alpha = 1 + \hat{G}(\infty) =			
	\begin{cases}
		1, \quad m < n \\
		1 + K, \quad m = n
	\end{cases}
\]

Nota questa, potremmo ricavare una condizione sugli angoli che i punti di $1 + \hat{G}(j\omega)$ spazzano con poli e zeri, in quanto:
\begin{itemize}
	\item Riguardo ai poli si ha:
	\begin{itemize}
		\item I poli a parte reale negativa portano a variazione di fase $-\pi$;
		\item I poli a parte reale nulla portano a variazione di fase $-\pi$;
		\item I poli a parte reale positiva portano a variazione di fase $\pi$.
	\end{itemize}
\item Riguardo agli zeri si ha:
	\begin{itemize}
		\item Gli zeri a parte reale negativa portano a variazione di fase $\pi$;
		\item Gli zeri a parte reale positiva portano a variazione di fase $-\pi$.
	\end{itemize}
\end{itemize}

Questo viene direttamente dal tracciamento degli angoli dei poli e degli zeri in catena chiusa con i punti sul semiasse immaginario positivo.

Chiaramente l'unico punto esente da questa condizione è quello $\hat{G}(j \omega) = -1$, in quanto in tal caso diventa impossibile definire il vettore fra i poli in catena chiusa e il punto sull'asse immaginario.
Abbiamo però che questo è il punto che annulla il denominatore della catena chiusa $W()s$, e quindi lo escludiamo a prescindere.

Definiamo quindi, riguardo all'espressione del polinomio caratteristico appena trovata:
$$
1 + \hat{G}(s) = \alpha \frac{ \prod_{i = 1}^n (j \omega - r_i) }{ \prod_{i = 1}(j \omega - p_i) }
$$
i valori:
\begin{itemize}
	\item $R^{(+)} =$ numero di radici a parte reale positiva;
	\item $R^{(-)} =$ numero di radici a parte reale negativa;
	\item $P^{(+)} =$ numero di poli a parte reale positiva;
	\item $P^{(-)} =$ numero di poli a parte reale negativa;
	\item $P^{(0)} =$ numero di poli sull'asse immaginario.
\end{itemize}
Varranno le identità:
\[
	\begin{cases}
		R^{(+)} + R^{(-)} = n \\
		P^{(+)} + P^{(-)} + P^{(0)} = n
	\end{cases}
\]
e la variazione totale di fase sarà:
$$
\angle 1 + \hat{G}(j \omega) = R^{(+)} \cdot - \pi + R^{(-)} \cdot \pi + P^{(+)} \cdot \pi + P^{(-)} \cdot - \pi + P^{(0)} \cdot - \pi
$$
da cui:
$$
\angle 1 + \hat{G}(j \omega) = \pi \left( -R^{(+)} + R^{(-)} + P^{(+)} - P^{(-)} - P^{(0)} \right) 
= \pi \left( n - 2 R^{(+)} - n + 2 \cdot P^{(+)} \right)
$$
cioè la variazione di fase non dipende dall'ordine $n$ del sistema, ma solo da dove si trovano i poli e le radici a parte reale positiva, e in particolare vale:
$$
\angle 1 + \hat{G}(j \omega) = 2 \pi \left( P^{(+)} - R^{(+)} \right)
$$
A questo punto basterà imporre la condizione di stabilità $R^{(+)} = 0$, per cui:
$$
N_{ao} = P^+
$$
dove $N_{ao}$ è il numero di giri in senso antiorario attorno al punto critico, che per $1 + \hat{G}(j\omega)$ era l'origine e per il comune diagramma di Nyquist della catena aperta sarà $-1$ (o $-\frac{1}{K}$ se la catena aperta è $K \cdot G(j\omega)$).

\end{document}
