
\documentclass[a4paper,11pt]{article}
\usepackage[a4paper, margin=8em]{geometry}

% usa i pacchetti per la scrittura in italiano
\usepackage[french,italian]{babel}
\usepackage[T1]{fontenc}
\usepackage[utf8]{inputenc}
\frenchspacing 

% usa i pacchetti per la formattazione matematica
\usepackage{amsmath, amssymb, amsthm, amsfonts}

% usa altri pacchetti
\usepackage{gensymb}
\usepackage{hyperref}
\usepackage{standalone}

% imposta il titolo
\title{Appunti Fondamenti di Automatica}
\author{Luca Seggiani}
\date{2025}

% disegni
\usepackage{pgfplots}
\pgfplotsset{width=10cm,compat=1.9}

% imposta lo stile
% usa helvetica
\usepackage[scaled]{helvet}
% usa palatino
\usepackage{palatino}
% usa un font monospazio guardabile
\usepackage{lmodern}

% tikz in sans
\tikzset{every picture/.style={/utils/exec={\sffamily}}}

\renewcommand{\rmdefault}{ppl}
\renewcommand{\sfdefault}{phv}
\renewcommand{\ttdefault}{lmtt}

% circuiti
\usepackage{circuitikz}
\usetikzlibrary{babel}

% disponi il titolo
\makeatletter
\renewcommand{\maketitle} {
	\begin{center} 
		\begin{minipage}[t]{.8\textwidth}
			\textsf{\huge\bfseries \@title} 
		\end{minipage}%
		\begin{minipage}[t]{.2\textwidth}
			\raggedleft \vspace{-1.65em}
			\textsf{\small \@author} \vfill
			\textsf{\small \@date}
		\end{minipage}
		\par
	\end{center}

	\thispagestyle{empty}
	\pagestyle{fancy}
}
\makeatother

% disponi teoremi
\usepackage{tcolorbox}
\newtcolorbox[auto counter, number within=section]{theorem}[2][]{%
	colback=blue!10, 
	colframe=blue!40!black, 
	sharp corners=northwest,
	fonttitle=\sffamily\bfseries, 
	title=Teorema~\thetcbcounter: #2, 
	#1
}

% disponi definizioni
\newtcolorbox[auto counter, number within=section]{definition}[2][]{%
	colback=red!10,
	colframe=red!40!black,
	sharp corners=northwest,
	fonttitle=\sffamily\bfseries,
	title=Definizione~\thetcbcounter: #2,
	#1
}

% disponi problemi
\newtcolorbox[auto counter, number within=section]{problem}[2][]{%
	colback=green!10,
	colframe=green!40!black,
	sharp corners=northwest,
	fonttitle=\sffamily\bfseries,
	title=Problema~\thetcbcounter: #2,
	#1
}

% disponi codice
\usepackage{listings}
\usepackage[table]{xcolor}

\lstdefinestyle{codestyle}{
	backgroundcolor=\color{black!5}, 
	commentstyle=\color{codegreen},
	keywordstyle=\bfseries\color{magenta},
	numberstyle=\sffamily\tiny\color{black!60},
	stringstyle=\color{green!50!black},
	basicstyle=\ttfamily\footnotesize,
	breakatwhitespace=false,         
	breaklines=true,                 
	captionpos=b,                    
	keepspaces=true,                 
	numbers=left,                    
	numbersep=5pt,                  
	showspaces=false,                
	showstringspaces=false,
	showtabs=false,                  
	tabsize=2
}

\lstdefinestyle{shellstyle}{
	backgroundcolor=\color{black!5}, 
	basicstyle=\ttfamily\footnotesize\color{black}, 
	commentstyle=\color{black}, 
	keywordstyle=\color{black},
	numberstyle=\color{black!5},
	stringstyle=\color{black}, 
	showspaces=false,
	showstringspaces=false, 
	showtabs=false, 
	tabsize=2, 
	numbers=none, 
	breaklines=true
}

\lstdefinelanguage{javascript}{
	keywords={typeof, new, true, false, catch, function, return, null, catch, switch, var, if, in, while, do, else, case, break},
	keywordstyle=\color{blue}\bfseries,
	ndkeywords={class, export, boolean, throw, implements, import, this},
	ndkeywordstyle=\color{darkgray}\bfseries,
	identifierstyle=\color{black},
	sensitive=false,
	comment=[l]{//},
	morecomment=[s]{/*}{*/},
	commentstyle=\color{purple}\ttfamily,
	stringstyle=\color{red}\ttfamily,
	morestring=[b]',
	morestring=[b]"
}

% disponi sezioni
\usepackage{titlesec}

\titleformat{\section}
{\sffamily\Large\bfseries} 
{\thesection}{1em}{} 
\titleformat{\subsection}
{\sffamily\large\bfseries}   
{\thesubsection}{1em}{} 
\titleformat{\subsubsection}
{\sffamily\normalsize\bfseries} 
{\thesubsubsection}{1em}{}

% disponi alberi
\usepackage{forest}

\forestset{
	rectstyle/.style={
		for tree={rectangle,draw,font=\large\sffamily}
	},
	roundstyle/.style={
		for tree={circle,draw,font=\large}
	}
}

% disponi algoritmi
\usepackage{algorithm}
\usepackage{algorithmic}
\makeatletter
\renewcommand{\ALG@name}{Algoritmo}
\makeatother

% disponi numeri di pagina
\usepackage{fancyhdr}
\fancyhf{} 
\fancyfoot[L]{\sffamily{\thepage}}

\makeatletter
\fancyhead[L]{\raisebox{1ex}[0pt][0pt]{\sffamily{\@title \ \@date}}} 
\fancyhead[R]{\raisebox{1ex}[0pt][0pt]{\sffamily{\@author}}}
\makeatother

\begin{document}

% sezione (data)
\section{Lezione del 08-05-25}

% stili pagina
\thispagestyle{empty}
\pagestyle{fancy}

% testo
Iniziamo quindi a vedere nel dettaglio le \textit{specifiche statiche}, cioè di reiettamento dei disturbi e di errore a regime.

\subsubsection{Reiettamento dei disturbi}
Consideriamo per adesso ingressi a gradino.
Vediamo che per i sistemi di tipo 0 che i guadagni statici $G(0)$ delle funzioni di trasferimento in anello chiuso sono i valori della risposta a regime.
Questo viene dal teorema del valor finale (vedi teorema 11.2), che afferma:
$$
\lim_{t \rightarrow +\infty} y(t) = \lim_{s \rightarrow 0} s \cdot Y(s)
$$
Perciò, posto $D(s) = \frac{1}{s}$ e le altre componenti a 0, si ha:
$$
\lim_{t \rightarrow +\infty} y(t) = \lim_{s \rightarrow 0} s \cdot C(s) G(s) \cdot \frac{1}{s} = G(0)
$$
che è esattamente il guadagno statico. \qed

Prendendo quindi il modello a disturbo interno visto in 25.2.3, idealmente vorremmo i guadagni statici:
$$
W(0) = 1, \quad W_d(0) =0
$$
cioè guadagno per il trasferimento in retroazione pari a 1 e per i disturbi pari a 0.
Questa si verifica per alti guadagni $C(0) >> 1$, cioè alti guadagni della funzione di catena $L(s) = C(s) G_1(s) G_2(s)$, in quanto secondo tale ipotesi vale:
$$
\lim_{s \rightarrow 0} W(s) = \lim_{s \rightarrow 0} \frac{C(s) G_1(s) G_2(s)}{1 + C(s) G_1(s) G_2(s)} \sim  \lim_{s \rightarrow 0} \frac{C(s) G_1(s) G_2(s)}{C(s) G_1(s) G_2(s)} \approx 1
$$
$$
\lim_{s \rightarrow 0} W_d(s) = \lim_{s \rightarrow 0} \frac{G_2(s)}{1 + C(s) G_1(s) G_2(s)} \sim \lim_{s \rightarrow 0} \frac{1}{C(s) G_1(s)} \approx 0
$$

Nel caso ideale, il guadagno a alle basse frequenze si massimizza con poli all'origine.
Poniamo ad esempio un modello con disturbo a gradino:
$$
D(s) = \frac{d_0}{s}
$$
e \textit{controllore integratore}, cioè che introduce un polo all'origine:
$$
C(s) = \frac{C_0(s)}{s}
$$

In questo caso l'uscita di disturbo $Y_d(s)$ sarà:
$$
Y_d(s) = \frac{G_2(s)}{1 + C(s) G_1(s) G_2(s)} \cdot D(s) = \frac{G_2(s)}{1 + C(s) G_1(s) G_2(s)} \cdot \frac{d_0}{s}
$$
dal teorema del valor finale varrà:
$$
\lim_{t \rightarrow + \infty} y_d(t) = \lim_{s \rightarrow 0} s \cdot Y_d(s) = \lim_{s \rightarrow 0} \frac{s G_2(s)}{1 + C(s) G_1(s) G_2(s)} \cdot \frac{d_0}{s} \approx \lim_{s \rightarrow 0} \frac{d_0}{ \frac{C_0(s)}{s} G_1(s) } = 0
$$
dove si nota che è proprio il polo introdotto dal controllore integratore a rendere stabile il sistema, in quanto l'ultimo limite ha valore definito grazie al termine $\frac{C_0(s)}{s}$. 
Fosse stata solo una costante, questo avrebbe avuto un valore definito e quindi si avrebbe avuto un errore pari a:
$$
\varepsilon = \frac{d_0}{C_0(s) G_1(s)}
$$

Si può quindi dare la regola:
\begin{theorem}{Reiettamento dei disturbi a gradino}
	Per controllori di tipo 1, o maggiore di 1, si ha l'annullamento dell'effetto del disturbo a gradino.
\end{theorem}

Quindi affinché a regime venga annullato l'effetto di un disturbo a gradino, occorre che sia presente nella catena diretta, a \textit{monte} del disturbo, un termine integrale.
Chiaramente questo vale solo per sistemi stabili.

\subsubsection{Errore a regime}
Prendiamo lo stesso esempio di prima (quindi con lo stesso controllore integratore $C(s) = \frac{C_0(s)}{s}$), ma poniamo il disturbo $D(s) = 0$ nullo e prendiamo il riferimento a gradino:
$$
R(s) = \frac{R_0}{s}
$$

In questo caso l'errore sarà:
$$
E(s) = \frac{1}{1 + C(s) G_1(s) G_2(s)} \cdot R(s) = \frac{1}{1 + C(s) G_1(s) G_2(s)} \cdot \frac{R_0}{s}
$$

In questo caso, visto che il disturbo interno è nullo e non ci interessa la distinzione fra parti del sistema disturbate e non, prendiamo per il sistema l'unica funzione di trasferimento $P(s)$:
$$
P(s) = G_1(s) G_2(s)
$$

A questo punto avremo che:
\begin{itemize}
	\item Per sistemi di tipo 0 (stabili) si ha l'errore a regime:
		$$
		e(t \rightarrow +\infty) = \frac{1}{1 + C(0) P(0)} = \frac{1}{1 + k_0}
		$$
		dove $k_0$ è il guadagno statico in catena diretta, cioè:
		$$
		k_0 = C(0) P(0)
		$$
	\item Per sistemi di tipo 1 o $> 1$ (stabili) si ha invece l'errore a regime:
		$$
		e(t \rightarrow 0) = 0
		$$
		direttamente dal teorema del valore finale, come nell'esempio precedente.
\end{itemize}

Vediamo quindi di poter dare la regola:
\begin{theorem}{Errore a regime dei riferimenti a gradino}
	Affinché a regime venga annullato l'errore in risposta ad un ingresso a gradino, occorre che nella catena diretta sia presente un termine integrale.
\end{theorem}

Notiamo che la posizione di tale termine integrale conta:
\begin{itemize}
	\item Se il termine integrale è in $G_2(s)$, cioè nella parte del sistema sensibile all'errore, si ha errore nullo ma non reiettamento dell'errore;
	\item Se il termine integrale è in $G_1(s)$ o nel controllore $C(s)$, quindi a monte dell'errore, si ha errore nullo e reiettamento dell'errore. 
\end{itemize}

Abbiamo quindi che la reiezione dell'errore e il corretto tracciamento del riferimento sono sostanzialmente lo stesso problema dal punto di vista matematico: perché si reietti un errore a gradino o si insegua un riferimento a gradino bisogna avere un polo integratore nella funzione di catena chiusa $L(s)$.
Vediamo come questo concetto può essere generalizzato a ingressi di ordine superiore allo 0 (il gradino considerato) finora.

\subsubsection{Ingressi canonici di ordine superiore}
Consideriamo gli ingressi a gradino "generalizzati" (rampa, parabola, ecc...) \textbf{ingressi canonici}.
Possiamo estendere quanto detto finora a tali ingressi e ottenere che l'andamento complessivo dell'errore a regime è:
\begin{table}[H]
	\center \rowcolors{2}{white}{black!10}
	\begin{tabular} { c || c | c | c }
		\bfseries Ingresso $r(t)$ & \bfseries Tipo 0 & \bfseries Tipo 1 & \bfseries Tipo 2 \\
		\hline
		$H(t)$ & $\frac{1}{1 + k_0}$ & $0$ & $0$ \\
		$t \cdot H(t)$ & $\infty$ & $\frac{1}{k_1}$ & $0$ \\
		$0.5 \cdot t^2 \cdot H(t)$ & $\infty$ & $\infty$ & $\frac{1}{k_2}$ \\
	\end{tabular}
\end{table}
e via dicendo.

Questo criterio viene generalizzato nel cosiddetto \textbf{principio del modello interno}:
\begin{theorem}{Principio del modello interno}
	Riguardo ai sistemi visti finora vale:
	\begin{itemize}
		\item 
			Affinché il sistema in catena chiusa abbia \textbf{errore nullo} a regime in risposta ad un ingresso, occorre che la funzione di trasferimento in catena diretta (cioè la funzione prodotto tra il controllore e il processo) sia in grado di generare un modo uguale al modo del segnale di ingresso.
		\item
			Per \textbf{reiettare il disturbo} a regime, occorre che il modo caratteristico sia riprodotto nei blocchi posti a monte del punto di immissione del disturbo stesso.
	\end{itemize}
\end{theorem}

\subsubsection{Configurazioni equivalenti per i disturbi}
Sfruttando l'algebra dei blocchi possiamo portare il blocco $G_2(s)$ a valle dell'errore sulla catena chiusa, e quindi includere l'errore derivante $D_1(s)$ nella catena chiusa, cioè dire:
\begin{center}
	\begin{tikzpicture}
		\draw (1,0) rectangle (3, 1);
		\node at (2, 0.5) {$G_1(s)$};

		\draw (-3,0) rectangle (-1, 1);
		\node at (-2, 0.5) {$C(s)$};

		\draw[-stealth] (-7, 0.5) -> (-5.1, 0.5);
		\draw[-stealth] (-5, 0.5) -> (-3, 0.5);
		\draw[-stealth] (-1, 0.5) -> (1, 0.5);
		\draw[-stealth] (3, 0.5) -> (7, 0.5);

		\draw (5, 0.5) -> (5, -1);
		\draw (-5, -1) -> (5, -1);
		\draw[-stealth] (-5, -1) -> (-5, 0.4);

		\draw[-stealth] (4, 3.5) -> (4, 2.5);
		\draw[-stealth] (4, 1.5) -> (4, 0.6);
		\node at (4, 3.75) {$D(s)$};

		\node at (4.6, 1) {$D_1(s)$};

		\draw (3,2.5) rectangle (5, 1.5);
		\node at (4, 2) {$G_2(s)$};

		\draw[fill=white] (-5, 0.5) circle (0.1);
		\draw[fill=white] (4, 0.5) circle (0.1);

		\node at (-6, 0.75) {$R(s)$};
		\node at (6, 0.75) {$Y(s)$};

		\node at (-4.75, 0.75) {$+$};
		\node at (-4.75, 0.25) {$-$};

		\node at (-3.8, 0.75) {$E(s)$};
	\end{tikzpicture}
\end{center}

Questo è utile in quanto spesso in sistemi reali abbiamo configurazioni del tipo:

\begin{center}
	\begin{tikzpicture}
		\draw (1,0) rectangle (3, 1);
		\node at (2, 0.5) {$G(s)$};

		\draw (-3,0) rectangle (-1, 1);
		\node at (-2, 0.5) {$R(s)$};

		\draw[-stealth] (-7, 0.5) -> (-5.1, 0.5);
		\draw[-stealth] (-5, 0.5) -> (-3, 0.5);
		\draw[-stealth] (-1, 0.5) -> (1, 0.5);
		\draw[-stealth] (3, 0.5) -> (7, 0.5);

		\draw (5, 0.5) -> (5, -1);
		\draw (-5, -1) -> (5, -1);
		\draw[-stealth] (-5, -1) -> (-5, 0.4);

		\draw[-stealth] (4, 1.5) -> (4, 0.6);
		\node at (4, 1.75) {$D(s)$};

		\draw[-stealth] (-4, 1.5) -> (-4, 0.6);
		\node at (-4, 1.75) {$N(s)$};

		\draw[fill=white] (-5, 0.5) circle (0.1);

		\draw[fill=white] (4, 0.5) circle (0.1);
		\draw[fill=white] (-4, 0.5) circle (0.1);

		\node at (-6, 0.75) {$Y_{rif}(s)$};
		\node at (-3.5, 0.75) {$E(s)$};
		\node at (0, 0.75) {$U(s)$};
		\node at (6, 0.75) {$Y(s)$};

		\node at (-4.75, 0.75) {$+$};
		\node at (-4.75, 0.25) {$-$};
	\end{tikzpicture}
\end{center}

Dove individuiamo due termini di disturbo:
\begin{itemize}
	\item $N(s)$, detto \textbf{rumore di misura};
	\item $D(s)$ detto \textbf{disturbo di carico}.
\end{itemize}

Notiamo inoltre che qui $R(s)$ è il \textit{regolatore} (cioè il controllore) e non il segnale di riferimento.

In questo caso valgono le relazioni:
\begin{itemize}
	\item Errore con solo riferimento:
		$$
		E(s) = \frac{1}{1 + R(s) G(s)} Y_{rif}(s)
		$$
	\item Controllo con solo riferimento:
		$$
		U(s) = \frac{R(s)}{1 + R(s) G(s)} Y_{rif}(s)
		$$
	\item Uscita con solo riferimento:
		$$
		Y = \frac{R(s) G(s)}{1 + R(s) G(s)}
		$$
	\item Uscita con solo disturbo di carico:
		$$
		Y(s) = \frac{1}{1 + R(s) G(s)} D(s)
		$$
	\item Uscita con solo rumore di misura:
		$$
		Y(s) = \frac{R(s) G(s)}{1 + R(s) G(s)} N(s)
		$$
\end{itemize}
dove possiamo comunque studiare separatamente uscite a regime e disturbi grazie al principio di sovrapposizione degli effetti.

In ogni caso, potremo calcolare la risposta complessiva tenendo conto di tutti gli ingressi, risolvendo il sistema:
\[
	\begin{cases}
		Y(s) = D(s) + R(s) G(s) E(s) \\
		E(s) = Y_{rif}(s) - Y(s) + N(s)
	\end{cases}
\]
da cui sostituendo:
$$
Y(s) = D(s) + R(s) G(s) \left[ Y_{rif}(s) - Y(s) + N(s) \right]
$$
$$
= D(s) + R(s) G(s) Y_{rif}(s) - R(s) G(s) Y(s) + R(s) G(s) N(s)
$$
$$
\implies Y(s) \left[ 1 + R(s) G(s) \right] = D(s) + R(s) G(s) Y_{rif}(s) + R(s) G(s N(s))
$$
$$
\implies Y(s) = \frac{ D(s) + R(s) G(s) \left[ Y_{rif}(s) + N(s) \right] }{ 1 + R(s) G(s) }
$$

Da queste relazioni ricaviamo due valori di proporzionalità che sono:
\begin{itemize}
	\item \textbf{Sensitività}, a cui è proporzionale l'effetto del disturbo $D(s$): 
		$$
		S(s) = \frac{1}{1 + R(s) G(s)}
		$$
	\item \textbf{Sensitività complementare:} a cui sono proporzionali gli effetti del riferimento $Y_{rif}(s)$ e del rumore $N(s)$:
		$$
		T(s) = \frac{R(s) G(s)}{1 + R(s) G(s)}
		$$
\end{itemize}
da cui chiaramente:
$$
S(s) + T(s) = \frac{1}{1 + R(s) G(s)} + \frac{R(s) G(s)}{1 + R(s) G(s)} = 1
$$
e si può riscrivere $Y(s)$ come:
$$
Y(s) = \frac{1}{1 + R(s) G(s)} D(s) + \frac{R(s) G(s)}{1 + R(s) G(s)} \left[ Y_{rif}(s) + N(s) \right] 
$$
$$
= S(s) \cdot D(s) + T(s) \cdot \left[ Y_{rif}(s) + N(s) \right]
$$

Abbiamo poi che il termine che compare nella sensitività e nella sensitività complementare:
$$
L(s) = R(s) G(s)
$$
viene detta \textbf{funzione di anello}, e che tutte le funzioni di trasferimento di anello hanno per denominatore $1 + L(s)$, per cui la stabilità di questa determina la stabilità di tutte le altre.

\subsubsection{Specifiche della risposta a gradino}
Definiamo quindi le seguenti caratteristiche d interesse per la risposta al gradino (che avevamo già visto nel dettaglio a 12.1.2 e poi in 14.1):
\begin{enumerate}
	\item Tempo di salita $T_s$, il tempo impiegato dall'uscita per passare dal 10\% al 90\% (o dal 95\% al 5\%) del valore di riferimento;
	\item Tempo di assestamento, $T_a$, il tempo oltre il quale l'uscita si discosta meno del 5\% rispetto al valore di riferimento (con specifiche più restrittive si considera anche 2\%);
	\item Tempo di ritardo $T_r$, tempo richiesto perché l'uscita raggiunga il 50\% del valore di riferimento;
	\item Istante di massima elongazione $T_m$, il punto dove l'uscita raggiunge il picco massimo di sovraelongazione;
	\item Massima sovraelongazione $S(\%)$, valore del massimo scostamento rispetto al valore di regime, solitamente espresso in percentuale rispetto al valore di regime stesso.
\end{enumerate}

\subsubsection{Specifiche della risposta armonica}
Ci possiamo aspettare specifiche anche nel domino della frequenza.

Definiamo innanzitutto la \textbf{banda passante}:
\begin{definition}{Banda passante}
	La banda passante a 3 dB è la regione determinata dalle frequenze per cui il diagramma di modulo di Bode si attenua di un fattore $\sqrt{2}$, cioè di 3 dB rispetto al valore di riferimento $W(s)$ 
\end{definition}

Possiamo quindi definire alcune specifiche di risposta in frequenza:
\begin{itemize}
	\item \textbf{Filtro passa basso:} questo è sostanzialmente un filtro che riproduce il segnale di ingresso in uscita solo se la sua frequenza è bassa, sotto una certa soglia, detta \textit{frequenza di taglio} $\omega_0$.

		Chiaramente questo è il caso ideale, mentre nella pratica si ha un andamento "smussato" attorno alla frequenza di taglio, e nel punto di banda passante il filtro passa basso ha trasferimento di -3 dB rispetto al trasferimento a frequenza bassa ideale $|W(0)|$.

		Vedremo che la maggior parte dei controllori sono essenzialmente filtri passa basso.
	\item \textbf{Filtro passa alto:} di contro, questo è un filtro che riproduce il segnale di ingresso in uscita solo se la sua frequenza è alta, sopra una certa soglia, chiamata sempre frequenza di taglio $\omega_0$.

		Anche qui si ha una modellizzazione ideale che si traduce in un'implementazione "smussata", dove nel punto di banda passante il filtro passa alto ha trasferimento di -3 dB rispetto al trasferimento a frequenza alta ideale $|W(\infty)|$.
	\item \textbf{Filtro passa banda:} la combinazione di un filtro passa basso e di un filtro passa alto dà un filtro passa banda.
		La banda passante in questo caso sarà compresa fra due frequenze di taglio $\omega_0$ e $\omega_1$, punti dove il grafico del modulo di Bode tocca il punto -3 dB (quindi $\frac{1}{\sqrt{2}}$ rispetto a un ipotetico punto posto fra $\omega_0$ e $\omega_1$, assunta fra questi una regione abbastanza grande perché la risposta arrivi a regime).
\end{itemize}

Riguardo a tutti i tipi di filtri possiamo quindi definire la \textbf{risonanza} o \textit{fattore Q}, determinata dal valore di smorzamento $\xi$ del sistema che genera il filtro.

Avevamo visto nel dettaglio in 18.1.3 come determinare il punto $\omega_R$ e il valore del picco di risonanza $M_R$ (chiamandoli con nomi diversi) di un sistema con poli complessi coniugati (cioè di un filtro passa basso risonante). 

\subsubsection{Legami globali}
I legami globali sono legami fra funzione tempo, risposta armonica e funzione di Laplace su tutto il dominio.

Ne riportiamo il più interessante, che afferma che per un sistema passa basso:
$$
B_{3dB} \cdot T_s \approx 0.35
$$
dove $B_{3dB}$ è la larghezza della regione determinata dagli estremi della banda passante, cioè semplicemente il valore della banda passante a 3 dB del filtro passa basso.

In sostanza, questa relazione afferma che più si allarga la banda passante, più diventa rapida la risposta del sistema nel dominio tempo. 

\end{document}
