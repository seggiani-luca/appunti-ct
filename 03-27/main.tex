
\documentclass[a4paper,11pt]{article}
\usepackage[a4paper, margin=8em]{geometry}

% usa i pacchetti per la scrittura in italiano
\usepackage[french,italian]{babel}
\usepackage[T1]{fontenc}
\usepackage[utf8]{inputenc}
\frenchspacing 

% usa i pacchetti per la formattazione matematica
\usepackage{amsmath, amssymb, amsthm, amsfonts}

% usa altri pacchetti
\usepackage{gensymb}
\usepackage{hyperref}
\usepackage{standalone}

% imposta il titolo
\title{Appunti Fondamenti di Automatica}
\author{Luca Seggiani}
\date{2025}

% disegni
\usepackage{pgfplots}
\pgfplotsset{width=10cm,compat=1.9}

% imposta lo stile
% usa helvetica
\usepackage[scaled]{helvet}
% usa palatino
\usepackage{palatino}
% usa un font monospazio guardabile
\usepackage{lmodern}

% tikz in sans
\tikzset{every picture/.style={/utils/exec={\sffamily}}}

\renewcommand{\rmdefault}{ppl}
\renewcommand{\sfdefault}{phv}
\renewcommand{\ttdefault}{lmtt}

% circuiti
\usepackage{circuitikz}
\usetikzlibrary{babel}

% disponi il titolo
\makeatletter
\renewcommand{\maketitle} {
	\begin{center} 
		\begin{minipage}[t]{.8\textwidth}
			\textsf{\huge\bfseries \@title} 
		\end{minipage}%
		\begin{minipage}[t]{.2\textwidth}
			\raggedleft \vspace{-1.65em}
			\textsf{\small \@author} \vfill
			\textsf{\small \@date}
		\end{minipage}
		\par
	\end{center}

	\thispagestyle{empty}
	\pagestyle{fancy}
}
\makeatother

% disponi teoremi
\usepackage{tcolorbox}
\newtcolorbox[auto counter, number within=section]{theorem}[2][]{%
	colback=blue!10, 
	colframe=blue!40!black, 
	sharp corners=northwest,
	fonttitle=\sffamily\bfseries, 
	title=Teorema~\thetcbcounter: #2, 
	#1
}

% disponi definizioni
\newtcolorbox[auto counter, number within=section]{definition}[2][]{%
	colback=red!10,
	colframe=red!40!black,
	sharp corners=northwest,
	fonttitle=\sffamily\bfseries,
	title=Definizione~\thetcbcounter: #2,
	#1
}

% disponi problemi
\newtcolorbox[auto counter, number within=section]{problem}[2][]{%
	colback=green!10,
	colframe=green!40!black,
	sharp corners=northwest,
	fonttitle=\sffamily\bfseries,
	title=Problema~\thetcbcounter: #2,
	#1
}

% disponi codice
\usepackage{listings}
\usepackage[table]{xcolor}

\lstdefinestyle{codestyle}{
		backgroundcolor=\color{black!5}, 
		commentstyle=\color{codegreen},
		keywordstyle=\bfseries\color{magenta},
		numberstyle=\sffamily\tiny\color{black!60},
		stringstyle=\color{green!50!black},
		basicstyle=\ttfamily\footnotesize,
		breakatwhitespace=false,         
		breaklines=true,                 
		captionpos=b,                    
		keepspaces=true,                 
		numbers=left,                    
		numbersep=5pt,                  
		showspaces=false,                
		showstringspaces=false,
		showtabs=false,                  
		tabsize=2
}

\lstdefinestyle{shellstyle}{
		backgroundcolor=\color{black!5}, 
		basicstyle=\ttfamily\footnotesize\color{black}, 
		commentstyle=\color{black}, 
		keywordstyle=\color{black},
		numberstyle=\color{black!5},
		stringstyle=\color{black}, 
		showspaces=false,
		showstringspaces=false, 
		showtabs=false, 
		tabsize=2, 
		numbers=none, 
		breaklines=true
}

\lstdefinelanguage{javascript}{
	keywords={typeof, new, true, false, catch, function, return, null, catch, switch, var, if, in, while, do, else, case, break},
	keywordstyle=\color{blue}\bfseries,
	ndkeywords={class, export, boolean, throw, implements, import, this},
	ndkeywordstyle=\color{darkgray}\bfseries,
	identifierstyle=\color{black},
	sensitive=false,
	comment=[l]{//},
	morecomment=[s]{/*}{*/},
	commentstyle=\color{purple}\ttfamily,
	stringstyle=\color{red}\ttfamily,
	morestring=[b]',
	morestring=[b]"
}

% disponi sezioni
\usepackage{titlesec}

\titleformat{\section}
	{\sffamily\Large\bfseries} 
	{\thesection}{1em}{} 
\titleformat{\subsection}
	{\sffamily\large\bfseries}   
	{\thesubsection}{1em}{} 
\titleformat{\subsubsection}
	{\sffamily\normalsize\bfseries} 
	{\thesubsubsection}{1em}{}

% disponi alberi
\usepackage{forest}

\forestset{
	rectstyle/.style={
		for tree={rectangle,draw,font=\large\sffamily}
	},
	roundstyle/.style={
		for tree={circle,draw,font=\large}
	}
}

% disponi algoritmi
\usepackage{algorithm}
\usepackage{algorithmic}
\makeatletter
\renewcommand{\ALG@name}{Algoritmo}
\makeatother

% disponi numeri di pagina
\usepackage{fancyhdr}
\fancyhf{} 
\fancyfoot[L]{\sffamily{\thepage}}

\makeatletter
\fancyhead[L]{\raisebox{1ex}[0pt][0pt]{\sffamily{\@title \ \@date}}} 
\fancyhead[R]{\raisebox{1ex}[0pt][0pt]{\sffamily{\@author}}}
\makeatother

\begin{document}

% sezione (data)
\section{Lezione del 27-03-25}

% stili pagina
\thispagestyle{empty}
\pagestyle{fancy}

% testo
\subsubsection{$\mathbf \Delta = 0$, poli reali coincidenti}
Vediamo il caso di poli reali coincidenti, cioè $p_1 = p_2$.
Notiamo che questo caso è puramente teorico, in quanto la risposta di un sistema reale sarà necessariamente leggermente sottosmorzata o leggermente sovrasmorzata.

Si ha quindi l'unica proprietà:
$$
T^2 = \frac{1}{p^2} = \frac{a_2}{a_0}
$$
che deriva direttamente dalle precedenti.

Vediamo quindi la \textbf{risposta al gradino}, adottando la stessa forma \textit{"ibrida"} di Evans della scorsa lezione:
$$
Y(s) = G(s) \cdot U(s) = \frac{ \frac{b_0}{a_2} }{ \left( s + \frac{1}{T} \right)^2 } \cdot \frac{1}{s} = \frac{A}{s} + \frac{B}{s + \frac{1}{T}} + \frac{C}{\left( s + \frac{1}{T} \right)^2}
$$

Calcoliamo quindi i residui:
$$
A = \lim_{s \rightarrow 0} \frac{ \frac{b_0}{a_2} }{ \left( s + \frac{1}{T} \right)^2 } = \frac{b_0 \cdot T^2}{a_2} = \frac{b_0}{a_0} = G(0)
$$
$$
C = \lim_{s \rightarrow - \frac{1}{T}} \frac{b_0}{a_2} \cdot \frac{1}{s} = - \frac{b_0 \cdot T}{a_2} = - \frac{b_0}{a_0} \cdot \frac{a_0}{a_2} \cdot T = - G(0) \cdot \frac{T}{T^2} = -\frac{G(0)}{T}
$$
Per calcolare il residuo in $B$ sfruttiamo la proprietà che troviamo sempre, dall'annullamento dei termini in $s^2$ di $A$ e $B$ al numeratore:
$$
A + B = 0 \implies B = -A = -G(0)
$$

Calcoliamo quindi l'antitrasformata:
$$
y(t) = \mathcal{L}^{-1} \{G(s) \cdot U(s)\} = G(0) \cdot \left( 1 - e^{-\frac{t}{T}} \left( 1 + \frac{t}{T} \right) \right) \cdot H(t)
$$
dove per il termine quadrato si ha:
$$
\mathcal{L}^{-1} \left\{ \frac{ k }{ \left( s + \frac{1}{T} \right)^2 } \right\} = k \, t e^{-\frac{t}{T}}
$$
sfruttando la proprietà:
$$
F(s - a) = \mathcal{L} \{ e^{at} \cdot f(t) \}
$$
presa $f(t) = \frac{1}{s^2}$ e $a = -\frac{1}{T}$.

\subsubsection{$\mathbf \Delta < 0$, poli complessi coniugati}
Vediamo infine il caso con poli complessi coniugati.
Avremo quindi che questi rispettano la forma:
$$
p_{1,2} =  - \left( \alpha \pm i \beta \right)
$$
con:
$$
\alpha = -\frac{a_1}{2a_2}, \quad \beta = \sqrt{ \frac{a_0}{a_2} - \left( \frac{a_1}{2 a_2} \right)^2 } = \sqrt{ \frac{a_0}{a_2} \left( 1 - \frac{a_1^2}{4 a_2 a_0} \right) }
$$
che derivano direttamente da $p_1$ e $p_2$ come li avevamo definiti con la formula quadratica (portando $2 a_2$ dentro per $\beta$).

Possiamo ricavare due valori fisicamente significativi, che sono la \textbf{pulsazione di risonanza}:
$$
\omega = \sqrt{\frac{a_0}{a_2}}
$$
e lo \textbf{smorzamento}:
$$
\xi = \frac{a_1}{2 \sqrt{a_0 a_2}}
$$
sapendo che questa situazione darà solitamente comportamenti oscillatori smorzati o meno (non smorzati se $a_1 = 0$).

Varrà allora, rispetto ai poli:
\[
	\begin{cases}
		\alpha = -\xi \omega \\
		\beta = \omega \sqrt{1 - \xi^2}
	\end{cases}
\]

Potremo quindi adottare le forme standard:
\begin{itemize}
	\item \textbf{Forma di Bode:}
$$
G(s) = \frac{ G(0) }{ \frac{1}{\omega} \cdot s^2 + 2 \frac{\xi}{\omega} \cdot s + 1 }
$$
	\item \textbf{Forma di Evans:}
$$
G(s) = \frac{ G(0) \cdot \omega^2 }{ s^2 + 2 \xi \omega \cdot s + \omega^2 }
$$
\end{itemize}

# non so se la risolverà, nel caso risolvila

\subsubsection{Esempio: modello a quarto di automobile}
Prendiamo come esempio di sistema del secondo ordine quello del \textbf{quarto di automobile}, inteso come il sistema formato dalla massa dell'automobile $M_s$, che agisce sulla sospensione (di costante elastica $k_s$) e sull'ammortizzatore (di smorzamento $c$), a loro volta collegati al pneumatico di massa $M_n$, che agisce anch'esso da sospensione (di costante elastica $k_p$) per il collegamento alla strada.

Prendiamo $x_s$ come la posizione verticale dell'auto, $x_n$ come la posizione verticale del pneumatico, e $x_f$ come la posizione verticale della strada (che \textit{varia} mentre la vettura si muove sul tracciato stradale, in base alle asperità stesse dell'asfalto).

Possiamo riportare la costante elastica del pneumatico (di per sé complicata da calcolare, e comunque solitamente abbastanza grande da essere considerata come perfettamente solida) alla sospensione, considerando quindi la sola costante elastica $k_s$ della sospensione.

Prendiamo quindi la posizione verticale dell'automobile $x_s$ come l'uscita del sistema, e la posizione verticale della strada $x_f$ come l'entrata.

Chiamando poi $L_r$ la lunghezza della molla a riposo, potremmo impostare l'equazione differenziale come:
$$
M_s \frac{d^2 x_s}{dt^2} = c \frac{d ( x_f - x_s )}{dt} + k_s (x_f - x_s) + k_s L_r - M_s g
$$
dove il termine $k_s L_r$ deriva effettivamente dal fatto che la molla della sospensione risponde alla variazione della lunghezza a riposo della molla:
$$
F_s \propto  L_r - (x_s - x_f) = x_f - x_s + L_r
$$
mentre tutti i termini derivati non risentono di questa $L_r$ e quindi chiaramente la ignorano.

Dividiamo quindi la differenziale in una soluzione particolare, o di equilibrio, $\overline{x}_s$, e in una soluzione generale $\Delta x_s$:
$$
x_s = \Delta x_s + \overline{x}_s
$$
La condizione di equilibrio di questo sistema sarà quindi, imponendo derivate nulle:
$$
\overline{x}_s = L_r - \frac{M_s g}{k_s}
$$

Potremo allora prendere l'omogenea per il calcolo della soluzione generale:
$$
M_s \frac{d^2 \Delta x_s}{dt^2} = c \frac{d ( x_f - \Delta x_s )}{dt} + k_s (x_f - \Delta x_s)
$$
Raggruppando ingressi e uscite (rispettivamente, $\Delta x_s$ e $x_f$) si ha:
$$
M_s \frac{d^2 \Delta x_s}{dt^2} + c \frac{d \Delta x_s}{dt} + k_s \Delta x_s = c \frac{dx_f}{dt} + k_s x_f
$$
da cui, portandosi, al dominio di Laplace:
$$
\left( s^2 M_s + cs + k_s \right) \Delta x_s = ( cs + k_s ) x_f
$$
troviamo la funzione di trasferimento $G(s)$:
$$
G(s) = \frac{\Delta x_s}{x_f} = \frac{c s + k_s}{s^2 M_s + c s + k_s}
$$

La funzione di trasferimento ha uno zero e due poli.
Possiamo intanto ricavarci pulsazione ($\omega$) e smorzamento ($\xi$):
$$
\omega = \frac{k_s}{M_s}, \quad \xi = \frac{c}{2 \sqrt{k_s M_s}}
$$

\end{document}
