
\documentclass[a4paper,11pt]{article}
\usepackage[a4paper, margin=8em]{geometry}

% usa i pacchetti per la scrittura in italiano
\usepackage[french,italian]{babel}
\usepackage[T1]{fontenc}
\usepackage[utf8]{inputenc}
\frenchspacing 

% usa i pacchetti per la formattazione matematica
\usepackage{amsmath, amssymb, amsthm, amsfonts}

% usa altri pacchetti
\usepackage{gensymb}
\usepackage{hyperref}
\usepackage{standalone}

% imposta il titolo
\title{Appunti Fondamenti di Automatica}
\author{Luca Seggiani}
\date{2025}

% disegni
\usepackage{pgfplots}
\pgfplotsset{width=10cm,compat=1.9}

% imposta lo stile
% usa helvetica
\usepackage[scaled]{helvet}
% usa palatino
\usepackage{palatino}
% usa un font monospazio guardabile
\usepackage{lmodern}

% tikz in sans
\tikzset{every picture/.style={/utils/exec={\sffamily}}}

\renewcommand{\rmdefault}{ppl}
\renewcommand{\sfdefault}{phv}
\renewcommand{\ttdefault}{lmtt}

% circuiti
\usepackage{circuitikz}
\usetikzlibrary{babel}

% disponi il titolo
\makeatletter
\renewcommand{\maketitle} {
	\begin{center} 
		\begin{minipage}[t]{.8\textwidth}
			\textsf{\huge\bfseries \@title} 
		\end{minipage}%
		\begin{minipage}[t]{.2\textwidth}
			\raggedleft \vspace{-1.65em}
			\textsf{\small \@author} \vfill
			\textsf{\small \@date}
		\end{minipage}
		\par
	\end{center}

	\thispagestyle{empty}
	\pagestyle{fancy}
}
\makeatother

% disponi teoremi
\usepackage{tcolorbox}
\newtcolorbox[auto counter, number within=section]{theorem}[2][]{%
	colback=blue!10, 
	colframe=blue!40!black, 
	sharp corners=northwest,
	fonttitle=\sffamily\bfseries, 
	title=Teorema~\thetcbcounter: #2, 
	#1
}

% disponi definizioni
\newtcolorbox[auto counter, number within=section]{definition}[2][]{%
	colback=red!10,
	colframe=red!40!black,
	sharp corners=northwest,
	fonttitle=\sffamily\bfseries,
	title=Definizione~\thetcbcounter: #2,
	#1
}

% disponi problemi
\newtcolorbox[auto counter, number within=section]{problem}[2][]{%
	colback=green!10,
	colframe=green!40!black,
	sharp corners=northwest,
	fonttitle=\sffamily\bfseries,
	title=Problema~\thetcbcounter: #2,
	#1
}

% disponi codice
\usepackage{listings}
\usepackage[table]{xcolor}

\lstdefinestyle{codestyle}{
		backgroundcolor=\color{black!5}, 
		commentstyle=\color{codegreen},
		keywordstyle=\bfseries\color{magenta},
		numberstyle=\sffamily\tiny\color{black!60},
		stringstyle=\color{green!50!black},
		basicstyle=\ttfamily\footnotesize,
		breakatwhitespace=false,         
		breaklines=true,                 
		captionpos=b,                    
		keepspaces=true,                 
		numbers=left,                    
		numbersep=5pt,                  
		showspaces=false,                
		showstringspaces=false,
		showtabs=false,                  
		tabsize=2
}

\lstdefinestyle{shellstyle}{
		backgroundcolor=\color{black!5}, 
		basicstyle=\ttfamily\footnotesize\color{black}, 
		commentstyle=\color{black}, 
		keywordstyle=\color{black},
		numberstyle=\color{black!5},
		stringstyle=\color{black}, 
		showspaces=false,
		showstringspaces=false, 
		showtabs=false, 
		tabsize=2, 
		numbers=none, 
		breaklines=true
}

\lstdefinelanguage{javascript}{
	keywords={typeof, new, true, false, catch, function, return, null, catch, switch, var, if, in, while, do, else, case, break},
	keywordstyle=\color{blue}\bfseries,
	ndkeywords={class, export, boolean, throw, implements, import, this},
	ndkeywordstyle=\color{darkgray}\bfseries,
	identifierstyle=\color{black},
	sensitive=false,
	comment=[l]{//},
	morecomment=[s]{/*}{*/},
	commentstyle=\color{purple}\ttfamily,
	stringstyle=\color{red}\ttfamily,
	morestring=[b]',
	morestring=[b]"
}

% disponi sezioni
\usepackage{titlesec}

\titleformat{\section}
	{\sffamily\Large\bfseries} 
	{\thesection}{1em}{} 
\titleformat{\subsection}
	{\sffamily\large\bfseries}   
	{\thesubsection}{1em}{} 
\titleformat{\subsubsection}
	{\sffamily\normalsize\bfseries} 
	{\thesubsubsection}{1em}{}

% disponi alberi
\usepackage{forest}

\forestset{
	rectstyle/.style={
		for tree={rectangle,draw,font=\large\sffamily}
	},
	roundstyle/.style={
		for tree={circle,draw,font=\large}
	}
}

% disponi algoritmi
\usepackage{algorithm}
\usepackage{algorithmic}
\makeatletter
\renewcommand{\ALG@name}{Algoritmo}
\makeatother

% disponi numeri di pagina
\usepackage{fancyhdr}
\fancyhf{} 
\fancyfoot[L]{\sffamily{\thepage}}

\makeatletter
\fancyhead[L]{\raisebox{1ex}[0pt][0pt]{\sffamily{\@title \ \@date}}} 
\fancyhead[R]{\raisebox{1ex}[0pt][0pt]{\sffamily{\@author}}}
\makeatother

\begin{document}

% sezione (data)
\section{Lezione del 11-03-25}

% stili pagina
\thispagestyle{empty}
\pagestyle{fancy}

% testo
Vediamo alcuni esempi sulla raggiungibilità introdotta alla scorsa lezione.

\subsubsection{Esempio: raggiungibilità della velocità di crociera}
Riprendiamo l'esempio 3.0.1 (in particolare, la linearizzazione data in 3.1.2) e studiamone la raggiungibilità.

Avevamo che il sistema era espresso con le matrici $A, B, C, D$:
\[
	\begin{cases}
		x' = \begin{pmatrix}
			0 & 1 \\
			0 & -\frac{\beta}{m}
		\end{pmatrix} x + \begin{pmatrix}
			0 & 0 \\ 
			\frac{\gamma}{m} & -g
		\end{pmatrix}	u \\ 
		y = \begin{pmatrix}
			0 & 1
		\end{pmatrix} x
	\end{cases}
\]

Ricaviamo quindi la matrice $\mathcal{M}_\mathcal{R}$:
$$
\mathcal{M}_\mathcal{R} = \begin{pmatrix}
	B & A B
\end{pmatrix} = \begin{pmatrix}
	0 & 0 & \frac{\gamma}{m} & -g \\
	\frac{\gamma}{m} & -g & -\frac{\beta \gamma}{m^2} & \frac{g \beta}{m}
\end{pmatrix}
$$
notiamo che le due righe sono necessariamente indipendenti, ergo $\mathrm{rank}(\mathcal{M}_\mathcal{R}) = n = 2$, e il sistema è completamente raggiungibile.

\subsubsection{Esempio: raggiungibilità di una coppia di condensatori}
Vediamo quindi un sistema non completamente raggiungibile per evidenziare come ricavare le parti raggiungibili e non raggiungibili.

Prendiamo il circuito dato dalla coppia di condensatori in serie:
\begin{center}
	\begin{circuitikz}
		\draw (0, 0) to[capacitor, v<=$V_1$] (3, 0)
			to[resistor, l=$R$] (3, -3)
			to[capacitor, v<=$V_2$] (0, -3)
			to[voltage source, v=$u$] (0, 0);	
	\end{circuitikz}
\end{center}

Dal punto di vista fisico avremo che:
$$
i = C \frac{d v_1}{dt} = C \frac{d v_2}{dt}
$$
e dalla seconda di Kirchoff:
$$
u = v_1 + v_2 + i R \implies i = \frac{u - v_1 - v_2}{R}
$$
quindi:
$$
\frac{u - v_1 - v_2}{RC} = \frac{d v_1}{dt} = \frac{d v_2}{dt}
$$
da cui si ricava il sistema finale:
\[
	\begin{cases}
		\frac{d v_1}{dt} = -\frac{1}{RC} (v_1 + v_2) + \frac{1}{RC} u \\ 	
		\frac{d v_2}{dt} = -\frac{1}{RC} (v_1 + v_2) + \frac{1}{RC} u \\ 	
	\end{cases}
\]
con matrici $A$ e $B$ (la $C$ non ci è immediatamente di interesse, e dipenderà comunque dalla variabile che desideriamo osservare):
$$
A = \begin{pmatrix}
	-\frac{1}{RC} & -\frac{1}{RC} \\
	-\frac{1}{RC} & -\frac{1}{RC} \\
\end{pmatrix}, \quad B = \begin{pmatrix}
	\frac{1}{RC} \\ \frac{1}{RC}
\end{pmatrix}
$$

Possiamo quindi calcolare la matrice $\mathcal{M}_\mathcal{R}$:
$$
\mathcal{M}_\mathcal{R} = \begin{pmatrix}
	B & A B
\end{pmatrix} = \begin{pmatrix}
	\frac{1}{RC} & -\frac{2}{R^2C^2} \\ 
	\frac{1}{RC} & -\frac{2}{R^2C^2} \\ 
\end{pmatrix}
$$
che ha chiaramente $\mathrm{rank}(\mathcal{M}_\mathcal{R}) = 1$ (righe linearmente dipendenti), cioè il sistema non è completamente raggiungibile.

In particolare, prendiamo la matrice di trasformazione $T_r$:
$$
T_r = \frac{1}{2} \begin{pmatrix}
	1 & 1 \\ 
	1 & -1
\end{pmatrix}, \quad
T_r^{-1} = \begin{pmatrix}
	1 & 1 \\ 
	1 & -1
\end{pmatrix}
$$

da cui la trasformazione in $\hat{A}$:
$$
\hat{A} = T_r A T_r^{-1} = \frac{1}{2} \begin{pmatrix}
	1 & 1 \\
	1 & -1
\end{pmatrix} \begin{pmatrix}
	\frac{1}{RC} & \frac{1}{RC} \\ 
	\frac{1}{RC} & \frac{1}{RC} 
\end{pmatrix}
\begin{pmatrix}
	-1 & -1 \\
	-1 & 1
\end{pmatrix} = \begin{pmatrix}
	-\frac{2}{RC} & 0 \\
	0 & 0
\end{pmatrix}
$$
e in $\hat{B}$:
$$
\hat{B} = T_R B = \frac{1}{2} \begin{pmatrix}
	1 & 1 \\ 
	1 & -1
\end{pmatrix} \begin{pmatrix}
	\frac{1}{RC} \\ \frac{1}{RC}
\end{pmatrix} = \begin{pmatrix}
	\frac{1}{RC} \\ 0
\end{pmatrix}
$$

Si ricava quindi il sistema:
\[
	\begin{cases}
		\hat{x}_1' = -\frac{1}{RC} (2\hat{x}_1 - u) \\
		\hat{x}_2' = 0
	\end{cases}
\]

Vediamo che la variabile $x_2$ ha derivata nulla indipendentemente dall'ingresso, ergo uno stato $x_2 \neq x_2(0)$ è impossibile da raggiungere.

Notiamo infine che questa è la stessa trasformazione che avremmo potuto ottenere, ad esempio, imponendo $\hat{x}_2 = v_1 - v_2$.
Si sarebbe infatti passati dal sistema:
\[
	\begin{cases}
		x_1' = -\frac{1}{RC} v_1 - \frac{1}{RC} v_2 + \frac{1}{RC} u \\	
		x_2' = -\frac{1}{RC} v_1 - \frac{1}{RC} v_2 + \frac{1}{RC} u \\	
	\end{cases}, \quad 
	x_1 = v_1, \ x_2 = v_2
\]
al sistema:
\[
	\begin{cases}
		x_1' = -\frac{2}{RC} v_1 + \frac{1}{RC} u \\	
		x_2' = 0 \\	
	\end{cases}, \quad 
	x_1 = v_1, \ x_2 = v_1 - v_2
\]
che è esattamente lo stesso di prima.

Dal punto di vista fisico, abbiamo che è semplicemente impossibile variare la tensione $u$ in modo da ottenere cadute di potenziale diverse sui due capacitori, entrambi con capacità $C$.
L'unico modo per ottenere tale situazione sarebbe quello di rendere variabile almeno uno dei capacitori (vedi i vecchi sintonizzatori radio), o variare direttamente la topologia del circuito. 

\subsubsection{Conclusioni sulla raggiungibilità}
Abbiamo quindi che, per sistemi LTI, la raggiungibilità corrisponde con la controllabilità (portare il sistema da uno stato qualsiasi all'origine).
Se la parte non controllabile di un sistema è asintoticamente stabile (cioè ha parte reale degli autovalori $< 0$ il sistema si dice \textbf{stabilizzante}).
Per un sistema completamente controllabile invece esiste sempre almeno un ingresso che permette di spostarsi da uno stato $x_a$ a uno stato $x_b$.

\subsection{Osservabilità}
Vediamo quindi l'ultima proprietà strutturale dei sistemi dinamici: quella di \textbf{osservabilità}.
Questa dipenderà dalle relazioni fra stato e uscita, e quindi dalle matrici $A$ e $C$.

Diamo innanzitutto la definizione:
\begin{definition}{Osservabilità}
	Preso il sistema dinamico di ordine $n$, con $m$ ingressi e $p$ uscite:
	\[
		\begin{cases}
			x' = Ax + Bu \\ 
			y = Cx + Du
		\end{cases}
	\]
	allora uno stato $\overline{x} \neq 0$ si dice non osservabile se, qualunque sia $\overline{t} > 0$ finito, detto $\overline{y}_l(t)$ su $t \geq 0$ il movimento libero dell'uscita generato da $\overline{x}$, risulta $\overline{y}_l(t) = 0$ per $0 \leq t \leq \overline{t}$.
\end{definition}

Come vediamo, l'osservabilità ci dà un indicazione della possibilità di "vedere" la presenza di un certo stato in un certo sistema controllandone l'uscita.


\subsubsection{Completa osservabilità}
Abbiamo quindi che uno stato iniziale $x$ si dice osservabile se è possibile determinare $x$ sulla base della misura delle uscite $y$.
In particolare, un sistema privo di stati non osservabili (cioè un sistema dove tutti gli stati sono osservabili) si dice \textbf{completamente osservabile}.

Si può verificare se un sistema è completamente osservabile sfruttando la matrice (\textit{di osservabilità}):
$$
\mathcal{M}_\mathcal{O} = \begin{pmatrix}
C \\ 
C A \\
C A^2 \\
... \\
C A^{n - 1}
\end{pmatrix}
$$
e verificando se:
$$
\mathrm{rank}(\mathcal{M}_\mathcal{O}) = n
$$

Anche l'osservabilità divide gli stati in una parte osservabile e in una parte non osservabile:
$$
\hat{x} = \begin{pmatrix}
	\hat{x}_o \\ 
	\hat{x}_{no}
\end{pmatrix}
$$
sfruttando la matrice $T_o$:
$$
x' = Ax + Bu, \quad \hat{x} = T_o x \implies \hat{x}' = \hat{A} \hat{x} 
$$
$$
y = Cx, \quad \hat{x} = T_o x \implies y = \hat{C} \hat{x}
$$
con:
$$
\hat{A} = \begin{pmatrix}
	\hat{A} & 0 \\ 
	\hat{A}_{ba} & \hat{A}_b
\end{pmatrix}, \quad 
\hat{C} = \begin{pmatrix}
	\hat{C}_a & 0
\end{pmatrix}
$$
con $\hat{A}_a \in \mathbb{R}^{n_o \times n_o}$ e $\hat{C}_a \in \mathbb{R}^{p \times n_o}$, e con $n_o = \mathrm{rank}(\mathcal{M}_\mathcal{O})$, $n_o < n$.

Svolgendo le moltiplicazioni si avrà che:
\[
	\begin{cases}
		\hat{x}_{o}' = \hat{A}_a \hat{x}_{o} \\	
		\hat{x}_{no}' = \hat{A}_{ba} \hat{x}_{o} + \hat{A}_{b} \hat{x}_{no} \\
		y = \hat{C}_a \hat{x}_{no}
	\end{cases}
\]

\subsubsection{Ricavare la matrice $\mathbf{T_o}$}
Per ricavare la matrice di trasformazione $T_o$ dovremo scegliere $n - n_o$ vettori $\xi_i$ linearmente indipendenti tali che:
$$
\mathcal{M}_\mathcal{O} \xi_i = 0, \quad \mathrm{span}\left( \{ \xi_i \} \right) = x_{no}
$$
tali che compongono una base del $\mathrm{ker}(\mathcal{M}_\mathcal{O})$, ovvero ogni vettore rappresenta uno stato non osservabile.

Si selezionano poi $n_o$ vettori linearmente indipendenti ad aribtrio per completare la matrice, tali che:
$$
\det(T_o^{-1}) \neq 0
$$
cioè $T_o$ è invertibile.

\subsubsection{Esempio: osservabilità di una massa con attrito}
Prendiamo l'esempio di una massa soggetta ad attrito dipendente dalla velocità:
\[
	\begin{cases}
		x_1' = -\frac{c}{m} x_1 + \frac{1}{m} u \\ 
		x_2' = x_1 \\
		y = x_1
	\end{cases}
\]
dove prendiamo $x_1$ come velocità e $x_2$ posizione (contro ogni aspettativa, perchè qui ci piace essere imprevedibili).

Le matrici $A$ e $C$ saranno allora:
$$
A = \begin{pmatrix}
	-\frac{c}{m} & 0 \\
	1 & 0
\end{pmatrix}, \quad C = \begin{pmatrix}
	1 & 0
\end{pmatrix}
$$
e la matrice di osservabilità sarà:
$$
\mathcal{M}_\mathcal{O} = \begin{pmatrix}
	C \\ C A
\end{pmatrix} = \begin{pmatrix}
	1 & 0 \\ 
	-\frac{c}{m} & 0
\end{pmatrix}
$$
che con le righe linearmente dipendenti dà $\mathrm{rank}(\mathcal{M}_\mathcal{O}) = 1$, cioè il sistema non è completamente osservabile (questo sarebbe risultato chiaro anche guardando la definizione di $y$, dove si ha solo lo stato $x_1$ e nessuna informazione riguardo a $x_2$).

\subsubsection{Esempio: osservabilità della velocità di crociera}
Riprendiamo nuovamente l'esempio 3.0.1 (linearizzazione 3.1.2), e studiamone, stavolta, l'osservabilità.
Ricordiamo ancora che il sistema era:
\[
	\begin{cases}
		x' = \begin{pmatrix}
			0 & 1 \\
			0 & -\frac{\beta}{m}
		\end{pmatrix} x + \begin{pmatrix}
			0 & 0 \\ 
			\frac{\gamma}{m} & -g
		\end{pmatrix}	u \\ 
		y = \begin{pmatrix}
			0 & 1
		\end{pmatrix} x
	\end{cases}
\]

Ricaviamo quindi la matrice $\mathcal{M}_\mathcal{O}$:
$$
\mathcal{M}_\mathcal{O} = \begin{pmatrix}
 C \\ C A
\end{pmatrix} = \begin{pmatrix}
	0 & 1 \\
	0 & -\frac{\beta}{m}
\end{pmatrix}
$$

Vediamo subbito che le due righe della matrice sono linearmente dipendenti, ergo $\mathrm{rank}(\mathcal{M}_\mathcal{O}) = 1$ e il sistema non è completamente osservabile.
Nel caso particolare dell'automobile, avremo che sarà impossibile ricostruire la posizione iniziale conoscendo solo la velocità.

\subsection{Scomposizione canonica}
Un sistema può essere sia non completamente osservabile che non completamente raggiungibile.
Esiste una scomposizione che porta il sistema in una forma che ne evidenzia tutte queste caratteristiche, detta \textbf{forma canonica} (\textit{forma canonica di Kalman}), solitamente indicata con $T_k$.

Vogliamo quindi la trasformazione:
$$
\hat{x} = T_k x \rightarrow
\begin{cases}
	\hat{x}' = \hat{A} \hat{x} + \hat{B} u \\
	y = \hat{C} \hat{x} + \hat{D} u
\end{cases}
$$
che prende la forma:
\[
	\begin{cases}			
\hat{x}' = \begin{pmatrix}
	\hat{A}_a & \hat{A}_{ab} & \hat{A}_{ac} & \hat{A}_{ad} & \\
	0 & \hat{A}_b & 0 & \hat{A}_{bd} & \\
	0 & 0 & \hat{A}_c & \hat{A}_{cd} \\ 
	0 & 0 & 0 & \hat{A}_d
\end{pmatrix} \hat{x} + \begin{pmatrix}
	\hat{B}_a \\ \hat{B}_b \\ 0 \\ 0
\end{pmatrix} u \\
y = \begin{pmatrix}
	0 & \hat{C}_b & 0 & \hat{C}_d
\end{pmatrix} \hat{x}
	\end{cases}
\]
con:
$$
\hat{x} = \begin{pmatrix}
	\hat{x}_a \\ 
	\hat{x}_b \\ 
	\hat{x}_c \\ 
	\hat{x}_d \\ 
\end{pmatrix}
$$
dove:
\begin{itemize}
	\item $\hat{x}_a$ rappresenta la parte \textbf{raggiungibile} e \textbf{non osservabile}; 
	\item $\hat{x}_b$ rappresenta la parte \textbf{raggiungibile} e \textbf{osservabile}; 
	\item $\hat{x}_c$ rappresenta la parte \textbf{non raggiungibile} e \textbf{non osservabile}; 
	\item $\hat{x}_d$ rappresenta la parte \textbf{non raggiungibile} e \textbf{osservabile}; 
\end{itemize}

Notiamo che gli autovalori della matrice triangolare a blocchi $A$ sono gli stessi della matrice $A$ originale (li ricaviamo dalla diagonale $\hat{A}_a, \hat{A}_b, \hat{A}_c, \hat{A}_d,$), ergo queste sono simili, anche se siamo riusciti ad isolare ogni parte di interesse.

\end{document}
