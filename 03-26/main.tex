
\documentclass[a4paper,11pt]{article}
\usepackage[a4paper, margin=8em]{geometry}

% usa i pacchetti per la scrittura in italiano
\usepackage[french,italian]{babel}
\usepackage[T1]{fontenc}
\usepackage[utf8]{inputenc}
\frenchspacing 

% usa i pacchetti per la formattazione matematica
\usepackage{amsmath, amssymb, amsthm, amsfonts}

% usa altri pacchetti
\usepackage{gensymb}
\usepackage{hyperref}
\usepackage{standalone}

% imposta il titolo
\title{Appunti Fondamenti di Automatica}
\author{Luca Seggiani}
\date{2025}

% disegni
\usepackage{pgfplots}
\pgfplotsset{width=10cm,compat=1.9}

% imposta lo stile
% usa helvetica
\usepackage[scaled]{helvet}
% usa palatino
\usepackage{palatino}
% usa un font monospazio guardabile
\usepackage{lmodern}

% tikz in sans
\tikzset{every picture/.style={/utils/exec={\sffamily}}}

\renewcommand{\rmdefault}{ppl}
\renewcommand{\sfdefault}{phv}
\renewcommand{\ttdefault}{lmtt}

% circuiti
\usepackage{circuitikz}
\usetikzlibrary{babel}

% disponi il titolo
\makeatletter
\renewcommand{\maketitle} {
	\begin{center} 
		\begin{minipage}[t]{.8\textwidth}
			\textsf{\huge\bfseries \@title} 
		\end{minipage}%
		\begin{minipage}[t]{.2\textwidth}
			\raggedleft \vspace{-1.65em}
			\textsf{\small \@author} \vfill
			\textsf{\small \@date}
		\end{minipage}
		\par
	\end{center}

	\thispagestyle{empty}
	\pagestyle{fancy}
}
\makeatother

% disponi teoremi
\usepackage{tcolorbox}
\newtcolorbox[auto counter, number within=section]{theorem}[2][]{%
	colback=blue!10, 
	colframe=blue!40!black, 
	sharp corners=northwest,
	fonttitle=\sffamily\bfseries, 
	title=Teorema~\thetcbcounter: #2, 
	#1
}

% disponi definizioni
\newtcolorbox[auto counter, number within=section]{definition}[2][]{%
	colback=red!10,
	colframe=red!40!black,
	sharp corners=northwest,
	fonttitle=\sffamily\bfseries,
	title=Definizione~\thetcbcounter: #2,
	#1
}

% disponi problemi
\newtcolorbox[auto counter, number within=section]{problem}[2][]{%
	colback=green!10,
	colframe=green!40!black,
	sharp corners=northwest,
	fonttitle=\sffamily\bfseries,
	title=Problema~\thetcbcounter: #2,
	#1
}

% disponi codice
\usepackage{listings}
\usepackage[table]{xcolor}

\lstdefinestyle{codestyle}{
		backgroundcolor=\color{black!5}, 
		commentstyle=\color{codegreen},
		keywordstyle=\bfseries\color{magenta},
		numberstyle=\sffamily\tiny\color{black!60},
		stringstyle=\color{green!50!black},
		basicstyle=\ttfamily\footnotesize,
		breakatwhitespace=false,         
		breaklines=true,                 
		captionpos=b,                    
		keepspaces=true,                 
		numbers=left,                    
		numbersep=5pt,                  
		showspaces=false,                
		showstringspaces=false,
		showtabs=false,                  
		tabsize=2
}

\lstdefinestyle{shellstyle}{
		backgroundcolor=\color{black!5}, 
		basicstyle=\ttfamily\footnotesize\color{black}, 
		commentstyle=\color{black}, 
		keywordstyle=\color{black},
		numberstyle=\color{black!5},
		stringstyle=\color{black}, 
		showspaces=false,
		showstringspaces=false, 
		showtabs=false, 
		tabsize=2, 
		numbers=none, 
		breaklines=true
}

\lstdefinelanguage{javascript}{
	keywords={typeof, new, true, false, catch, function, return, null, catch, switch, var, if, in, while, do, else, case, break},
	keywordstyle=\color{blue}\bfseries,
	ndkeywords={class, export, boolean, throw, implements, import, this},
	ndkeywordstyle=\color{darkgray}\bfseries,
	identifierstyle=\color{black},
	sensitive=false,
	comment=[l]{//},
	morecomment=[s]{/*}{*/},
	commentstyle=\color{purple}\ttfamily,
	stringstyle=\color{red}\ttfamily,
	morestring=[b]',
	morestring=[b]"
}

% disponi sezioni
\usepackage{titlesec}

\titleformat{\section}
	{\sffamily\Large\bfseries} 
	{\thesection}{1em}{} 
\titleformat{\subsection}
	{\sffamily\large\bfseries}   
	{\thesubsection}{1em}{} 
\titleformat{\subsubsection}
	{\sffamily\normalsize\bfseries} 
	{\thesubsubsection}{1em}{}

% disponi alberi
\usepackage{forest}

\forestset{
	rectstyle/.style={
		for tree={rectangle,draw,font=\large\sffamily}
	},
	roundstyle/.style={
		for tree={circle,draw,font=\large}
	}
}

% disponi algoritmi
\usepackage{algorithm}
\usepackage{algorithmic}
\makeatletter
\renewcommand{\ALG@name}{Algoritmo}
\makeatother

% disponi numeri di pagina
\usepackage{fancyhdr}
\fancyhf{} 
\fancyfoot[L]{\sffamily{\thepage}}

\makeatletter
\fancyhead[L]{\raisebox{1ex}[0pt][0pt]{\sffamily{\@title \ \@date}}} 
\fancyhead[R]{\raisebox{1ex}[0pt][0pt]{\sffamily{\@author}}}
\makeatother

\begin{document}

% sezione (data)
\section{Lezione del 26-03-25}

% stili pagina
\thispagestyle{empty}
\pagestyle{fancy}

% testo
\subsection{Trasformata di Laplace ed equazioni differenziali}
Vediamo come applicare la trasformata di Laplace nella risoluzione delle equazioni differenziali ordinarie.

\subsubsection{Esempio: risposta al gradino di un sistema del primo ordine}
Prendiamo quindi un equazione differenziale del primo ordine e vediamone la trasformazione nel domino di Laplace:
$$
a_1 \frac{dy}{dt} + a_0 f(t) = b_0 u, \quad y(0) = 0
$$
Ricordiamo quindi la trasformata di Laplace della derivata:
$$
\mathcal{L}\left\{ \frac{dy}{dt} \right\} = sY(s) - y(0) = sY(s)
$$
nel nostro caso $y(0) = 0$.
Otteniamo allora:
$$
a_1 \cdot s Y(s) + a_0 \cdot Y(s) = b_0 \cdot U(s)
$$

Notiamo che quest'equazione non è più differenziale, e possiamo quindi risolverla per via algebrica.Troviamo allora la funzione di trasferimento $G(s)$ come il rapporto uscita/ingresso:
$$
\frac{Y(s)}{U(s)} = \frac{b_0}{a_0 + a_1 s} = G(s) 
$$
L'uscita sarà allora data da:
$$
Y(s) = G(s) \cdot U(s)
$$

Otteniamo ad esempio la risposta a $u(t) = H(t)$, cioè al gradino di trasformata $U(s) = \frac{1}{s}$:
$$
Y(s) = G(s) \cdot \frac{1}{s} = \frac{b_0}{s (a_0 + a_1 s)}
$$

Possiamo applicare il teorema del valor iniziale e del valor finale per verificare, rispettivamente, l'aderenza alle condizioni iniziali ($y(0) = 0$) e l'andamento a $t \rightarrow +\infty$:
\begin{itemize}
	\item Valor iniziale:
		$$
		\lim_{s \rightarrow + \infty} s \cdot Y(s) = 0 =\lim_{t \rightarrow 0} y(t)
		$$
	\item Valor finale:
		$$
		\lim_{s \rightarrow 0} s \cdot Y(s) = \frac{b_0}{a_0} =\lim_{t \rightarrow +\infty} y(t)
		$$
\end{itemize}

Calcoliamo quindi l'antitrasformata.
Rendiamo innanzitutto il polinomio al denominatore \textit{monico}, e scomponiamo:
$$
= \frac{\frac{b_0}{a_1}}{s \left( \frac{a_0}{a_1} + s \right)} = \frac{A}{s} + \frac{B}{\frac{a_0}{a_1} + s}
$$
con:
$$
A = \lim_{s \rightarrow 0} s \cdot \frac{b_0}{s(a_0 + a_1 s)} = \frac{b_0}{a_0}
$$
$$
B = \lim_{s \rightarrow -\frac{a_0}{a_1}} \left( s + \frac{a_0}{a_1} \right)  \frac{\frac{b_0}{a_1}}{s \left( \frac{a_0}{a_1} + s \right)} = -\frac{b_0}{a_0}
$$
da cui:
$$
Y(s) = \frac{b_0}{a_0 s} - \frac{b_0}{a_0 \left( s + \frac{a_0}{a_1} \right)}
$$
e l'antitrasformata:
$$
\mathcal{L}^{-1} \left\{ Y(s) \right\} = \frac{b_0}{a_0} \left( 1 - e^{-\frac{a_0}{a_1}t} \right) \cdot H(t)
$$
Notiamo che questo rispetta le condizioni iniziali, e quindi il teorema del valor iniziale, nonché il limite ad infinito posto dal teorema del valor finale.

\subsubsection{Parametri significativi dei sistemi del primo ordine}
Possiamo dare ad alcune delle grandezze trovate dei nomi significativi dal punto di vista delle grandezze.
Ad esempio, introdurremo:
\begin{itemize}
	\item Il \textbf{guadagno statico} $G(0) = \frac{b_0}{a_0}$;
	\item La \textbf{costante tempo}, o \textit{tempo caratteristico} $T$, data da $\frac{1}{T} = \frac{a_0}{a_1}$.
\end{itemize}

Questo ci permette di riscrivere la risposta $y(t)$ come:
$$
y(t) = G(0) \left( 1 - e^{-\frac{t}{T}} \right)
$$

In particolare, avremo che il valore di $y(T)$ al tempo caratteristico $T$ sarà approssimativamente il $63.2 \%$ del valore del guadagno statico $G(0)$.

Possiamo tracciare un grafico dell'andamento di $y(t)$ in Python, evidenziando il valore raggiunto in $t = T$:
\begin{center}
	\includegraphics[scale=0.62]{../figures/first_degree_step_response.png}
\end{center}

\subsubsection{Tempo di assestamento}
Un'altro valore di interesse è il \textbf{tempo di assestamento} al percentile $i$, inteso come il tempo che serve a un sistema dinamico per rimanere una fascia $\pm i \%$ attorno al valore di regime.

Ad esempio, se vogliamo calcolare il tempo di assestamento al $5\%$, possiamo dire:
$$
y(t) = G(0) \left( 1 - e^{-\frac{t_{ss}}{T}} \right) = 0.95 \cdot G(0) \implies 0.05 = e^{-\frac{t_{ss}}{T}}
$$
da cui:
$$
t_{ss} = T \cdot \ln(20) \approx 3T
$$

\subsection{Forme di Bode e di Evans}
Vediamo un'altra equazione differenziale, dove adesso compare la derivata dell'ingresso:
$$
a_1 \frac{dy}{dt} + a_0 y = b_1 \frac{du}{dt} + b_0 u
$$
Nel dominio di Laplace questa avrà l'aspetto:
$$
a_1 \cdot s Y(s) + a_0 \cdot Y(s) + b_1 \cdot s U(s) + b_0 \cdot U(s)
$$
da cui la funzione di trasferimento:
$$
\frac{Y(s)}{U(s)} = \frac{b_0 + b_1 s}{a_0 + a_1 s} = G(s)
$$

Nota questa funzione di trasferimento, possiamo riportare due forme standard di riscrittura, che ne evidenziano proprietà diverse:
\begin{itemize}
	\item \textbf{Forma di Bode:} evidenzia le costanti tempo del sistema dinamico.
$$
G(s) = \frac{b_0}{a_0} \frac{1 + s \cdot \frac{b_1}{b_0}}{1 + s \cdot \frac{a_1}{a_0}} = G(0) \cdot \frac{1 + s \cdot \tau_z}{1 + s \cdot \tau}
$$
Notiamo che l'effetto della derivata dell'ingresso è stato quello di portare un termine $s$ al numeratore, e quindi quello di introdurre uno zero (a $-\frac{b_0}{b_1}$)nella funzione di trasferimento (zero per cui abbiamo introdotto il tempo caratteristico $\tau_z$).
	\item \textbf{Forma di Evans:} evidenzia le singolarità dinamiche del sistema, ovvero poli e zeri:
		$$
		G(s) = \frac{b_1}{a_1} \frac{s + \frac{b_0}{b_1}}{s + \frac{a_0}{a_1}}
		$$
\end{itemize}

\subsection{Sistemi del secondo ordine}
Abbiamo visto finora sistemi del primo ordine risolti tramite la trasformata di Laplace.
Vediamo adesso il caso dei sistemi del secondo ordine, che ci permettono di modellizzare una vasta gamma di fenomeni di interesse ingegneristico:
$$
s_2 \frac{d^2 y}{dt^2} + a_1 \frac{dy}{dt} + a_0 y = b_0 u
$$
complicando la situazione con condizioni iniziali non nulle ($y(0), y'(0) \neq 0$).

Applichiamo quindi Laplace alle due derivate successive:
$$
\mathcal{L} \left\{ \frac{dy}{dt} \right\} = s \cdot Y(s) - y(0)
$$
$$
\mathcal{L} \left\{ \frac{d^2y}{dt^2} \right\} = s \left( Y(s) - y(0) \right) - y'(0)
$$
da cui ricaviamo la forma:
$$
a_2 \cdot \left( s^2 \cdot Y(s) - s \cdot y(0) - y'(0) \right) + a_1 \cdot \left( s \cdot Y(s) - y(0) \right) + a_0 \cdot Y(s) = b_0 \cdot U(s)
$$

Esplicitando $Y(s)$ si otterrà:
$$
Y(s) = \frac{a_2 s \cdot y(0) + a_2 \cdot y'(0) + a_1 \cdot y(0)}{a_0 + a_1 s + a_2 s^2} + \frac{b_0}{a_0 + a_1 s + a_2 s^2} U(s) 
$$
dove riconosciamo che il primo termine rappresenta la \textbf{risposta libera} e il secondo la \textbf{risposta forzata} del sistema. 

Decidiamo di concentrarci sulla risposta forzata, ricavando la funzione di trasferimento:
$$
G(s) = \frac{Y(s)}{U(s)} = \frac{b_0}{a_0 + a_1 s + a_2 s^2}
$$
da cui i poli al denominatore:
$$
p{1, 2} = \frac{-a_1 \pm \sqrt{a_1^2 - 4 a_0 a_2}}{2 a_2}
$$

Potremmo quindi individuare il \textbf{determinante}:
$$
\Delta = \frac{a_1^2}{4} - a_0a_2
$$
sulla base del quale distinguiamo 3 situazioni:
\begin{itemize}
	\item $\Delta > 0$, si hanno 2 poli \textbf{reali distinti}.

		In questo caso possiamo adottare la \textbf{forma di Bode}:
		$$
		G(s) = G(0) \cdot \frac{1}{(1 + s \cdot T_1)(1 + s \cdot T_2)}
		$$
		con:
		$$
		G(0) = \frac{b_0}{a_0}, \quad T_1 = \frac{1}{p_1}, \quad T_2 = \frac{1}{p_2}
		$$

		Altrimenti, si può adottare la \textbf{forma di Evans}:
		$$
		G(s) = \frac{\frac{b_0}{a_2}}{(s + p_1)(s + p_2)}
		$$

		Notiamo inoltre le proprietà:
		\[
			\begin{cases}
		p_1 \cdot p_2 = \frac{1}{T_1 \cdot T_2} = \frac{a_0}{a_2} \\
		p_1 + p_2 = -\frac{a_1}{a_2}
			\end{cases}
		\]

		Vediamo quindi la risposta al gradino:
		$$
		Y(s) = G(s) \cdot U(s) = \frac{\frac{b_0}{a_2}}{(s + \frac{1}{T_1})(s + \frac{1}{T_2}) \cdot s} = \frac{A}{s} + \frac{B}{s + \frac{1}{T_1}} + \frac{C}{s + \frac{1}{T_2}}
		$$
		dove si è adottata questa forma \textit{"ibrida"} di Evans, dove manteniamo esplicite le costanti tempo.

	Calcoliamo quindi i residui:
	$$
	A = \lim_{s \rightarrow 0} \frac{\frac{b_0}{a_2}}{(s + \frac{1}{T_1})(s + \frac{1}{T_2})} = \frac{b_0}{a_2} \cdot T_1 \cdot T_2 = \frac{b_0 \cdot a_2}{a_2 \cdot a_0} = \frac{b_0}{a_0} = G(0)
	$$
	sfruttando le due proprietà riportate prima.
	Si ha poi:
	$$
	B = \lim_{s \rightarrow - \frac{1}{T_1}} \frac{\frac{b_0}{a_2}}{s (s + \frac{1}{T_2})} 
	= \frac{b_0}{a_2} \cdot \frac{1}{-\frac{1}{T_1} (\frac{1}{T_2} - \frac{1}{T_1})} 
	= \frac{b_0 \cdot T_1^2 T_2}{a_2 (T_2 - T_1)} 
	= \frac{T_1}{T_2 - T_1} \cdot G(0)
	$$
	$$
	C = \lim_{s \rightarrow - \frac{1}{T_2}} \frac{\frac{b_0}{a_2}}{s (s + \frac{1}{T_1})} 
	= \frac{b_0}{a_2} \cdot \frac{1}{-\frac{1}{T_2} (\frac{1}{T_1} - \frac{1}{T_2})} 
	= \frac{b_0 \cdot T_1 T_2^2}{a_2 (T_2 - T_1)} 
	= -\frac{T_2}{T_2 - T_1} \cdot G(0)
	$$

	Calcoliamo quindi l'antitrasformata:
	$$
	y(t) = \mathcal{L}^{-1} \{ G(s) \cdot U(s) \} = G(0) \cdot \left( 1 + \frac{T_1 e^{\frac{-t}{T_1}} - T_2 e^{\frac{-t}{T_2}}}{T_2 - T_1} \right) \cdot H(t)
	$$

	\item $\Delta = 0$, si hanno 2 poli \textbf{reali coincidenti};
	\item $\Delta < 0$, si hanno 2 poli \textbf{complessi coniugati};
\end{itemize}

\end{document}
