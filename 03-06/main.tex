
\documentclass[a4paper,11pt]{article}
\usepackage[a4paper, margin=8em]{geometry}

% usa i pacchetti per la scrittura in italiano
\usepackage[french,italian]{babel}
\usepackage[T1]{fontenc}
\usepackage[utf8]{inputenc}
\frenchspacing 

% usa i pacchetti per la formattazione matematica
\usepackage{amsmath, amssymb, amsthm, amsfonts}

% usa altri pacchetti
\usepackage{gensymb}
\usepackage{hyperref}
\usepackage{standalone}

% imposta il titolo
\title{Appunti Fondamenti di Automatica}
\author{Luca Seggiani}
\date{2025}

% disegni
\usepackage{pgfplots}
\pgfplotsset{width=10cm,compat=1.9}

% imposta lo stile
% usa helvetica
\usepackage[scaled]{helvet}
% usa palatino
\usepackage{palatino}
% usa un font monospazio guardabile
\usepackage{lmodern}

% tikz in sans
\tikzset{every picture/.style={/utils/exec={\sffamily}}}

\renewcommand{\rmdefault}{ppl}
\renewcommand{\sfdefault}{phv}
\renewcommand{\ttdefault}{lmtt}

% circuiti
\usepackage{circuitikz}
\usetikzlibrary{babel}

% disponi il titolo
\makeatletter
\renewcommand{\maketitle} {
	\begin{center} 
		\begin{minipage}[t]{.8\textwidth}
			\textsf{\huge\bfseries \@title} 
		\end{minipage}%
		\begin{minipage}[t]{.2\textwidth}
			\raggedleft \vspace{-1.65em}
			\textsf{\small \@author} \vfill
			\textsf{\small \@date}
		\end{minipage}
		\par
	\end{center}

	\thispagestyle{empty}
	\pagestyle{fancy}
}
\makeatother

% disponi teoremi
\usepackage{tcolorbox}
\newtcolorbox[auto counter, number within=section]{theorem}[2][]{%
	colback=blue!10, 
	colframe=blue!40!black, 
	sharp corners=northwest,
	fonttitle=\sffamily\bfseries, 
	title=Teorema~\thetcbcounter: #2, 
	#1
}

% disponi definizioni
\newtcolorbox[auto counter, number within=section]{definition}[2][]{%
	colback=red!10,
	colframe=red!40!black,
	sharp corners=northwest,
	fonttitle=\sffamily\bfseries,
	title=Definizione~\thetcbcounter: #2,
	#1
}

% disponi problemi
\newtcolorbox[auto counter, number within=section]{problem}[2][]{%
	colback=green!10,
	colframe=green!40!black,
	sharp corners=northwest,
	fonttitle=\sffamily\bfseries,
	title=Problema~\thetcbcounter: #2,
	#1
}

% disponi codice
\usepackage{listings}
\usepackage[table]{xcolor}

\lstdefinestyle{codestyle}{
		backgroundcolor=\color{black!5}, 
		commentstyle=\color{codegreen},
		keywordstyle=\bfseries\color{magenta},
		numberstyle=\sffamily\tiny\color{black!60},
		stringstyle=\color{green!50!black},
		basicstyle=\ttfamily\footnotesize,
		breakatwhitespace=false,         
		breaklines=true,                 
		captionpos=b,                    
		keepspaces=true,                 
		numbers=left,                    
		numbersep=5pt,                  
		showspaces=false,                
		showstringspaces=false,
		showtabs=false,                  
		tabsize=2
}

\lstdefinestyle{shellstyle}{
		backgroundcolor=\color{black!5}, 
		basicstyle=\ttfamily\footnotesize\color{black}, 
		commentstyle=\color{black}, 
		keywordstyle=\color{black},
		numberstyle=\color{black!5},
		stringstyle=\color{black}, 
		showspaces=false,
		showstringspaces=false, 
		showtabs=false, 
		tabsize=2, 
		numbers=none, 
		breaklines=true
}

\lstdefinelanguage{javascript}{
	keywords={typeof, new, true, false, catch, function, return, null, catch, switch, var, if, in, while, do, else, case, break},
	keywordstyle=\color{blue}\bfseries,
	ndkeywords={class, export, boolean, throw, implements, import, this},
	ndkeywordstyle=\color{darkgray}\bfseries,
	identifierstyle=\color{black},
	sensitive=false,
	comment=[l]{//},
	morecomment=[s]{/*}{*/},
	commentstyle=\color{purple}\ttfamily,
	stringstyle=\color{red}\ttfamily,
	morestring=[b]',
	morestring=[b]"
}

% disponi sezioni
\usepackage{titlesec}

\titleformat{\section}
	{\sffamily\Large\bfseries} 
	{\thesection}{1em}{} 
\titleformat{\subsection}
	{\sffamily\large\bfseries}   
	{\thesubsection}{1em}{} 
\titleformat{\subsubsection}
	{\sffamily\normalsize\bfseries} 
	{\thesubsubsection}{1em}{}

% disponi alberi
\usepackage{forest}

\forestset{
	rectstyle/.style={
		for tree={rectangle,draw,font=\large\sffamily}
	},
	roundstyle/.style={
		for tree={circle,draw,font=\large}
	}
}

% disponi algoritmi
\usepackage{algorithm}
\usepackage{algorithmic}
\makeatletter
\renewcommand{\ALG@name}{Algoritmo}
\makeatother

% disponi numeri di pagina
\usepackage{fancyhdr}
\fancyhf{} 
\fancyfoot[L]{\sffamily{\thepage}}

\makeatletter
\fancyhead[L]{\raisebox{1ex}[0pt][0pt]{\sffamily{\@title \ \@date}}} 
\fancyhead[R]{\raisebox{1ex}[0pt][0pt]{\sffamily{\@author}}}
\makeatother

\begin{document}

% sezione (data)
\section{Lezione del 06-03-25}

% stili pagina
\thispagestyle{empty}
\pagestyle{fancy}

% testo
\subsection{Forma di Jordan con autovalori complessi e coniugati}
Nel caso si trovi $A$ matrice reale con autovalori complessi e coniugati, è possibile usare un cambio di coordinate che trasforma la matrice diagonale complessa in una matrice reale diagonale a blocchi.

Questo è il caso che abbiamo già visto di:
$$
\begin{pmatrix}
	\sigma & \omega \\
	-\omega & \sigma
\end{pmatrix}
\sim
\begin{pmatrix}
	\sigma + j\omega & 0 \\
	0 & \sigma - j \omega
\end{pmatrix}
$$
e:
$$
\begin{pmatrix}
	\sigma & \omega & 1 & 0 \\
	-\omega & \sigma & 0 & 1 \\
	0 & 0 & \sigma & \omega \\
	0 & 0 & -\omega & \sigma \\
\end{pmatrix}
\sim
\begin{pmatrix}
	\sigma + j \omega & 1 & 0 & 0 \\
	0 & \sigma + j \omega & 0 & 0 \\
	0 & 0 & \sigma - j \omega & 1 \\
	0 & 0 & 0 & \sigma - j \omega
\end{pmatrix} 
$$

In generale, quindi, abbiamo matrici di Jordan:
$$
J_r = \begin{pmatrix}
	M & I & 0 & ... & 0 \\
	0 & M & I & ... & 0 \\
	0 & ... & M & I & 0 \\ 
	0 & ... &  0 & M & I \\
	0 & ... & ... & 0 & M
\end{pmatrix}
$$
dove gli $M$ rappresentano i singoli miniblocchi:
$$
M = \begin{pmatrix}
	\sigma & \omega \\
	-\omega & \sigma
\end{pmatrix}
$$

Il numero di blocchi è quindi pari al numero di coppie di autovalori complessi coniugati.

\subsection{Raggiungibilità}
Abbiamo visto come la proprietà di stabilita dipende solo dalla struttura del sistema, e in particolare dalla sola matrice $A$.

Vediamo come in verità esistono altre proprietà che dipendono dalla struttura del sistema e che ci sono di interesse dal punto di vista della regolazione automatica.
Una di queste proprietà è la \textbf{raggiungibilità}.

Diamo quindi la definizione:
\begin{definition}{Raggiungibilità}
	Dato il sistema dinamico di ordine $n$ con $m$ ingressi e $p$ uscite:
	\[
		\begin{cases}
			x' = Ax + Bu \\
			y = Cx + Du
		\end{cases}
	\]
		allora uno stato $\overline{x}$ si dice raggiungibile se esistono un istante di tempo finito $\overline{t} > 0$ e un ingresso $\overline{u}$ definito tra $0$ e $\overline{t}$ tali che, detto $\overline{x}_f(t)$ il movimento forzato dello stato generato da $\overline{u}$, risulti che $\overline{x}_f(\overline{t}) = \overline{x}$.
\end{definition}

La proprieta di raggiungibilità degli stati divide gli stessi in due categorie: stati raggiungibili e stati non raggiungibili.

\subsubsection{Completa raggiungibilità}
Se tutti gli stati di un sistema sono raggiungibili, allora il sistema è detto \textbf{completamente raggiungibile}.

Si può verificare se un sistema è completamente raggiungible sfruttando la matrice:

$$
\mathcal{M}_\mathcal{R} = \begin{pmatrix}
	B & AB & A^B & ... & A^{n - 1}B
\end{pmatrix}
$$
e verificando se:
$$
\mathrm{rank}(\mathcal{M}_\mathcal{R}) = n
$$

Nel caso un sistema non sia completamente raggiungibile si può isolare la parte raggiungibile, cioè definire una trasformazione $T_r$ che ci porti:
$$
x' = Ax + Bu, \quad \hat{x} = T_r x \implies \hat{x}' = \hat{A} \hat{x} + \hat{B} u
$$
con:
$$
\hat{A} = \begin{pmatrix}
\hat{A}_a & \hat{A}_{ab} \\
0 & \hat{A}_b
\end{pmatrix}, \quad
\hat{B} = \begin{pmatrix}
	\hat{B}_a \\
	0
\end{pmatrix}
$$

con $\hat{A}_a \in \mathbb{R}^{n_r \times n_r}$ e $\hat{B}_a \in \mathbb{R}^{n_r \times m}$, con $n_r = \mathrm{rank}(\mathcal{M}_\mathcal{R})$.

Posto:
$$
\hat{x} = \begin{pmatrix}
	\hat{x}_r \\ 
	\hat{x}_{nr}
\end{pmatrix}
$$
si avrà svolgendo le moltiplicazioni che:
\[
	\begin{cases}
		\hat{x}_r' = \hat{A}_a \hat{x}_r + \hat{A}_{ab} \hat{x}_{nr} + \hat{B}_a u\\
		\hat{x}_{nr}' = \hat{A}_b \hat{x}_{nr}
	\end{cases}
\]
ovvero si divide lo stato in una parte raggiungibile e in una parte non raggiungibile.

\subsubsection{Ricavare la matrice $\mathbf{T_r}$}
Per ricavare la matrice di trasformazione $T_r$ basterà scegliere $n_r$ colonne linearmente indipendenti di $\mathcal{M}_\mathcal{R}$.
Ogni stato raggiungibile sarà combinazione lineare di queste colonne.
Si aggiungono poi $n - n_r$ colonne linearmente indipendenti, prese ad arbitrio.

\end{document}
