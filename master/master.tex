
\documentclass[a4paper,11pt]{article}
\usepackage[a4paper, margin=8em]{geometry}

% usa i pacchetti per la scrittura in italiano
\usepackage[french,italian]{babel}
\usepackage[T1]{fontenc}
\usepackage[utf8]{inputenc}
\frenchspacing 

% usa i pacchetti per la formattazione matematica
\usepackage{amsmath, amssymb, amsthm, amsfonts}

% usa altri pacchetti
\usepackage{gensymb}
\usepackage{hyperref}
\usepackage{standalone}

% cose fluttuanti
\usepackage{float}

% imposta il titolo
\title{Appunti Fondamenti di Automatica}
\author{Luca Seggiani}
\date{2025}

% disegni
\usepackage{pgfplots}
\pgfplotsset{width=10cm,compat=1.9}

% imposta lo stile
% usa helvetica
\usepackage[scaled]{helvet}
% usa palatino
\usepackage{palatino}
% usa un font monospazio guardabile
\usepackage{lmodern}

% tikz in sans
\tikzset{every picture/.style={/utils/exec={\sffamily}}}

\renewcommand{\rmdefault}{ppl}
\renewcommand{\sfdefault}{phv}
\renewcommand{\ttdefault}{lmtt}

% circuiti
\usepackage{circuitikz}
\usetikzlibrary{babel}

% disponi il titolo
\makeatletter
\renewcommand{\maketitle} {
	\begin{center} 
		\begin{minipage}[t]{.8\textwidth}
			\textsf{\huge\bfseries \@title} 
		\end{minipage}%
		\begin{minipage}[t]{.2\textwidth}
			\raggedleft \vspace{-1.65em}
			\textsf{\small \@author} \vfill
			\textsf{\small \@date}
		\end{minipage}
		\par
	\end{center}

	\thispagestyle{empty}
	\pagestyle{fancy}
}
\makeatother

% disponi teoremi
\usepackage{tcolorbox}
\newtcolorbox[auto counter, number within=section]{theorem}[2][]{%
	colback=blue!10, 
	colframe=blue!40!black, 
	sharp corners=northwest,
	fonttitle=\sffamily\bfseries, 
	title=Teorema~\thetcbcounter: #2, 
	#1
}

% disponi definizioni
\newtcolorbox[auto counter, number within=section]{definition}[2][]{%
	colback=red!10,
	colframe=red!40!black,
	sharp corners=northwest,
	fonttitle=\sffamily\bfseries,
	title=Definizione~\thetcbcounter: #2,
	#1
}

% disponi problemi
\newtcolorbox[auto counter, number within=section]{problem}[2][]{%
	colback=green!10,
	colframe=green!40!black,
	sharp corners=northwest,
	fonttitle=\sffamily\bfseries,
	title=Problema~\thetcbcounter: #2,
	#1
}

% disponi codice
\usepackage{listings}
\usepackage[table]{xcolor}

\lstdefinestyle{codestyle}{
		backgroundcolor=\color{black!5}, 
		commentstyle=\color{codegreen},
		keywordstyle=\bfseries\color{magenta},
		numberstyle=\sffamily\tiny\color{black!60},
		stringstyle=\color{green!50!black},
		basicstyle=\ttfamily\footnotesize,
		breakatwhitespace=false,         
		breaklines=true,                 
		captionpos=b,                    
		keepspaces=true,                 
		numbers=left,                    
		numbersep=5pt,                  
		showspaces=false,                
		showstringspaces=false,
		showtabs=false,                  
		tabsize=2
}

\lstdefinestyle{shellstyle}{
		backgroundcolor=\color{black!5}, 
		basicstyle=\ttfamily\footnotesize\color{black}, 
		commentstyle=\color{black}, 
		keywordstyle=\color{black},
		numberstyle=\color{black!5},
		stringstyle=\color{black}, 
		showspaces=false,
		showstringspaces=false, 
		showtabs=false, 
		tabsize=2, 
		numbers=none, 
		breaklines=true
}

\lstdefinelanguage{javascript}{
	keywords={typeof, new, true, false, catch, function, return, null, catch, switch, var, if, in, while, do, else, case, break},
	keywordstyle=\color{blue}\bfseries,
	ndkeywords={class, export, boolean, throw, implements, import, this},
	ndkeywordstyle=\color{darkgray}\bfseries,
	identifierstyle=\color{black},
	sensitive=false,
	comment=[l]{//},
	morecomment=[s]{/*}{*/},
	commentstyle=\color{purple}\ttfamily,
	stringstyle=\color{red}\ttfamily,
	morestring=[b]',
	morestring=[b]"
}

% disponi sezioni
\usepackage{titlesec}

\titleformat{\section}
	{\sffamily\Large\bfseries} 
	{\thesection}{1em}{} 
\titleformat{\subsection}
	{\sffamily\large\bfseries}   
	{\thesubsection}{1em}{} 
\titleformat{\subsubsection}
	{\sffamily\normalsize\bfseries} 
	{\thesubsubsection}{1em}{}

% disponi alberi
\usepackage{forest}

\forestset{
	rectstyle/.style={
		for tree={rectangle,draw,font=\large\sffamily}
	},
	roundstyle/.style={
		for tree={circle,draw,font=\large}
	}
}

% disponi algoritmi
\usepackage{algorithm}
\usepackage{algorithmic}
\makeatletter
\renewcommand{\ALG@name}{Algoritmo}
\makeatother

% disponi numeri di pagina
\usepackage{fancyhdr}
\fancyhf{} 
\fancyfoot[L]{\sffamily{\thepage}}

\makeatletter
\fancyhead[L]{\raisebox{1ex}[0pt][0pt]{\sffamily{\@title \ \@date}}} 
\fancyhead[R]{\raisebox{1ex}[0pt][0pt]{\sffamily{\@author}}}
\makeatother

\begin{document}

\pagestyle{fancy}
\thispagestyle{empty}
\renewcommand{\thispagestyle}[1]{}

\maketitle

\documentclass[a4paper,11pt]{article}
\usepackage[a4paper, margin=8em]{geometry}

% usa i pacchetti per la scrittura in italiano
\usepackage[french,italian]{babel}
\usepackage[T1]{fontenc}
\usepackage[utf8]{inputenc}
\frenchspacing 

% usa i pacchetti per la formattazione matematica
\usepackage{amsmath, amssymb, amsthm, amsfonts}

% usa altri pacchetti
\usepackage{gensymb}
\usepackage{hyperref}
\usepackage{standalone}

% imposta il titolo
\title{Appunti Fondamenti di Automatica}
\author{Luca Seggiani}
\date{2025}

% disegni
\usepackage{pgfplots}
\pgfplotsset{width=10cm,compat=1.9}

% imposta lo stile
% usa helvetica
\usepackage[scaled]{helvet}
% usa palatino
\usepackage{palatino}
% usa un font monospazio guardabile
\usepackage{lmodern}

% tikz in sans
\tikzset{every picture/.style={/utils/exec={\sffamily}}}

\renewcommand{\rmdefault}{ppl}
\renewcommand{\sfdefault}{phv}
\renewcommand{\ttdefault}{lmtt}

% disponi il titolo
\makeatletter
\renewcommand{\maketitle} {
	\begin{center} 
		\begin{minipage}[t]{.8\textwidth}
			\textsf{\huge\bfseries \@title} 
		\end{minipage}%
		\begin{minipage}[t]{.2\textwidth}
			\raggedleft \vspace{-1.65em}
			\textsf{\small \@author} \vfill
			\textsf{\small \@date}
		\end{minipage}
		\par
	\end{center}

	\thispagestyle{empty}
	\pagestyle{fancy}
}
\makeatother

% disponi teoremi
\usepackage{tcolorbox}
\newtcolorbox[auto counter, number within=section]{theorem}[2][]{%
	colback=blue!10, 
	colframe=blue!40!black, 
	sharp corners=northwest,
	fonttitle=\sffamily\bfseries, 
	title=Teorema~\thetcbcounter: #2, 
	#1
}

% disponi definizioni
\newtcolorbox[auto counter, number within=section]{definition}[2][]{%
	colback=red!10,
	colframe=red!40!black,
	sharp corners=northwest,
	fonttitle=\sffamily\bfseries,
	title=Definizione~\thetcbcounter: #2,
	#1
}

% disponi problemi
\newtcolorbox[auto counter, number within=section]{problem}[2][]{%
	colback=green!10,
	colframe=green!40!black,
	sharp corners=northwest,
	fonttitle=\sffamily\bfseries,
	title=Problema~\thetcbcounter: #2,
	#1
}

% disponi codice
\usepackage{listings}
\usepackage[table]{xcolor}

\lstdefinestyle{codestyle}{
		backgroundcolor=\color{black!5}, 
		commentstyle=\color{codegreen},
		keywordstyle=\bfseries\color{magenta},
		numberstyle=\sffamily\tiny\color{black!60},
		stringstyle=\color{green!50!black},
		basicstyle=\ttfamily\footnotesize,
		breakatwhitespace=false,         
		breaklines=true,                 
		captionpos=b,                    
		keepspaces=true,                 
		numbers=left,                    
		numbersep=5pt,                  
		showspaces=false,                
		showstringspaces=false,
		showtabs=false,                  
		tabsize=2
}

\lstdefinestyle{shellstyle}{
		backgroundcolor=\color{black!5}, 
		basicstyle=\ttfamily\footnotesize\color{black}, 
		commentstyle=\color{black}, 
		keywordstyle=\color{black},
		numberstyle=\color{black!5},
		stringstyle=\color{black}, 
		showspaces=false,
		showstringspaces=false, 
		showtabs=false, 
		tabsize=2, 
		numbers=none, 
		breaklines=true
}

\lstdefinelanguage{javascript}{
	keywords={typeof, new, true, false, catch, function, return, null, catch, switch, var, if, in, while, do, else, case, break},
	keywordstyle=\color{blue}\bfseries,
	ndkeywords={class, export, boolean, throw, implements, import, this},
	ndkeywordstyle=\color{darkgray}\bfseries,
	identifierstyle=\color{black},
	sensitive=false,
	comment=[l]{//},
	morecomment=[s]{/*}{*/},
	commentstyle=\color{purple}\ttfamily,
	stringstyle=\color{red}\ttfamily,
	morestring=[b]',
	morestring=[b]"
}

% disponi sezioni
\usepackage{titlesec}

\titleformat{\section}
	{\sffamily\Large\bfseries} 
	{\thesection}{1em}{} 
\titleformat{\subsection}
	{\sffamily\large\bfseries}   
	{\thesubsection}{1em}{} 
\titleformat{\subsubsection}
	{\sffamily\normalsize\bfseries} 
	{\thesubsubsection}{1em}{}

% disponi alberi
\usepackage{forest}

\forestset{
	rectstyle/.style={
		for tree={rectangle,draw,font=\large\sffamily}
	},
	roundstyle/.style={
		for tree={circle,draw,font=\large}
	}
}

% disponi algoritmi
\usepackage{algorithm}
\usepackage{algorithmic}
\makeatletter
\renewcommand{\ALG@name}{Algoritmo}
\makeatother

% disponi numeri di pagina
\usepackage{fancyhdr}
\fancyhf{} 
\fancyfoot[L]{\sffamily{\thepage}}

\makeatletter
\fancyhead[L]{\raisebox{1ex}[0pt][0pt]{\sffamily{\@title \ \@date}}} 
\fancyhead[R]{\raisebox{1ex}[0pt][0pt]{\sffamily{\@author}}}
\makeatother

\begin{document}

% sezione (data)
\section{Lezione del 25-02-25}

% stili pagina
\thispagestyle{empty}
\pagestyle{fancy}

% testo
\subsubsection{Introduzione al corso}
Il corso di fondamenti di automatica introduce i concetti di base dell'automazione:
\begin{enumerate}
	\item Introduzione all'automazione;
	\item Modellistica matematica (variabili di stato, trasformata di Laplace, ecc...);
	\item Analisi dei sistemi dinamici (funzioni di trasferimento, ecc...);
	\item Strumenti per l'analisi dei sistemi dinamici (diagrammi di Bode, Nyquist, ecc...);
	\item Sistemi di controllo (PID, ecc...)
\end{enumerate}

\subsection{Introduzione all'automazione}
L'automazione si può intendere come la capacità di eseguire un compito \textit{in modo automatico}.
Per ogni \textit{compito} visto durante il corso si creerà un modello matematico, e un sistema capace di eseguirlo in maniera autonoma, senza l'intervento di esterni.

Elemento chiave nei sistemi che verranno studiati sarà il \textbf{feedback}, in italiano \textit{retroazione}, che rappresenta l'informazione che possiamo prendere indietro dal sistema in modo da influenzare i sistemi di controllo automatico.

\subsubsection{Fasi di sviluppo}
Il termine inglese \textit{"automation"} viene introdotto come contrazione di \textit{"automatic production"} dalla Ford Motor Company nel 1947 per denominare l'insieme di apparati di movimentazione automatica installati nelle loro linee di produzione.

Possiamo tracciare diverse fasi di sviluppo della disciplina dell'automazione:
\begin{enumerate}
	\item \textbf{Prima rivoluzione industriale} ($\sim$1780): \\
		Vengono introdotti i primi strumenti meccanici di produzione (macchine a vapore, ecc...).
	\item \textbf{Seconda rivoluzione industriale} ($\sim$1870): \\
		Si organizza il lavoro in catene di produzione sfruttando l'energia elettrica (catena Ford, ecc...).
	\item \textbf{Terza rivoluzione industriale} ($\sim$1970): \\ 
		Si automatizza il processo di produzione grazie a sistemi IT ed elettronici (PLC, ecc...).
	\item \textbf{Quarta rivoluzione industriale} ($\sim$2010): \\
		Prodotti e processi interconnessi grazie all'IoT e a tecnologie digitali (industria 4.0, ecc...).
\end{enumerate}

\subsubsection{Tecnologie abilitanti}
Possiamo individuare alcune macroaree dell'automazione moderna, in ambito (perlopiù) industriale:
\begin{itemize}
	\item Tecniche di produzione avanzate (stampa 3D (additiva), ecc...);
	\item Realtà aumentata;
	\item Simulazione;
	\item Sviluppo orizzontale e verticale;
	\item IoT industriale (IIoT);
	\item Cloud computing;
	\item Cybersecurity;
	\item Big data analytics.
\end{itemize}

\subsection{Modellistica matematica}
\subsubsection{Sistemi}
Veniamo quindi alle tecniche che ci permettono di studiare matematicamente un dato \textbf{sistema}.
\begin{definition}{Sistema}
	Un sistema è un insieme di parti interconnesse e interagenti che formano un insieme più grande e complesso.
\end{definition}

Di un sistema ci interessano gli \textbf{ingressi}, cioè cosa \textit{entra} nel sistema, e le \textbf{uscite}, cioè cosa \textit{esce} dal sistema.
Ogni sistema è caratterizzato poi da un certo livello di \textbf{disturbo}.
L'idea è spesso quella di \textit{inseguire} gli ingressi e \textit{reiettare} i disturbi.

\subsubsection{Sistemi dinamici}
Ad interessarci in particolare sono  sistemi \textbf{dinamici}.
\begin{definition}{Sistema dinamico}
	Un sistema si dice dinamico quando le sue grandezze si evolvono nel tempo.
\end{definition}

Le grandezze di un sistema dinamico (le uscite e lo stato interno) sono quindi caratterizzate da funzioni con variabile tempo, che quindi \textit{variano} nel tempo e interagiscono con l'ambiente esterno.

Solitamente i sistemi dinamici sono costituiti da più sottosistemi che interagiscono fra di loro.

\par\medskip

Incontriamo diversi problemi nello studio dei sistemi:
\begin{itemize}
	\item Se è ignoto il sistema, abbiamo il problema della \textbf{modellistica}, che consiste nel creare un modello matematico del sistema, e quindi del suo comportamento;
	\item Se è ignota l'uscita, abbiamo il problema della \textbf{simulazione}, che consiste nel simulare la risposta del sistema, nel tempo, in base alla variazione degli ingressi;
	\item Se conosciamo il sistema e la sua uscita, abbiamo il problema del \textbf{controllo}, che consiste nello studiare come agire \textit{dall'esterno} su un sistema per modificarne la naturale evoluzione ed ottenere un comportamento desiderato.
		Ad agire sul sistema sarà spesso un altro sistema, detto \textbf{sistema di controllo}, che produrrà gli ingressi del sistema vero e proprio sulla base di dati \textbf{comandi}.
\end{itemize}

\subsubsection{Diagrammi a blocchi}
Rappresentiamo i modelli che facciamo sistemi attraverso diagrammi a \textit{"scatole"} o a \textbf{blocchi}:

\begin{center}
	\begin{tikzpicture}
		\draw[fill=cyan] (0,0) rectangle (2, 1);
		\draw[-stealth] (-2, 0.5) -> (0, 0.5);
		\draw[-stealth] (2, 0.5) -> (4, 0.5);
		\draw[-stealth] (1, 2) -> (1, 1);

		\node at (1, 0.5) {Sistema};
		\node at (-1, 0.8) {Ingressi};
		\node at (3, 0.8) {Uscite};
		\node at (1, 2.3) {Disturbi};
	\end{tikzpicture}
\end{center} 

La scatola sistema è spesso caratterizzata da una certa funzione matematica $f(u, \xi, \theta)$, dove $\xi$ rappresenta i disturbi e $\theta$ i parametri del modello.
Gli ingressi e le uscite saranno quindi rappresentati da variabili $u$ e $y$, con $y = f(u, \xi, \theta)$. 
Notiamo che la funzione che rappresenta il sistema ha la variabile di ingresso $u$ come argomento.

\begin{center}
	\begin{tikzpicture}
		\draw (0,0) rectangle (2, 1);
		\draw[-stealth] (-2, 0.5) -> (0, 0.5);
		\draw[-stealth] (2, 0.5) -> (4, 0.5);
		\node at (1, 0.5) {$f(u, \xi, \theta)$};
		\node at (-1, 0.7) {$u$};
		\node at (3, 0.7) {$y$};
	\end{tikzpicture}
\end{center}

\subsubsection{Proprietà dei sistemi dinamici}
Notiamo alcune proprietà dei sistemi dinamici:
\begin{itemize}
	\item Questi devono essere \textbf{causali}, cioè l'uscita non deve dipendere da valori futuri dell'ingresso: se l'uscita è rappresentata da $v(t)$ e l'ingresso da $u(t)$, si ha che $v(t)$ dipende da $u(t)$ solo per $t < t_0$ con $t_0$ l'istante corrente;
	\item Possono essere sia \textbf{stocastici} che \textbf{deterministici}, se sono presenti o meno fenomeni aleatori nel legame ingresso uscita. Il corso in particolare tratterà di sistemi \textit{deterministici};
\end{itemize}

\subsection{Modellistica di sistemi}
Vediamo gli approcci più comuni alla modellazione di sistemi dinamici.

\subsubsection{Modello a scatola nera}
Possiamo adottare un approccio sperimentale (o \textit{induttivo}) alla modellistica di un sistema, considerandolo come una \textbf{scatola nera} (\textit{black box}) 

\begin{center}
	\begin{tikzpicture}
		% scatola nera
		\draw[fill=darkgray] (0,0) rectangle (2, 1);
		\node[text width=1.5cm, text centered, color=white] at (1, 0.5) {Scatola nera};
		% misura
		\draw (4,0) rectangle (6, 1);
		\node at (5, 0.5) {Misura};
		
		\draw[-stealth] (-2, 0.5) -> (0, 0.5);
		\draw[-stealth] (2, 0.5) -> (4, 0.5);
		
		\draw[-stealth] (3, -2) -> (3, -3);

		\draw (-1, 0.5) -> (-1, -1.5);
		\draw (5, 0) -> (5, -1.5);
		\draw[-stealth] (-1, -1.5) -> (2, -1.5);
		\draw[-stealth] (5, -1.5) -> (4, -1.5);

		% tabelle
		\draw (2,-2) rectangle (4, -1);
		\node at (3, -1.5) {Tabelle};
		% fitting
		\draw (2,-4) rectangle (4, -3);
		\node at (3, -3.5) {Fitting};
		
		% modello
		\draw[fill=lime] (5,-4) rectangle (7, -3);
		\node at (6, -3.5) {Modello};

		\draw[-stealth] (4, -3.5) -> (5, -3.5); 

		\node at (-1, 0.8) {Ingressi};
	\end{tikzpicture}
\end{center}

L'idea è quella di studiare il comportamento del sistema in risposta a diversi stimoli, e costruire un'associazione fra stimolo e risposta corrispondente.
Il problema (\textit{fitting}) da qui in poi sarà quello di trovare una certa funzione che approssima, per quanto possibile, il comportamento del sistema in risposta ai diversi stimoli.
Approcci per il fitting della funzione di risposta possono essere:
\begin{itemize}
	\item Regole stato-azione;
	\item Alberi decisionali;
	\item Regressione lineare;
	\item Modelli statistici;
	\item Reti neurali.
\end{itemize}

Un approccio di questo tipo non richiede alcuna conoscenza del principio di funzionamento interno del sistema (dettagli fisici, tecnici, ecc...), ed è quindi puramente sperimentale.

Notiamo che approcci di questo tipo sono effettivamente alla base delle tecniche di apprendimento automatico.
In questo le reti neurali per l'apprendimento \textit{supervisionato} non sono che una tecnica molto potente per il fitting di varie relazioni ingresso-uscita.

\subsubsection{Modello a scatola bianca}
Un altro approccio è quello analitico (o \textit{deduttivo}).
In questo caso si costruisce un modello astratto del sistema di interesse, e si \textit{valida} confrontandolo col sistema reale, agendo con delle modifiche nel caso di incongruenze. 

\begin{center}
	\begin{tikzpicture}
		% sistema reale
		\draw[fill=cyan] (0,0) rectangle (2, 1);
		\node[text width=1.5cm, text centered] at (1, 0.5) {Sistema reale};
		% modello astratto
		\draw (0,-2) rectangle (2, -1);
		\node[text width=1.5cm, text centered] at (1, -1.5) {Modello astratto};
		
		\draw[-stealth] (-2, 0.5) -> (0, 0.5);
		\draw[-stealth] (2, 0.5) -> (4, 0.5);
		
		\draw (-1, 0.5) -> (-1, -1.5);
		\draw[stealth-] (5, 0) -> (5, -1.5);
		\draw[-stealth] (-1, -1.5) -> (0, -1.5);
		\draw (5, -1.5) -> (2, -1.5);
		
		\draw (5.5, 0) -> (5.5, -1.7);
		\draw[-stealth] (5.5, -1.7) -> (2, -1.7);

		% validazione
		\draw (4,0) rectangle (6, 1);
		\node at (5, 0.5) {Validazione};
		
		\draw[-stealth] (6, 0.5) -> (7, 0.5);

		% modello 
		\draw[fill=lime] (7,0) rectangle (9, 1);
		\node at (8, 0.5) {Modello};

		\node at (-1, 0.8) {Ingressi};
	
		\node at (3.5, -1.9) {Modifica};
	\end{tikzpicture}
\end{center}

Approcci di questo tipo richiedono di ricavare modelli astratti di sistemi, conoscendone quindi le leggi di funzionamento interno.
Per questo sono molto accurati per sistemi ben conosciuti, ma difficili da realizzare altrimenti.

\subsubsection{Modello a scatola grigia}
Un approccio intermedio fra scatola nera e scatola bianca è quello a \textbf{scatola grigia}.
In questo caso assumiamo di conoscere il comportamento generale del sistema, ma di dover identificare parametri specifici.
Si crea quindi un modello a scatola bianca, lasciando vuoti i parametri che non si conoscono (\textit{scatola bianca}). 
Svolgendo varie misurazioni e confrontando i risultati del modello astratto e del sistema reale si andranno poi a determinare nel dettaglio tali parametri (\textit{scatola nera}).

\subsubsection{Approccio pragmatico}
Un approccio pragmatico consiste nello sviluppare un modello a priori, senza considerare necessariamente un sistema reale, e poi adattarlo fino alla convergenza con un sistema veramente esistente.

Dal punto di vista ingegneristico, quento significa ipotizzare un modello per un sistema ideale, che andiamo quindi a definire in maniera astratta, e poi ad implementare nella realtà cercando di avvicinarci il più possibile all'idea originale. 

\end{document}


\documentclass[a4paper,11pt]{article}
\usepackage[a4paper, margin=8em]{geometry}

% usa i pacchetti per la scrittura in italiano
\usepackage[french,italian]{babel}
\usepackage[T1]{fontenc}
\usepackage[utf8]{inputenc}
\frenchspacing 

% usa i pacchetti per la formattazione matematica
\usepackage{amsmath, amssymb, amsthm, amsfonts}

% usa altri pacchetti
\usepackage{gensymb}
\usepackage{hyperref}
\usepackage{standalone}

% imposta il titolo
\title{Appunti Fondamenti di Automatica}
\author{Luca Seggiani}
\date{2025}

% disegni
\usepackage{pgfplots}
\pgfplotsset{width=10cm,compat=1.9}

% imposta lo stile
% usa helvetica
\usepackage[scaled]{helvet}
% usa palatino
\usepackage{palatino}
% usa un font monospazio guardabile
\usepackage{lmodern}

% tikz in sans
\tikzset{every picture/.style={/utils/exec={\sffamily}}}

\renewcommand{\rmdefault}{ppl}
\renewcommand{\sfdefault}{phv}
\renewcommand{\ttdefault}{lmtt}

% circuiti
\usepackage{circuitikz}
\usetikzlibrary{babel}

% disponi il titolo
\makeatletter
\renewcommand{\maketitle} {
	\begin{center} 
		\begin{minipage}[t]{.8\textwidth}
			\textsf{\huge\bfseries \@title} 
		\end{minipage}%
		\begin{minipage}[t]{.2\textwidth}
			\raggedleft \vspace{-1.65em}
			\textsf{\small \@author} \vfill
			\textsf{\small \@date}
		\end{minipage}
		\par
	\end{center}

	\thispagestyle{empty}
	\pagestyle{fancy}
}
\makeatother

% disponi teoremi
\usepackage{tcolorbox}
\newtcolorbox[auto counter, number within=section]{theorem}[2][]{%
	colback=blue!10, 
	colframe=blue!40!black, 
	sharp corners=northwest,
	fonttitle=\sffamily\bfseries, 
	title=Teorema~\thetcbcounter: #2, 
	#1
}

% disponi definizioni
\newtcolorbox[auto counter, number within=section]{definition}[2][]{%
	colback=red!10,
	colframe=red!40!black,
	sharp corners=northwest,
	fonttitle=\sffamily\bfseries,
	title=Definizione~\thetcbcounter: #2,
	#1
}

% disponi problemi
\newtcolorbox[auto counter, number within=section]{problem}[2][]{%
	colback=green!10,
	colframe=green!40!black,
	sharp corners=northwest,
	fonttitle=\sffamily\bfseries,
	title=Problema~\thetcbcounter: #2,
	#1
}

% disponi codice
\usepackage{listings}
\usepackage[table]{xcolor}

\lstdefinestyle{codestyle}{
		backgroundcolor=\color{black!5}, 
		commentstyle=\color{codegreen},
		keywordstyle=\bfseries\color{magenta},
		numberstyle=\sffamily\tiny\color{black!60},
		stringstyle=\color{green!50!black},
		basicstyle=\ttfamily\footnotesize,
		breakatwhitespace=false,         
		breaklines=true,                 
		captionpos=b,                    
		keepspaces=true,                 
		numbers=left,                    
		numbersep=5pt,                  
		showspaces=false,                
		showstringspaces=false,
		showtabs=false,                  
		tabsize=2
}

\lstdefinestyle{shellstyle}{
		backgroundcolor=\color{black!5}, 
		basicstyle=\ttfamily\footnotesize\color{black}, 
		commentstyle=\color{black}, 
		keywordstyle=\color{black},
		numberstyle=\color{black!5},
		stringstyle=\color{black}, 
		showspaces=false,
		showstringspaces=false, 
		showtabs=false, 
		tabsize=2, 
		numbers=none, 
		breaklines=true
}

\lstdefinelanguage{javascript}{
	keywords={typeof, new, true, false, catch, function, return, null, catch, switch, var, if, in, while, do, else, case, break},
	keywordstyle=\color{blue}\bfseries,
	ndkeywords={class, export, boolean, throw, implements, import, this},
	ndkeywordstyle=\color{darkgray}\bfseries,
	identifierstyle=\color{black},
	sensitive=false,
	comment=[l]{//},
	morecomment=[s]{/*}{*/},
	commentstyle=\color{purple}\ttfamily,
	stringstyle=\color{red}\ttfamily,
	morestring=[b]',
	morestring=[b]"
}

% disponi sezioni
\usepackage{titlesec}

\titleformat{\section}
	{\sffamily\Large\bfseries} 
	{\thesection}{1em}{} 
\titleformat{\subsection}
	{\sffamily\large\bfseries}   
	{\thesubsection}{1em}{} 
\titleformat{\subsubsection}
	{\sffamily\normalsize\bfseries} 
	{\thesubsubsection}{1em}{}

% disponi alberi
\usepackage{forest}

\forestset{
	rectstyle/.style={
		for tree={rectangle,draw,font=\large\sffamily}
	},
	roundstyle/.style={
		for tree={circle,draw,font=\large}
	}
}

% disponi algoritmi
\usepackage{algorithm}
\usepackage{algorithmic}
\makeatletter
\renewcommand{\ALG@name}{Algoritmo}
\makeatother

% disponi numeri di pagina
\usepackage{fancyhdr}
\fancyhf{} 
\fancyfoot[L]{\sffamily{\thepage}}

\makeatletter
\fancyhead[L]{\raisebox{1ex}[0pt][0pt]{\sffamily{\@title \ \@date}}} 
\fancyhead[R]{\raisebox{1ex}[0pt][0pt]{\sffamily{\@author}}}
\makeatother

\begin{document}

% sezione (data)
\section{Lezione del 26-02-25}

% stili pagina
\thispagestyle{empty}
\pagestyle{fancy}

% testo
\subsubsection{Modello a stati}
Un possibile approccio alla modellistica di un sistema è la creazione di una \textbf{macchina a stati finiti} (FSM, \textit{Finite State Machine}).

Questo chiaramente porterà alla definizione di un determinato numero di \textbf{stati} in cui il sistema si potrà trovare in qualsiasi momento.
Azioni sul sistema corrisponderanno quindi a \textbf{transizioni} fra uno stato un altro, e ogni transizione comporterà un aggiornamento dell'uscita del sistema.
La risposta del sistema a ogni azione dall'esterno dipende quindi non solo dall'azione, ma dallo stato interno corrente del sistema, che è quindi dotato di \textbf{memoria}.

La modellizzazione con macchine a stati finiti risulta particolarmente utile in informatica, in quanto modellizza molto bene una vasta gamma di sistemi digitali.

\subsection{Modellizzazione a scatola bianca semplice}
Iniziamo a vedere come si possono modellizzare sistemi fisici attraverso approcci a scatola bianca. 

\subsubsection{Esempio: serbatoio a galleggiante}
Prendiamo il caso di un serbatoio, con in in ingresso un rubinetto di portata $p$, di cui si vuole tenere costante il livello del liquido ad un valore $\overline{h}$. 
Sistemi di questo tipo sono comuni in diversi contesti, fra cui gli sciaquoni dei water e le vaschette dei carburatori per motori a scoppio.

Chiamiamo il livello del liquido ad ogni istante temporale $h(t)$.
Il valore desiderato di tale funzione, che indichiamo come $h_o(t) = \overline{h}$, verrà detto \textbf{segnale di riferimento}.

L'ingresso del sistema su cui vogliamo agire sarà $u(t) = p(t)$, cioè la portata del rubinetto (che modificheremo ad esempio controllando una valvola).

Nel caso non ci siano disturbi, ergo tutto il liquido immesso nel serbatoio resti nel serbatoio, avremo una variazione di volume $\Delta V$ pari a:
$$
\Delta V = A \cdot (h_1 - h_0) = \int_{t_0}^{t_1} u(\tau) \, d\tau
$$
dove $A$ rappresenta l'area di base del serbatoio.

Da questo si ricava:
$$
A \cdot \frac{dh}{dt} = u(t)
$$

Chiamando $x \leftarrow h$ si otterrà l'equazione differenziale:
$$
x' = \frac{1}{A}u(t)
$$
Di cui chiaramente ci interessa la \textbf{condizione iniziale} $x_0$, cioè il livello del liquido all'istante temporale $t=0$.

\subsubsection{Esempio: circuito di carica di un condensatore}
Prendiamo il circuito formato da un condensatore $C$ posto in serie ad un generatore di corrente $i$:

\begin{center}
	\begin{circuitikz}
		\draw (0, 0) -- (3, 0)
			to[capacitor, v<=$V$] (3, -3)
			-- (0, -3)
			to[current source, i=$i$] (0, 0);
			
	\end{circuitikz}
\end{center}

Avremo che la variazione di carica sulle armature $\Delta q$ varrà:
$$
\Delta q = C \cdot (V_1 - V_0) = \int_{t_0}^{t_1} i(\tau) d \tau \implies C \frac{dV}{dt} = i(t)
$$
da cui si ricava l'equazione differenziale del tutto analoga a prima:
$$
x' = \frac{1}{C} i(t)
$$

L'unica cosa che cambia, rispetto al sistema precedente, sono i nomi delle variabili e le grandezze fisiche che rappresentano.
Dal punto di vista matematico, quindi, possiamo assumere i due sistemi come equivalenti.

\subsubsection{Esempio: legge di Newton}
Vediamo come equazioni analoghe alle precedenti si ricavano in verità da qualsiasi sistema meccanico che obbedisce alla legge di Newton:
$$
F = m a \implies m \frac{dv}{dt} = F(t)
$$
da cui:
$$
x' = \frac{1}{m}F(t)
$$
notando che chiamiamo $x \leftarrow v$ (ci riferiamo alle velocità, non alle posizioni).

\par\smallskip

Gli elementi visti finora non sono altro che \textbf{integratori}, cioè sistemi dinamici che prendono $x'$ e restituiscono $x$ per una certa variabile.

Indichiamo questi come segue:

\begin{center}
	\begin{tikzpicture}
		\draw (0,0) rectangle (1, 1);
		\draw[-stealth] (-2, 0.5) -> (0, 0.5);
		\draw[-stealth] (1, 0.5) -> (3, 0.5);
		\node at (0.5, 0.5) {$\int$};
		\node at (-1, 0.75) {$x'$};
		\node at (2, 0.7) {$x$};
	\end{tikzpicture}
\end{center}

e, se vogliamo, vedremo che nel dominio di Laplace vale la divisione per la variabile $s = \sigma + i \omega$:

\begin{center}
	\begin{tikzpicture}
		\draw (0,0) rectangle (1, 1);
		\draw[-stealth] (-2, 0.5) -> (0, 0.5);
		\draw[-stealth] (1, 0.5) -> (3, 0.5);
		\node at (0.5, 0.5) {$\frac{1}{s}$};
		\node at (-1, 0.75) {$x'$};
		\node at (2, 0.7) {$x$};
	\end{tikzpicture}
\end{center}

\subsection{Sistemi Lineari e Tempo Invarianti (LTI)}
Vediamo un particolare tipo di sistema dinamico:
\begin{definition}{Sistema LTI}
	Un sistema LTI (Lineare e Tempo Invariante) rispetta le seguenti proprietà:
	\begin{itemize}
		\item \textbf{Omogeneità:} se si scala l'ingresso $u(t)$, l'uscita $y(t)$ viene scalata dello stesso fattore:
			$$
				au(t) \implies ay(t)
			$$

		\item \textbf{Sovrapposizione degli effetti:} se un modello ha risposte $y_1(t)$ e $y_2(t)$ a due ingressi qualsiasi $u_1(t)$ e $u_2(t)$, allora la risposta a un ingresso dato dalla combinazione lineare di $u_1(t)$ e $u_2(t)$:
			$$
				u(t) = \alpha_1 u_1(t) + \alpha_2 u_2(t)
			$$

			sarà data ancora da una combinazione lineare delle uscite in risposta ai singoli ingressi:
			$$
				y(t) = \alpha_1 y_1(t) + \alpha_2 y_2(t)
			$$
			
			Notiamo che in questo la sovrapposizione degli effetti non rappresenta altro che la proprietà di \textit{linearità} dei sistemi lineari.
	
		\item \textbf{Tempo invarianza:} iil sistema si comporta allo stesso modo indipendentemente da quando avviene l'azione, cioè nelle equazioni che lo descrivono non vi è una dipendenza esplicita dal tempo.
			Questo significa che un ingresso $u(t - \tau)$ produce un uscita $y(t - \tau)$, cioè solo traslata in tempo:

		\begin{center}
			\begin{tikzpicture}
				\draw (0,0) rectangle (2, 1);
				\draw[-stealth] (-2, 0.5) -> (0, 0.5);
				\draw[-stealth] (2, 0.5) -> (4, 0.5);
				\node at (1, 0.5) {$\mathrm{LTI}$};
				\node at (-1, 0.7) {$u(t - \tau)$};
				\node at (3, 0.7) {$y(t - \tau)$};
			\end{tikzpicture}
		\end{center}
	\end{itemize}
\end{definition}

I sistemi lineari e tempo invarianti ci sono particolarmente utili in quanto sono completamente compresi e facili da risolvere, e riescono a modellizzare in maniera abbastanza accurata una vasta gamma di fenomeni.

\subsection{Modellizzazione a equazioni differenziali lineari}
Estendiamo l'approccio a scatola bianca introducendo formalmente una forma ad \textbf{equazioni differenziali lineari}.

\begin{definition}{Modello a variabili di stato}
	Introduciamo il modello per sistemi a equazioni differenziali lineari:
	\[
		\begin{cases}
			x' = Ax + Bu \\ 
			y = Cx + Du
		\end{cases}
	\]
	dove $u \in \mathbb{R}^r$ è l'\textbf{ingresso} e $x \in \mathbb{R}^n$ è lo \textbf{stato} del sistema.
	$x'$ sarà quindi la variazione dello stato, e $y \in \mathbb{R}^m$ l'uscita del sistema.
\end{definition}

Abbiamo quindi un sistema di equazioni vettoriali differenziali che mettono in relazione fra di loro lo \textit{stato interno presente} $x$, la variazione di stato $x'$, l'ingresso $u$ e l'uscita $y$ del sistema.

Le dimensioni delle matrici che correlano variazione di stato e uscita a stato interno presente e ingresso, cioè le matrici $A, B, C, D$, si ricaveranno dalle dimensioni delle variabili ($r, n$ e $m$):

$$
A \in \mathbb{R}^{n \times n}, \, 
B \in \mathbb{R}^{n \times r}, \, 
C \in \mathbb{R}^{m \times n}, \, 
D \in \mathbb{R}^{m \times r}, \, 
$$

La tempo invarianza del sistema è data proprio dalla tempo invarianza delle matrici, che non hanno il tempo come argomento ma sono costanti.

\subsubsection{Proprietà strutturali}
L'analisi delle matrici $A, B, C, D$ definisce le cosiddette \textbf{proprietà strutturali} del sistema.
Ad esempio abbiamo che:
\begin{itemize}
	\item La matrice $A$ ci dà informazioni riguardo alla proprietà di \textbf{stabilità} del sistema;
	\item Le matrici $A, B$ ci danno informazioni riguardo alla \textbf{controllabilità} del sistema;
	\item Le matrici $A, C$ ci danno informazioni riguardo all'\textbf{osservabilità} del sistema.
\end{itemize}

\subsubsection{Diagramma a blocchi}
Possiamo usare l'operatore integrale definito prima per creare il diagramma a blocchi del sistema a equazioni differenziali linari:

\begin{center}
	\begin{tikzpicture}
		% integratore
		\draw (0,0) rectangle (1, 1);

		% A 
		\draw (0,-1) rectangle (1, -2);
		
		% D
		\draw (0,-3) rectangle (1, -4);
		
		% B 
		\draw (-5,0) rectangle (-4, 1);
		
		% C 
		\draw (3,0) rectangle (4, 1);

		\draw[-stealth] (-2, 0.5) -> (0, 0.5);
		\draw[-stealth] (1, 0.5) -> (3, 0.5);
		
		\draw[-stealth] (-7, 0.5) -> (-5, 0.5);
		\draw[-stealth] (-4, 0.5) -> (-2.1, 0.5);
		
		\draw[-stealth] (4, 0.5) -> (5.9, 0.5);
		\draw[-stealth] (6, 0.5) -> (8, 0.5);

		\draw (2, 0.5) -> (2, -1.5);
		\draw[-stealth] (2, -1.5) -> (1, -1.5);

		\draw (0, -1.5) -> (-2, -1.5);
		\draw[-stealth] (-2, -1.5) -> (-2, 0.4);


		\draw (-6, 0.5) -> (-6, -3.5);
		\draw (-6, -3.5) -> (0, -3.5);

		\draw (1, -3.5) -> (6, -3.5);
		\draw[-stealth] (6, -3.5) -> (6, 0.4);


		\node at (0.5, 0.5) {$\int$};
		\node at (0.5, -1.5) {$A$};
		\node at (0.5, -3.5) {$D$};
		
		\node at (-4.5, 0.5) {$B$};
		\node at (3.5, 0.5) {$C$};

		\node at (-6, 0.75) {$u$};
		\node at (7, 0.75) {$y$};
		
		\node at (-1, 0.75) {$x'$};
		\node at (2, 0.7) {$x$};
	
		% sommatore B - integratore
		\draw[fill=white] (-2, 0.5) circle (0.1);
		
		% sommatore C - uscita 
		\draw[fill=white] (6, 0.5) circle (0.1);

	\end{tikzpicture}
\end{center}

Notando che il simbolo $\circ$ rappresenta una somma.

\subsubsection{Esempio: partitore di tensione}
Vediamo per il modello introdotto l'esempio del \textit{partitore resistivo} o partitore di tensione.

\begin{center}
	\begin{circuitikz}
		\draw (0,0) -- (2,0)
			to[resistor, R=$R_1$] (2,-2)
			to[resistor, R=$R_2$] (2,-4) node[ground]{};

		\draw (2, -2) -- (4, -2);

		\node at (-0.5, 0) {$V_{in}$};
		\node at (4.5, -2) {$V_{out}$};
	\end{circuitikz}
\end{center}

La relazione che otteniamo, dalla legge del partitore di tensione, per il voltaggio, è:
$$
V_{out} = \frac{R_2}{R_1 + R_2} V_{in} = \alpha V_{in}, \quad \alpha = \frac{R_1}{R_1 + R_2}
$$

da cui:
$$
A = B = C = 0, \ D = \alpha 
$$ # rivedi

\subsubsection{Esempio: sistema massa-molla-smorzatore}
Vediamo il sistema fisico dato da un carrello di massa $M$ legato ad una parete da una molla con costante $K$ e sottoposto ad attrito proporzionale alla sua velocità $v$.
Agiamo sul sistema introducendo una \textit{forza} $F$ sul carrello, che consideriamo come \textbf{entrata} del sistema.

Il sistema verrà descritto dall'equazione:
$$
F(t) = M \frac{dv}{dt} + Bv + Kx
$$

Decidiamo di usare come \textbf{variabili di stato} la \textit{posizione} $x$ e la \textit{velocità} $v$: $x_s = \binom{x}{v}$.
Vedremo che questo accade spesso nel caso di sistemi meccanici.
Avremo quindi la variazione di stato:
$$
	\begin{cases}		
		\frac{dx}{dt} = v \\
		\frac{dv}{dt} = -\frac{K}{M}x - \frac{B}{M}v + \frac{1}{M}F
	\end{cases}
$$

Avremo allora la prima equazione del sistema:
$$
\begin{pmatrix}
	\frac{dx}{dt} \\ 
	\frac{dv}{dt}
\end{pmatrix}
=
\begin{pmatrix}
	0 & 1 \\ 
	-\frac{K}{M} & -\frac{B}{M}
\end{pmatrix}
+
\begin{pmatrix}
	0 \\ 
	\frac{1}{M}
\end{pmatrix}
F
$$
da cui le matrici $A, B$:
$$
A = 
\begin{pmatrix}
	0 & 1 \\ 
	-\frac{K}{M} & -\frac{B}{M}
\end{pmatrix}, \quad 
B = 
\begin{pmatrix}
	0 \\ 
	\frac{1}{M}
\end{pmatrix}
$$

Scegliamo quindi come \textbf{uscita} la \textit{posizione} $x$, come:
$$
y = x = 
\begin{pmatrix}
	1 & 0
\end{pmatrix}
 \binom{x}{v}
$$
da cui le matrici $C, D$:
$$
C =
\begin{pmatrix}
	1 & 0
\end{pmatrix}, \quad 
D =
\begin{pmatrix}
0
\end{pmatrix}
$$ 
# qui riguarda bene

Notiamo che si potrebbe scegliere, analogamente, la v\textit{velocità} $v$ come \textbf{uscita}, da cui si otterrebbe:
$$
y = v = 
\begin{pmatrix}
	0 & 1
\end{pmatrix}
 \binom{x}{v}
$$
da cui le matrici $C, D$:
$$
C =
\begin{pmatrix}
	0 & 1
\end{pmatrix}, \quad 
D =
\begin{pmatrix}
0
\end{pmatrix}
$$

\par\medskip 

Abbiamo quindi visto come il modello a variabili di stato permette di rappresentare una vasta gamma di situazionifisiche con facilità e secondo un formato standard, con cui sono tra l'altro compatibili diversi sistemi di calcolo e simulazione al calcolatore.

\subsubsection{Limiti dei sistemi lineari}
Notiamo come il mondo reale spesso è \textbf{non lineare}.
Prendiamo il caso del serbatoio visto prima, e assumiamo la sezione $A$ come non costante lungo l'altezza $h$, ergo esiste una funzione $A(h)$ che lega altezza a sezione.
Avremo quindi la relazione:
$$
h' = B(h) u, \quad B(h) = \frac{1}{A_h} \implies A(h) \, dh = u \, dt
$$

Al momento di apertura della valvola di uscita avremo quindi una variazione del livello:
$$
h' = B(h)(u - y) = \frac{u - y}{A(h)}
$$

Assunta \textit{sezione} in variazione lineare, potremo sfruttare la legge:
$$
y = k\sqrt(h)
$$
da cui:
$$
h' = \frac{-k\sqrt(h)}{A(h)} + \frac{u}{A(h)}
$$

Notiamo come questa forma si avvicina al modello standard visto prima, ma presenta una relazinoe non lineare all'altezza $h$.
Indichiamo quindi questa relazione come:
$$
h' = g(h) + f(h) u
$$

Chiamiamo questa forma \textbf{forma bilineare}.
\begin{definition}{Forma bilineare}
	Introduciamo la forma:
	\[
		\begin{cases}
			x' = f(x) + g(x) u \\ 
			y = h(x)
		\end{cases}
	\]
\end{definition}


\end{document}

\end{document}