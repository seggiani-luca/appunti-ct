
\documentclass[a4paper,11pt]{article}
\usepackage[a4paper, margin=8em]{geometry}

% usa i pacchetti per la scrittura in italiano
\usepackage[french,italian]{babel}
\usepackage[T1]{fontenc}
\usepackage[utf8]{inputenc}
\frenchspacing 

% usa i pacchetti per la formattazione matematica
\usepackage{amsmath, amssymb, amsthm, amsfonts}

% usa altri pacchetti
\usepackage{gensymb}
\usepackage{hyperref}
\usepackage{standalone}

% imposta il titolo
\title{Appunti Fondamenti di Automatica}
\author{Luca Seggiani}
\date{2025}

% disegni
\usepackage{pgfplots}
\pgfplotsset{width=10cm,compat=1.9}

% imposta lo stile
% usa helvetica
\usepackage[scaled]{helvet}
% usa palatino
\usepackage{palatino}
% usa un font monospazio guardabile
\usepackage{lmodern}

% tikz in sans
\tikzset{every picture/.style={/utils/exec={\sffamily}}}

\renewcommand{\rmdefault}{ppl}
\renewcommand{\sfdefault}{phv}
\renewcommand{\ttdefault}{lmtt}

% circuiti
\usepackage{circuitikz}
\usetikzlibrary{babel}

% disponi il titolo
\makeatletter
\renewcommand{\maketitle} {
	\begin{center} 
		\begin{minipage}[t]{.8\textwidth}
			\textsf{\huge\bfseries \@title} 
		\end{minipage}%
		\begin{minipage}[t]{.2\textwidth}
			\raggedleft \vspace{-1.65em}
			\textsf{\small \@author} \vfill
			\textsf{\small \@date}
		\end{minipage}
		\par
	\end{center}

	\thispagestyle{empty}
	\pagestyle{fancy}
}
\makeatother

% disponi teoremi
\usepackage{tcolorbox}
\newtcolorbox[auto counter, number within=section]{theorem}[2][]{%
	colback=blue!10, 
	colframe=blue!40!black, 
	sharp corners=northwest,
	fonttitle=\sffamily\bfseries, 
	title=Teorema~\thetcbcounter: #2, 
	#1
}

% disponi definizioni
\newtcolorbox[auto counter, number within=section]{definition}[2][]{%
	colback=red!10,
	colframe=red!40!black,
	sharp corners=northwest,
	fonttitle=\sffamily\bfseries,
	title=Definizione~\thetcbcounter: #2,
	#1
}

% disponi problemi
\newtcolorbox[auto counter, number within=section]{problem}[2][]{%
	colback=green!10,
	colframe=green!40!black,
	sharp corners=northwest,
	fonttitle=\sffamily\bfseries,
	title=Problema~\thetcbcounter: #2,
	#1
}

% disponi codice
\usepackage{listings}
\usepackage[table]{xcolor}

\lstdefinestyle{codestyle}{
		backgroundcolor=\color{black!5}, 
		commentstyle=\color{codegreen},
		keywordstyle=\bfseries\color{magenta},
		numberstyle=\sffamily\tiny\color{black!60},
		stringstyle=\color{green!50!black},
		basicstyle=\ttfamily\footnotesize,
		breakatwhitespace=false,         
		breaklines=true,                 
		captionpos=b,                    
		keepspaces=true,                 
		numbers=left,                    
		numbersep=5pt,                  
		showspaces=false,                
		showstringspaces=false,
		showtabs=false,                  
		tabsize=2
}

\lstdefinestyle{shellstyle}{
		backgroundcolor=\color{black!5}, 
		basicstyle=\ttfamily\footnotesize\color{black}, 
		commentstyle=\color{black}, 
		keywordstyle=\color{black},
		numberstyle=\color{black!5},
		stringstyle=\color{black}, 
		showspaces=false,
		showstringspaces=false, 
		showtabs=false, 
		tabsize=2, 
		numbers=none, 
		breaklines=true
}

\lstdefinelanguage{javascript}{
	keywords={typeof, new, true, false, catch, function, return, null, catch, switch, var, if, in, while, do, else, case, break},
	keywordstyle=\color{blue}\bfseries,
	ndkeywords={class, export, boolean, throw, implements, import, this},
	ndkeywordstyle=\color{darkgray}\bfseries,
	identifierstyle=\color{black},
	sensitive=false,
	comment=[l]{//},
	morecomment=[s]{/*}{*/},
	commentstyle=\color{purple}\ttfamily,
	stringstyle=\color{red}\ttfamily,
	morestring=[b]',
	morestring=[b]"
}

% disponi sezioni
\usepackage{titlesec}

\titleformat{\section}
	{\sffamily\Large\bfseries} 
	{\thesection}{1em}{} 
\titleformat{\subsection}
	{\sffamily\large\bfseries}   
	{\thesubsection}{1em}{} 
\titleformat{\subsubsection}
	{\sffamily\normalsize\bfseries} 
	{\thesubsubsection}{1em}{}

% disponi alberi
\usepackage{forest}

\forestset{
	rectstyle/.style={
		for tree={rectangle,draw,font=\large\sffamily}
	},
	roundstyle/.style={
		for tree={circle,draw,font=\large}
	}
}

% disponi algoritmi
\usepackage{algorithm}
\usepackage{algorithmic}
\makeatletter
\renewcommand{\ALG@name}{Algoritmo}
\makeatother

% disponi numeri di pagina
\usepackage{fancyhdr}
\fancyhf{} 
\fancyfoot[L]{\sffamily{\thepage}}

\makeatletter
\fancyhead[L]{\raisebox{1ex}[0pt][0pt]{\sffamily{\@title \ \@date}}} 
\fancyhead[R]{\raisebox{1ex}[0pt][0pt]{\sffamily{\@author}}}
\makeatother

\begin{document}

% sezione (data)
\section{Lezione del 12-03-25}

% stili pagina
\thispagestyle{empty}
\pagestyle{fancy}

% testo
\subsection{Forma minima}
Abbiamo studiato finora sistemi modellizzati attraverso \textit{variabili di stato}, espressi come:
\[
	\begin{cases}
		x' = Ax + Bu \\ 
		y = Cx + Du
	\end{cases}
\]

Di questi, abbiamo che:
\begin{itemize}
	\item La \textbf{stabilità} dipende da $A$, e in particolare dai suoi autovalori;
	\item La \textbf{raggiungibilità} dipende da $A$ e $B$, e in particolare dal rango della matrice $\mathcal{M}_\mathcal{R}$ che se ne ricava. Abbiamo visto che le variabili non raggiungibili possono essere esplicitate attraverso la matrice di trasformazione $T_r$;
	\item L'\textbf{osservabilità} dipende da $A$ e $C$, e in particolare dal rango della matrice $\mathcal{M}_\mathcal{O}$ che se ne ricava. Anche qui, abbiamo visto che le variabili non osservabili possono essere esplicitate attraverso la matrice di trasformazione $T_o$.
\end{itemize}

Infine, avevamo detto che un sistema può essere stabile, raggiungibile e osservabili, nessuna di queste o una loro combinazione.
Per evidenziare queste caratteristiche avevamo introdotto la \textbf{forma canonica} di Kalman.

Ripartiamo da qui per introdurre i sistemi in \textbf{forma minima}:
\begin{definition}{Forma minima}
	Un sistema si dice in forma minima se è completamente osservabile e completamente raggiungibile.
\end{definition}

Questo significa che non è possibile usare un numero di variabili di stato minore del suo ordine per descrivere la relazione ingresso-uscita (movimento forzato).

Le parti non raggiungibili e non osservabili non rappresentano quindi questa relazione, anche se possono essere comunque importanti per lo studio del movimento libero (ad esempio, per la stabilità).

\subsection{Metodi di ispezione diretta}
Iniziamo a vedere i metodi di \textbf{ispezione diretta} per raggiungibilità e osservabilità.
Questi sono applicabili in casi particolari dove la struttura delle matrici $A$ e $B$ ci permette di capire direttamente la raggiungibilità del sistema.

\subsubsection{Ispezione diretta di raggiungibilità}
Iniziamo col metodo di ispezione diretta di raggiungibilità, presentando prima il caso SISO con matrici $A$ diagonali e generalizzandolo a sistemi MIMO con matrici $A$ arbitrarie.

\begin{itemize}
	\item \textbf{Caso SISO diagonale}: poniamo che il sistema sia a ingresso e uscita singola, e la matrice $A$ sia diagonale, cioè:
		$$
			A = \begin{pmatrix}
				\lambda_1 & ... & 0 \\
				& ... & \\ 
				0 & ... & \lambda_n
			\end{pmatrix}, \quad B = \begin{pmatrix}
				b_1 \\ ... \\ b_n
			\end{pmatrix}
		$$
		In questo caso $\mathcal{M}_\mathcal{R}$ sarà:
		$$
		\mathcal{M}_\mathcal{R} = \begin{pmatrix}
			b_1 & \lambda_1 b_1 & ... & \lambda_1^{n - 1} b_1 \\
			... \\
			b_n & \lambda_n b_n & ... & \lambda_n^{n - 1} b_n
		\end{pmatrix}
		$$
		cioè la condizione di $\mathrm{rank}(\mathcal{M}_\mathcal{R}) = n$ sarà $B$ a elementi non nulli e $\lambda_i$ distinti.

		Possiamo interpretare questo caso come quello dove ogni variabile di stato $x_i$ è indipendente dalle altre: in questo ogni dimensione dello spazio di stato rappresenterà effettivamente un sistema a sé, e quello che vorremo sarà che i $\lambda_i$ della risposta di ogni \textit{sottosistema} siano distinti (in modo da poterli distinguere), e che l'ingresso arrivi ad ogni sottosistema con un $b_i \neq 0$, cioè la variabile $i$-esima possa effettivamente esserne influenzata.
	\item \textbf{Caso MIMO in forma di Jordan}: cerchiamo di generalizzare quanto visto per il caso SISO a sistemi a più variabili, con matrici $A$ non necessariamente diagonali.

		Sfrutteremo adesso il teorema:
		\begin{theorem}{Lemma di Popov-Belevitch-Hautus (PBH) ragg.}
			Il sistema dinamico LTI $x' = Ax + Bu$ è completamente raggiungibile se e solo se $\mathrm{rank}(\lambda I - A \, | \, B) = n$, $\forall \lambda \in \mathbb{C}$.
		\end{theorem}
		La dimensione della matrice ottenuta sara $n \times (n + m)$, in quanto sia $A$ che $B$ hanno $n$ righe, $A$ ha $n$ colonne e $B$ ne ha $m$.

		Abbiamo quindi che con $\lambda$ non autovalore, la condizione è sempre verificata (in quanto $\det(\lambda I - A) \neq 0$, altrimenti si viola la definizione di autovalore).
		Nel caso in cui invece $\lambda$ è autovalore, la condizione deve essere verificata dall'aggiunta di $B$ (in quanto $\det(\lambda I - A) < n$).

		Per dimostrare questo teorema assumiamo che $\lambda_i$ tale per cui $\det(\lambda_i I - A \, | \, B) < n$. 
		Allora $\exists q \neq 0$ tale che:
		$$
			q^T (\lambda_i I - A \, | \, B) = 0
		$$
		cioè $q \in \mathrm{ker}(\lambda_i I - A \, | \, B)$ \textit{nullo sinistro} (si pensi alla definizione di indipendenza lineare).
		Da questo si può dividere il prodotto in:
		$$
		q^T (\lambda_i - A) = 0, \quad q^T B = 0
		$$
		Dalla prima, si ha, moltiplicando per $B$:
		$$
		q^T \lambda_i = q^T A \implies q^T A B = \lambda_i q^T B = 0
		$$
		quindi $q^T A B = 0$.
		Potremo moltiplicare, anziché per $B$, anche per $AB$, $A^2 B$, ecc... e trovare sempre $q^T A^j B = 0$, e quindi:
		$$
		q^T \begin{pmatrix}
			B & AB & ... & A^{n - 1} B
		\end{pmatrix} = 0
		$$
		cioè la matrice di raggiungibilità $\mathcal{M}_\mathcal{R}$ non ha rango massimo e il sistema non è completamente raggiungibile. \qed

		Abbiamo quindi che nel caso generico MIMO, la matrice $A$ è in forma di Jordan con $p$ blocchi ($p \neq$ numero di uscite) di dimensioni $m_i$, cioè:		
		$$
			A = \begin{pmatrix}
				\text{blocco}_1 & ... & 0 \\
												& ... & \\
				0 & ... & \text{blocco}_p
			\end{pmatrix}, \quad B = \begin{pmatrix}
				b_{11} \\ ... \\ b_{1m_1} \\ ... \\ b_{p1} \\ ... \\ b_{pm_1}
			\end{pmatrix}
		$$
		con:
		$$
			\text{blocco}_i = \begin{pmatrix}
				\lambda_i & ... & 0 \\
				& ... & \\ 
				0 & ... & \lambda_i
			\end{pmatrix}
		$$
		Se il sistema fosse SISO, la molteplicità geometrica degli autovalori sarebbe uguale a 1, e $B$ dovrebbe avere tanti elementi diversi da zero almeno quanti gli autovalori distinti in $A$.
	Di contro, un sistema con $\mu$ miniblocchi associati ad un unico autovalore $\lambda$ può essere raggiungibile solo se ha almeno $\mu$ ingressi (elementi $\neq 0$ di $B$).
\end{itemize}

\subsubsection{Ispezione diretta di osservabilità}
Esiste una variante del lemma PBH per l'osservabilità:
		\begin{theorem}{Lemma di Popov-Belevitch-Hautus (PBH) oss.}
			Il sistema dinamico LTI $x' = Ax + Bu$ è completamente osservabile se e solo se: $$\mathrm{rank}\begin{pmatrix}
			\lambda I - A \\ B
			\end{pmatrix} = n, \quad \forall \lambda \in \mathbb{C}$$
		\end{theorem}
		
		\begin{itemize}
			\item \textbf{Caso MIMO in forma di Jordan:} riprendiamo direttamente il caso MIMO. Stavolta le matrici saranno:
		$$
			A = \begin{pmatrix}
				\text{blocco}_1 & ... & 0 \\
												& ... & \\
				0 & ... & \text{blocco}_p
			\end{pmatrix}, \quad C = \begin{pmatrix}
				b_{11} & ... & b_{1m_1} \, | & ... & | \, b_{p1} & ... &&b_{pm_1}
			\end{pmatrix}
		$$
		con:
		$$
			\text{blocco}_i = \begin{pmatrix}
				\lambda_i & ... & 0 \\
				& ... & \\ 
				0 & ... & \lambda_i
			\end{pmatrix}
		$$
		Vorremmo imporre le stesse condizioni di prima, cioè per $\mu$ miniblocchi associati ad un unico autovalore $\lambda$ vogliamo almeno $\mu$ uscite (elementi $\neq 0$ di $C$).

		\end{itemize}

\subsection{Funzione di trasferimento}
Introduciamo adesso dei metodi che evitano di sfruttare direttamente le variabili di stato per rappresentare sistemi dinamici.
Utilizziamo a questo a scopo la \textbf{funzione di trasferimento}.

Si noti che è sempre possibile passare dalla forma a variabili di stato alla forma a funzione di trasferimento, cioè queste sono intercambiabili e differsicono solo per la semplicità dei calcoli.

La funzione di trasferimento $F$ di un sistema dinamico nella variabile $s$ è il \textbf{rapporto} fra l'\textit{uscita} $Y$ e l'\textit{ingresso} $U$:
$$
F(s) = \frac{\text{uscita}}{\text{ingresso}} = \frac{Y(s)}{U(s)}
$$

Possiamo rappresentare anche il diagramma a blocchi:
\begin{center}
	\begin{tikzpicture}
		\draw (0,0) rectangle (2, 1);
		\draw[-stealth] (-2, 0.5) -> (0, 0.5);
		\draw[-stealth] (2, 0.5) -> (4, 0.5);
		\node at (1, 0.5) {$F(S)$};
		\node at (-1, 0.75) {$U(S)$};
		\node at (3, 0.75) {$Y(S)$};
	\end{tikzpicture}
\end{center}

Questo diagramma rappresenta in particolare il sistema rappresentato dalla funzione $F(S)$, posto in \textbf{catena aperta}.
Vedremo fra poco sistemi dove la variabile di uscita $Y(S)$ è chiusa in retroazione sulla variabile di ingresso $U(S)$, cioè sistemi in \textbf{catena chiusa}.

\subsection{Controllo in feedback}
Vediamo quindi nel dettaglio il modello di controllo a catena chiusa più popolare: quello del \textbf{controllo in feedback}, o \textit{controllo in retroazione}.

Il diagramma a blocchi avrà in questo caso l'aspetto:
\begin{center}
	\begin{tikzpicture}
		\draw (1,0) rectangle (3, 1);
		\node at (2, 0.5) {impianto};
		
		\draw (-3,0) rectangle (-1, 1);
		\node at (-2, 0.5) {controllore};

		\draw (-1.5, -0.5) rectangle (0.5, -1.5);
		\node at (-0.5, -1) {sensore};

		\draw[-stealth] (-7, 0.5) -> (-5.1, 0.5);
		\draw[-stealth] (-5, 0.5) -> (-3, 0.5);
		\draw[-stealth] (-1, 0.5) -> (1, 0.5);
		\draw[-stealth] (3, 0.5) -> (3.9, 0.5);
		\draw[-stealth] (4, 0.5) -> (7, 0.5);
		
		\draw (5, 0.5) -> (5, -1);
		\draw[-stealth] (5, -1) -> (0.5, -1);
		\draw (-1.5, -1) -> (-5, -1);
		\draw[-stealth] (-5, -1) -> (-5, 0.5);

		\draw[-stealth] (4, 1.5) -> (4, 0.6);
		\node at (4, 1.75) {disturbo};

		\draw[fill=white] (-5, 0.5) circle (0.1);
		\draw[fill=white] (4, 0.5) circle (0.1);

		\node at (-6, 0.75) {ingresso};
		\node at (6, 0.75) {uscita};

		\node at (-4.75, 0.75) {$+$};
		\node at (-4.75, 0.25) {$-$};

		\node at (-3.8, 0.75) {\textit{errore}};
		\node at (0, 0.75) {\textit{controllo}};
		\node at (-3.5, -1.5) {\textit{feedback}};
	\end{tikzpicture}
\end{center}

L'idea fondamentale è che il sensore \textit{rileva} l'effetto del controllo sull'impianto, e quindi la variabile di uscita, e lo usa per correggere (tramite il \textit{feedback}) il controllo stesso.

\end{document}
