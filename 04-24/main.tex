
\documentclass[a4paper,11pt]{article}
\usepackage[a4paper, margin=8em]{geometry}

% usa i pacchetti per la scrittura in italiano
\usepackage[french,italian]{babel}
\usepackage[T1]{fontenc}
\usepackage[utf8]{inputenc}
\frenchspacing 

% usa i pacchetti per la formattazione matematica
\usepackage{amsmath, amssymb, amsthm, amsfonts}

% usa altri pacchetti
\usepackage{gensymb}
\usepackage{hyperref}
\usepackage{standalone}

% imposta il titolo
\title{Appunti Fondamenti di Automatica}
\author{Luca Seggiani}
\date{2025}

% disegni
\usepackage{pgfplots}
\pgfplotsset{width=10cm,compat=1.9}

% imposta lo stile
% usa helvetica
\usepackage[scaled]{helvet}
% usa palatino
\usepackage{palatino}
% usa un font monospazio guardabile
\usepackage{lmodern}

% tikz in sans
\tikzset{every picture/.style={/utils/exec={\sffamily}}}

\renewcommand{\rmdefault}{ppl}
\renewcommand{\sfdefault}{phv}
\renewcommand{\ttdefault}{lmtt}

% circuiti
\usepackage{circuitikz}
\usetikzlibrary{babel}

% disponi il titolo
\makeatletter
\renewcommand{\maketitle} {
	\begin{center} 
		\begin{minipage}[t]{.8\textwidth}
			\textsf{\huge\bfseries \@title} 
		\end{minipage}%
		\begin{minipage}[t]{.2\textwidth}
			\raggedleft \vspace{-1.65em}
			\textsf{\small \@author} \vfill
			\textsf{\small \@date}
		\end{minipage}
		\par
	\end{center}

	\thispagestyle{empty}
	\pagestyle{fancy}
}
\makeatother

% disponi teoremi
\usepackage{tcolorbox}
\newtcolorbox[auto counter, number within=section]{theorem}[2][]{%
	colback=blue!10, 
	colframe=blue!40!black, 
	sharp corners=northwest,
	fonttitle=\sffamily\bfseries, 
	title=Teorema~\thetcbcounter: #2, 
	#1
}

% disponi definizioni
\newtcolorbox[auto counter, number within=section]{definition}[2][]{%
	colback=red!10,
	colframe=red!40!black,
	sharp corners=northwest,
	fonttitle=\sffamily\bfseries,
	title=Definizione~\thetcbcounter: #2,
	#1
}

% disponi problemi
\newtcolorbox[auto counter, number within=section]{problem}[2][]{%
	colback=green!10,
	colframe=green!40!black,
	sharp corners=northwest,
	fonttitle=\sffamily\bfseries,
	title=Problema~\thetcbcounter: #2,
	#1
}

% disponi codice
\usepackage{listings}
\usepackage[table]{xcolor}

\lstdefinestyle{codestyle}{
	backgroundcolor=\color{black!5}, 
	commentstyle=\color{codegreen},
	keywordstyle=\bfseries\color{magenta},
	numberstyle=\sffamily\tiny\color{black!60},
	stringstyle=\color{green!50!black},
	basicstyle=\ttfamily\footnotesize,
	breakatwhitespace=false,         
	breaklines=true,                 
	captionpos=b,                    
	keepspaces=true,                 
	numbers=left,                    
	numbersep=5pt,                  
	showspaces=false,                
	showstringspaces=false,
	showtabs=false,                  
	tabsize=2
}

\lstdefinestyle{shellstyle}{
	backgroundcolor=\color{black!5}, 
	basicstyle=\ttfamily\footnotesize\color{black}, 
	commentstyle=\color{black}, 
	keywordstyle=\color{black},
	numberstyle=\color{black!5},
	stringstyle=\color{black}, 
	showspaces=false,
	showstringspaces=false, 
	showtabs=false, 
	tabsize=2, 
	numbers=none, 
	breaklines=true
}

\lstdefinelanguage{javascript}{
	keywords={typeof, new, true, false, catch, function, return, null, catch, switch, var, if, in, while, do, else, case, break},
	keywordstyle=\color{blue}\bfseries,
	ndkeywords={class, export, boolean, throw, implements, import, this},
	ndkeywordstyle=\color{darkgray}\bfseries,
	identifierstyle=\color{black},
	sensitive=false,
	comment=[l]{//},
	morecomment=[s]{/*}{*/},
	commentstyle=\color{purple}\ttfamily,
	stringstyle=\color{red}\ttfamily,
	morestring=[b]',
	morestring=[b]"
}

% disponi sezioni
\usepackage{titlesec}

\titleformat{\section}
{\sffamily\Large\bfseries} 
{\thesection}{1em}{} 
\titleformat{\subsection}
{\sffamily\large\bfseries}   
{\thesubsection}{1em}{} 
\titleformat{\subsubsection}
{\sffamily\normalsize\bfseries} 
{\thesubsubsection}{1em}{}

% disponi alberi
\usepackage{forest}

\forestset{
	rectstyle/.style={
		for tree={rectangle,draw,font=\large\sffamily}
	},
	roundstyle/.style={
		for tree={circle,draw,font=\large}
	}
}

% disponi algoritmi
\usepackage{algorithm}
\usepackage{algorithmic}
\makeatletter
\renewcommand{\ALG@name}{Algoritmo}
\makeatother

% disponi numeri di pagina
\usepackage{fancyhdr}
\fancyhf{} 
\fancyfoot[L]{\sffamily{\thepage}}

\makeatletter
\fancyhead[L]{\raisebox{1ex}[0pt][0pt]{\sffamily{\@title \ \@date}}} 
\fancyhead[R]{\raisebox{1ex}[0pt][0pt]{\sffamily{\@author}}}
\makeatother

\begin{document}

% sezione (data)
\section{Lezione del 24-04-25}

% stili pagina
\thispagestyle{empty}
\pagestyle{fancy}

% testo
Riprendiamo la trattazione delle regole di tracciamento del luogo delle radici.

\subsubsection{Regole di tracciamento 3}
\begin{enumerate}
	\item[8.] Come anticipato, la regola (8) è in qualche modo duale alla (7), e riguarda gli angoli dei rami entranti negli zeri.

		Riprendiamo la condizione di fase:
		\[
			\begin{cases}
				\angle n(s) - \angle d(s) = -\pi \pm 2 h \pi, \quad K > 0 \\
				\angle n(s) - \angle d(s) = \pm 2 h \pi, \quad K < 0
			\end{cases}
		\]
		E notiamo che preso uno zero particolare in $z^*$, ad esempio riguardo al luogo diretto, si avrà che l'angolo di entrata $\theta^*$ è:
		$$
		\angle n^*(s) + \theta^* - \angle d(s) \Big|_{s = z^*} = -\pi \implies \theta^* = -\pi + \angle d(s) - \angle n^*(s) \Big|_{s = z^*}
		$$
		dove $n^*(s)$ è il numeratore rimosso lo zero $z^*$ preso in considerazione.

		Potremo quindi usare questa formula generale per calcolare l'angolo di entrata dei rami negli zeri.
		Da questo ricaviamo fra l'altro che i rami entranti nei poli con molteplicità $> 1$ dividono il piano in parti equiangole e simmetriche rispetto all'asse reale.

		\par\medskip
		\noindent
		\textbf{\sffamily{Esempio}}
		
		Calcoliamo gli angoli di entrata negli zeri della funzione di trasferimento:
		$$
		G(s) = \frac{s^2 + 20s + 101}{s (s + 5)^3}
		$$
		Partiamo sempre col trovare zeri e poli, che saranno:
		$$
		z_{1, 2} = -10 \pm i, \quad p_1 = 0, \quad p_{2, 3, 4} = -5
		$$
		per cui potremo fattorizzare la funzione di trasferimento:
		$$
		G(s) = \frac{ (s + 10 - i)(s + 10 + i) }{ s (s + 5)^3 }
		$$

		Calcoliamo allora gli angoli veri e propri per ogni zero, applicando la legge appena trovata:
		\begin{itemize}
			\item Per lo zero $z_1$ si ha:
				$$
					\theta_1 = - \pi + \angle d(s) - \angle n^1(s) \Big|_{s = z_1} = -\pi + \angle(10 + i) + 3 \cdot \angle (-5 + i) - \angle  2i 
				$$
				$$
				= -180^\circ + 174.3^\circ + 506.07^\circ -90^\circ = 50.37^\circ
				$$
			\item Per lo zero $z_2$, complesso coniugato allo $z_1$, ci aspettiamo di ottenere l'opposto di $\theta_1$. Verifichiamo:
				$$
					\theta_2 = - \pi + \angle d(s) - \angle n^2(s) \Big|_{s = z_2} = -\pi + \angle(-10 - i) + 3 \cdot \angle (-5 - i) - \angle -2i 
				$$
				$$
				= -180^\circ - 174.3^\circ - 506.07^\circ +90^\circ = -50.37^\circ
				$$
		\end{itemize}

\end{enumerate}

\subsubsection{Riassunto sul luogo delle radici}
Abbiamo quindi che il luogo delle radici è una tecnica per determinare lo spostamento dei poli dal ciclo aperto al ciclo chiuso: inserito un controllore proporzionale di guadagno $K$, abbiamo ch al variare di $K$ i poli in ciclo chiuso cambiano posizione, appunto quelle che figurano nel luogo delle radici.

Il procedimento che seguiremo nel progetto dei controllori sarà quindi:
\begin{itemize}
	\item Fissare i poli/zeri del regolatore sulla base delle specifiche date;
	\item Variare $K$ per individuare il valore che fissa i \textit{poli dominanti} del sistema nella regione opportuna, sulla base delle specifiche, del piano complesso.
\end{itemize}

\end{document}
